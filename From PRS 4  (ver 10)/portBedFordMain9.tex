%2multibyte Version: 5.50.0.2960 CodePage: 65001
\documentclass[titlepage,fleqn]{article}%
\usepackage{eurosym}
\usepackage{amssymb}
\usepackage{amsfonts}
\usepackage{amsmath}
\usepackage{appendix}
\usepackage{rotating}
\usepackage{enumerate}
\usepackage{graphicx}
\usepackage{float}%
\setcounter{MaxMatrixCols}{30}
%TCIDATA{OutputFilter=latex2.dll}
%TCIDATA{Version=5.50.0.2960}
%TCIDATA{Codepage=65001}
%TCIDATA{CSTFile=40 LaTeX article.cst}
%TCIDATA{Created=Thursday, August 18, 2016 12:12:28}
%TCIDATA{LastRevised=Thursday, September 14, 2017 15:53:31}
%TCIDATA{<META NAME="GraphicsSave" CONTENT="32">}
%TCIDATA{<META NAME="SaveForMode" CONTENT="1">}
%TCIDATA{BibliographyScheme=Manual}
%TCIDATA{<META NAME="DocumentShell" CONTENT="Standard LaTeX\Standard LaTeX Article">}
%TCIDATA{Language=American English}
%BeginMSIPreambleData
\providecommand{\U}[1]{\protect\rule{.1in}{.1in}}
%EndMSIPreambleData
\newtheorem{theorem}{Theorem}
\newtheorem{acknowledgement}[theorem]{Acknowledgement}
\newtheorem{algorithm}[theorem]{Algorithm}
\newtheorem{axiom}[theorem]{Axiom}
\newtheorem{case}[theorem]{Case}
\newtheorem{claim}[theorem]{Claim}
\newtheorem{conclusion}[theorem]{Conclusion}
\newtheorem{condition}[theorem]{Condition}
\newtheorem{conjecture}[theorem]{Conjecture}
\newtheorem{corollary}[theorem]{Corollary}
\newtheorem{criterion}[theorem]{Criterion}
\newtheorem{definition}[theorem]{Definition}
\newtheorem{example}[theorem]{Example}
\newtheorem{exercise}[theorem]{Exercise}
\newtheorem{lemma}[theorem]{Lemma}
\newtheorem{notation}[theorem]{Notation}
\newtheorem{problem}[theorem]{Problem}
\newtheorem{proposition}[theorem]{Proposition}
\newtheorem{remark}[theorem]{Remark}
\newtheorem{solution}[theorem]{Solution}
\newtheorem{summary}[theorem]{Summary}
\newenvironment{proof}[1][Proof]{\noindent\textbf{#1.} }{\ \rule{0.5em}{0.5em}}
\begin{document}

\title{[Draft \#9] About Distributions of Significant Leading Digits}
\author{Vladimir S. Berman\\vb7654321@gmail.com}
\maketitle

\begin{abstract}
This paper examines the various forms of distribution densities for leading
digits. The formulation of the problem of the distribution of leading digits
imposes certain limitations on the Distribution of Significant Leading Digits
(DSLD) and should have some specific properties. This provides the possibility
of finding an exact solution or alternatively constructing an approximation.
This paper presents solutions for the various underlying distributions, with
explicit analytical solutions for the first significant digit densities, but
for others we find highly accurate approximations. This paper rigorously
deduces a new type of DSLD instead of the frequently used empirical NBL.

\end{abstract}
\tableofcontents
\listoffigures
\listoftables

\section{Notations}

The following notation and definitions will be used in this article.%

%TCIMACRO{\TeXButton{Notation Enumiration}{\begin{enumerate}[align=right]
%\item[$N$]  number of elements  in set (infinit or finit)
%\item[$\rho(k)=\frac{n_k}{N}$]  proportion of  $k$  in set
%\item[DSLD] Distribution of Significant Leading Digits
%\item[UD] Uniform Distribution.
%\item[$\displaystyle\log{x}=\frac{\ln{x}}{\ln{10}}%
%$] the logarithm of x to base 10.
%\item[$\displaystyle\lfloor{x}\rfloor
%$] floor, greatest integer less than or equal to a number $x$.
%\item[$x\mod{}1$]  remainder
%$\{x\}=x-\lfloor{x}\rfloor$.
%\item[$\digamma=\int\limits_{t=0}^\infty
%f(t) dt$]  probability of positive numbers.
%\item[$\imath=\sqrt{-1}$]  the imaginary unit.
%\item[$\rho(k)$]  probability mass function (PMF)
%\item[CDF]  Cumulative Distribution Function
%\item[PDF]  Probability Density Function
%\item[NBL]  The Newcomb--Benford Law \cite{formann0}
%\item[$\mathbb{N}$] The set of natural numbers: $\{1, 2, 3, . . .\}$
%\item[$\mathbb{Z}^{*}%
%$] The set of non-negative integers: $\{0,1, 2, 3, . . .\}$
%\item[$\mathbb{Z}$] The set of integers: $\{...,-3,-2,-1,0,1, 2, 3, . . .\}$
%\end{enumerate}
%}}%
%BeginExpansion
\begin{enumerate}[align=right]
\item[$N$]  number of elements  in set (infinit or finit)
\item[$\rho(k)=\frac{n_k}{N}$]  proportion of  $k$  in set
\item[DSLD] Distribution of Significant Leading Digits
\item[UD] Uniform Distribution.
\item[$\displaystyle\log{x}=\frac{\ln{x}}{\ln{10}}%
$] the logarithm of x to base 10.
\item[$\displaystyle\lfloor{x}\rfloor
$] floor, greatest integer less than or equal to a number $x$.
\item[$x\mod{}1$]  remainder
$\{x\}=x-\lfloor{x}\rfloor$.
\item[$\digamma=\int\limits_{t=0}^\infty
f(t) dt$]  probability of positive numbers.
\item[$\imath=\sqrt{-1}$]  the imaginary unit.
\item[$\rho(k)$]  probability mass function (PMF)
\item[CDF]  Cumulative Distribution Function
\item[PDF]  Probability Density Function
\item[NBL]  The Newcomb--Benford Law \cite{formann0}
\item[$\mathbb{N}$] The set of natural numbers: $\{1, 2, 3, . . .\}$
\item[$\mathbb{Z}^{*}%
$] The set of non-negative integers: $\{0,1, 2, 3, . . .\}$
\item[$\mathbb{Z}$] The set of integers: $\{...,-3,-2,-1,0,1, 2, 3, . . .\}$
\end{enumerate}
%EndExpansion


\section{Introduction}%

%TCIMACRO{\TeXButton{\label{introduction}}{\label{introduction}}}%
%BeginExpansion
\label{introduction}%
%EndExpansion




For the past $135$ years, ever since the problem of the distribution of
leading digits began to attract attention
%TCIMACRO{\TeXButton{cite0}{\cite{newcombs}  and \cite{benford}}}%
%BeginExpansion
\cite{newcombs}  and \cite{benford}%
%EndExpansion
, more than $1,500$ articles and books have been published on this subject
%TCIMACRO{\TeXButton{cite1}{\cite{hill},\cite{bergerhill},\cite{arnotheodore}%
%,\cite{miller}}}%
%BeginExpansion
\cite{hill},\cite{bergerhill},\cite{arnotheodore},\cite{miller}%
%EndExpansion
. Particular attention has been paid to the NBL which describes the DSLD,
chosen from empirical distributions of leading digits%

\begin{equation}
\rho_{NBL}(k)=\log(k+1)-\log(k),1\leq k\leq9. \label{BF_Distr}%
\end{equation}
\medskip%

%TCIMACRO{\TeXButton{B}{\begin{table}[!htbp] \centering}}%
%BeginExpansion
\begin{table}[!htbp] \centering
%EndExpansion%
\begin{tabular}
[c]{|c|c|c|c|c|c|c|c|c|c|}\hline\hline
$k$ & $1$ & $2$ & $3$ & $4$ & $5$ & $6$ & $7$ & $8$ & $9$\\\hline
$\rho(k)$ & ${\small .3010}$ & .${\small 1760}$ & ${\small .1249}$ &
${\small .09690}$ & ${\small .07918}$ & ${\small .06691}$ & ${\small .05799}$
& ${\small .05115}$ & ${\small .04576}$\\\hline\hline
\end{tabular}
\bigskip\caption{Table Caption 1}\label{TableKey1}%
%TCIMACRO{\TeXButton{E}{\end{table}}}%
%BeginExpansion
\end{table}%
%EndExpansion


Many authors believe that the NBL has a universal character and that the
majority of cases of number sequences lead to a realization of this law. And
in most cases, the distribution of the leading digits based on experimental
approaches have been followed by numerical approximations. Often, in the
literature, the appearance of the NBL is described as mysterious and magical.
Such statements have naturally attracted the attention of various researchers
%TCIMACRO{\TeXButton{cite1a}{\cite{diaconis},\cite{raimi},\cite{nigrini1}%
%,\cite{arnold0},\cite{arnold1},\cite{bergerhill},\cite{nigrini},\cite
%{arnotheodore} and \cite{miller}}}%
%BeginExpansion
\cite{diaconis},\cite{raimi},\cite{nigrini1},\cite{arnold0},\cite
{arnold1},\cite{bergerhill},\cite{nigrini},\cite{arnotheodore} and \cite
{miller}%
%EndExpansion
.

An aspiration to use a universal distribution is, in our view, due to the
charm of the central limit theorem, where it is proved that for any underlying
distribution, we get the normal distribution the in limit. The main attraction
of the NBL is due to its simplicity.

However, for the sake of simplicity there is the sacrifice of the important
properties of the underlying distributions. So many underlying distributions
have parameters whose influence is not reflected in the NBL. Another important
feature is the periodic dependence of the DSLD on the parameters underlying
the distribution, e.g., for the exponential ($0<\lambda$) and normal
($\mu=0,0<\sigma$) distributions we have, from numerical simulations, a
periodic dependence of the DSLD on $\log(\lambda)$ and $\log(\sigma)$, respectively.

Most publications have considered of empirical data sets or numerical
interpretation. It is surprising that only a few publications have addressed
the underlying probability distributions. In
%TCIMACRO{\TeXButton{cite2}{\cite{pinkham} and \cite{raimi}} }%
%BeginExpansion
\cite{pinkham} and \cite{raimi}
%EndExpansion
irregular attempts to use certain symmetry properties have been made to
explain the logarithmic nature of the NBL. Symmetry properties are powerful
tools that will be used in this paper to find the exact solutions and
approximate analysis of the distributions. In our opinion, symmetry plays a
very important role in the behavior of the DSLD and is a powerful instrument
for analyzing the DSLD. In many cases, the NBL can be regarded as an ad hoc
approach to empirical data.

We will build a few basic types of DSLDs. We will find analytical forms of the
distribution of leading digits. The main results of this paper concern the
presentation of three main constructions of a DSLD.

Let $F(z)$ be a cumulative distribution function (CDF). We will generate a
random sample drawn from the distribution given by \ a continuous random
variable $Z$. We have the distribution function for the sequence of numbers
from which we select the sample, and then the density of the leading digits
$k$\ is defined as\bigskip%
\begin{align}
10^{m}k &  \leq z<10^{m}(k+1),\label{MainFormulation}\\
\rho(k) &  =%
%TCIMACRO{\dsum \limits_{m=m_{\min}}^{m_{\max}}}%
%BeginExpansion
{\displaystyle\sum\limits_{m=m_{\min}}^{m_{\max}}}
%EndExpansion
\left(  F(10^{m}(k+1))-F(10^{m}k)\right)  ,\\
F(z) &  =%
%TCIMACRO{\dint \limits_{t=a}^{z}}%
%BeginExpansion
{\displaystyle\int\limits_{t=a}^{z}}
%EndExpansion
f(t)dt,10^{L}\leq k\leq10^{L}-1,\nonumber\\
L &  \in%
%TCIMACRO{\U{2124} }%
%BeginExpansion
\mathbb{Z}
%EndExpansion
,L=\left\lfloor \log(k)\right\rfloor .\nonumber
\end{align}


Where \ $f(z)$\ is\ the probability density function (PDF) of the underlying
probability distribution, $k$\ is the significant ($0<k$) leading digit of an
element of the sample, and $L=\left\lfloor \log(k)\right\rfloor $\ \ defines
the order of magnitude of $k$.

It should be noted that the probability density of the leading digit is not an
arbitrary function. It is limited by a number of strong restrictions and
symmetries. The explicit form of \ $\rho(k,L)$ \ depends on the underlying distribution.

This paper is organized as follows.

The\
%TCIMACRO{\TeXButton{\ref{FormulationOf prob}}{\ref{FormulationOf prob}} }%
%BeginExpansion
\ref{FormulationOf prob}
%EndExpansion
Section discusses the mathematical formulation of the problems. After we solve
the functional equations, we have three main types of forms of DSLD.

In Section
%TCIMACRO{\TeXButton{\ref{InerPesent}}{\ref{InerPesent}} }%
%BeginExpansion
\ref{InerPesent}
%EndExpansion
we give integral relationships for different cases of DSLD.

In Section
%TCIMACRO{\TeXButton{\ref{Diocret Distributions}}{\ref{Diocret Distributions}}
%}%
%BeginExpansion
\ref{Diocret Distributions}
%EndExpansion
we consider a few examples of applications of our approach to discrete sets of data.

In the following sections, we present examples capable of exact solutions. In
Section
%TCIMACRO{\TeXButton{\ref{TofD}}{\ref{TofD}}}%
%BeginExpansion
\ref{TofD}%
%EndExpansion
, there are some examples of explicit formulas for $\rho(k)$. More
specifically, we present various solutions not only in the interval $1\ldots9$
but in the interval $10\ldots99$ and so on.~We will also discuss how to build
analytical solutions for a number of popular underlying distributions. We then
build the solutions for this distributions. It is shown that in many cases
even a small number of members of sets gives a good approximation of the exact
solution. We then proceed, in Sections
%TCIMACRO{\TeXButton{\ref{DfED}+}{\ref{DfWD},\ref{DofLD2P} and \ref{NormD}} }%
%BeginExpansion
\ref{DfWD},\ref{DofLD2P} and \ref{NormD}
%EndExpansion
to review the distribution with infinite area of a random sample from an
underlying distribution.

Section
%TCIMACRO{\TeXButton{\ref{Discussion}}{\ref{Discussion}} }%
%BeginExpansion
\ref{Discussion}
%EndExpansion
presents in more detail the assumptions made implicitly. We also present some
harmonic and numerical approximations of DSLD which in many cases give exact
or `almost exact' \ solutions of the problem.

~In conclusion, we present examples that show the evolution of different forms
of DSLD, and the transition from one form to another when changing the
parameters of the underlying distribution. We discuss the characteristics of
the DSLD found and forms of correspondence with the NBL.

~~In the appendices we introduce some technical results useful for
understanding the subject of the paper, including integral formulas for the
calculation of a DSLD. There we consider in detail many technical aspects of
finding solutions and also, for the purpose of self-sufficiency, we have the
necessary reference information.

We present a few examples of distributions and show how the properties of the
underlying distributions determine the form of the distribution of the first
digits. We will consider different underlying distribution, such as, ratio $2$
positive numbers from UD, the production of $n$ nonnegative numbers, the
distribution of leading digits for an exponential underlying distribution, and
ratio $2$ positive numbers. In some cases we can derive explicit formulas for
the DSLD and trace the evolution of the distribution as a function of the
parameters. In other cases, on the basis of a knowledge of the form of the
function $\rho(k)$, we will derive approximate forms. Below we give a brief
description of the different distributions.

\section{Formulation of the Problem}%

%TCIMACRO{\TeXButton{\label{FormulationOf prob}}{\label{FormulationOf prob}}}%
%BeginExpansion
\label{FormulationOf prob}%
%EndExpansion


Before formulating the common task, let us consider a simple example. For
example, consider the set of two numbers $[0.1,\pi]$, which is equivalent to
$[1000\ldots,31415\ldots]$. Then for $\lfloor\log(k)\rfloor=0$, first digits
are $[1,3]$, and for $\lfloor\log(k)\rfloor=1$ the first two digits are
$[10,31]$, and so on.

We will consider a function \ $\rho(k)$ that has the following symmetry
properties. For example, for $k=1,$ we have%

\begin{align}
\rho(1) &  =\rho(10)+\rho(11)+\ldots+\rho(19),\label{Cond1}\\
\rho(1) &  =\rho(100)+\rho(101)+\ldots+\rho(199),\nonumber\\
\rho(1) &  =\rho(1000)+\rho(1001)+\ldots+\rho(1999),\nonumber\\
\rho(10) &  =\rho(100)+\rho(101)+\ldots+\rho(109),\nonumber\\
&  \ldots\nonumber
\end{align}
In the general case, we have%
\begin{equation}
\rho(k,0)=%
%TCIMACRO{\dsum \limits_{i=0}^{10^{L}-1}}%
%BeginExpansion
{\displaystyle\sum\limits_{i=0}^{10^{L}-1}}
%EndExpansion
\rho(10^{L}k+i,L),L\in%
%TCIMACRO{\U{2124} }%
%BeginExpansion
\mathbb{Z}
%EndExpansion
^{\ast}.\label{MainEq_1}%
\end{equation}
%

\begin{align}
\rho(k,L) &  =%
%TCIMACRO{\dsum \limits_{i=0}^{10^{\delta}-1}}%
%BeginExpansion
{\displaystyle\sum\limits_{i=0}^{10^{\delta}-1}}
%EndExpansion
\rho(10^{\delta}k+i,L+\delta),\label{MainEq_2}\\
L &  \in%
%TCIMACRO{\U{2124} }%
%BeginExpansion
\mathbb{Z}
%EndExpansion
^{\ast},\delta\in%
%TCIMACRO{\U{2124} }%
%BeginExpansion
\mathbb{Z}
%EndExpansion
^{\ast}.\nonumber
\end{align}
%

\begin{equation}%
%TCIMACRO{\dsum \limits_{k=10^{L}}^{10^{L+1}-1}}%
%BeginExpansion
{\displaystyle\sum\limits_{k=10^{L}}^{10^{L+1}-1}}
%EndExpansion
\rho(k,L)=\digamma,L\in%
%TCIMACRO{\U{2124} }%
%BeginExpansion
\mathbb{Z}
%EndExpansion
^{\ast}.\label{MainEq_0}%
\end{equation}%
\[
L=\lfloor\log(k)\rfloor,0\leq\rho(k,L)\leq1.
\]
\noindent Here, $0\leq\digamma$ is the probability of positive numbers.

These requirements put restrictions on the possible forms of the function
$\rho(k)$. This raises the interesting question, `What kind of form should the
distribution density of significant leading digits have?'

The relations (\ref{MainEq_2}--\ref{MainEq_0}) can be interpreted as a system
of functional equations. In some cases, these equations have explicit solutions.

For the considered sum, (\ref{MainFormulation}) can be expressed through one
of the types of construction from the list below.

Some solutions of the functional equations (\ref{MainEq_2}--\ref{MainEq_0})
\ are in Appendix
%TCIMACRO{\TeXButton{\ref{SofFE}}{\ref{SofFE}}}%
%BeginExpansion
\ref{SofFE}%
%EndExpansion
.

\begin{itemize}
\item In case $\rho(k,L)$ only depends on $L$, then all the digit for the same
$L$ have the same probability, and we have%
\begin{equation}
\rho(k)=\frac{1}{9}10^{-\lfloor\log(k)\rfloor}. \label{S1}%
\end{equation}


\item In case $\rho(k,L)$ depends only on $k$, the solution is%
\begin{align}
\rho(k)  &  =\digamma\left(  \log(k+1)-\log(k)\right) \label{S2}\\
&  +Q_{1}(\log(k+1))-Q_{1}(\log(k)),\nonumber\\
Q_{1}(s)  &  =Q_{1}(s+1),\forall s.\nonumber
\end{align}


\item In the more general case, when $\rho(k,\lfloor\log(k)\rfloor)$, we have%
\begin{equation}
\rho(k)=\Omega(\log(k+1)-\lfloor\log(k)\rfloor)-\Omega(\log(k)-\lfloor
\log(k)\rfloor). \label{S3}%
\end{equation}

\end{itemize}

\noindent Here, $\Omega(s)$ is any admissible function.

In the case when $j_{\max}=+\infty$ and \ $j_{\min}=-\infty$ we can see that
the function $G(k)=const+%
%TCIMACRO{\dsum \limits_{j=-\infty}^{\infty}}%
%BeginExpansion
{\displaystyle\sum\limits_{j=-\infty}^{\infty}}
%EndExpansion
F(10^{j}k)$ is invariant under transformation with integer $a$
\[
j\rightarrow j^{\prime}-a,\log(k)\rightarrow\log(k^{\prime}10^{a}%
)=\log(k^{\prime})+a.
\]
In this case \ we have a solution in the form of the sum of a 1-period
function and a linear function.%
\begin{align*}
const+%
%TCIMACRO{\dsum \limits_{j=-\infty}^{\infty}}%
%BeginExpansion
{\displaystyle\sum\limits_{j=-\infty}^{\infty}}
%EndExpansion
F(10^{j+\log(k)})  &  =const+A\log(k)+Q_{1}(\log(k)),\\
Q_{1}(s+1)  &  =Q_{1}(s),\forall s.\text{ }%
\end{align*}
Sometimes it is convenient to add, without loss of generality, to the argument
of $Q_{1}$, the integer $-\lfloor\log(k)\rfloor$%

\[
const+%
%TCIMACRO{\dsum \limits_{j=-\infty}^{\infty}}%
%BeginExpansion
{\displaystyle\sum\limits_{j=-\infty}^{\infty}}
%EndExpansion
F(10^{j+\log(k)})=A\log(k)+Q_{1}(\log(k)-\lfloor\log(k)\rfloor).
\]
\ \ and (\ref{S2}) takes the form%
\begin{align}
\rho(k)  &  =A\left(  \log(k+1)-\log(k)\right)  +Q_{1}(\log(k+1)\label{S2a}\\
&  -\lfloor\log(k)\rfloor)-Q_{1}(\log(k)-\lfloor\log(k)\rfloor).\nonumber
\end{align}%
\[
\log(k)-\lfloor\log(k)\rfloor\equiv\log(k)\text{ }\operatorname{mod}\text{
}1=\left\{  k\right\}  .
\]


From (\ref{S3}) we can present (\ref{S1}) as%
\[
\Omega(s)=\frac{1}{9}10^{s},
\]
and (\ref{S2}) as%
\begin{equation}
\Omega(s)=\digamma\log(s)+Q_{1}(s). \label{S2b}%
\end{equation}


We can enter a condition in which the implemented solution (\ref{S2}) for all
permissible $L$ and $k$%
\begin{align}
&  \Omega(\log(k+1))-\Omega(\log(k+1)+L)\label{S2c}\\
-\Omega(\log(k))+\Omega(\log(k)+L) &  =0,\nonumber\\
k &  \in%
%TCIMACRO{\U{2115} }%
%BeginExpansion
\mathbb{N}
%EndExpansion
,L\in%
%TCIMACRO{\U{2124} }%
%BeginExpansion
\mathbb{Z}
%EndExpansion
^{\ast}.\nonumber
\end{align}


\section{Integral representation of DSLD}%

%TCIMACRO{\TeXButton{\labelf{InerPesent}}{\label{InerPesent}}}%
%BeginExpansion
\label{InerPesent}%
%EndExpansion


Because operate with discrete relations, which are less convenient than
continuous ones, it would be useful to replace the summation in
(\ref{MainFormulation}) with an integration. We use the well-known
Euler--Maclaurin formula (EMf)%
%TCIMACRO{\TeXButton{\cite_abramowitz}{\cite{knuth},\cite{wabramowitzstegun}
%and \cite{knuth2}} }%
%BeginExpansion
\cite{knuth},\cite{wabramowitzstegun}
and \cite{knuth2}
%EndExpansion
and make an exact replacement of the sum by the integral.%
%TCIMACRO{\TeXButton{EulerMaclaurin}{\label{EulerMaclaurin}}}%
%BeginExpansion
\label{EulerMaclaurin}%
%EndExpansion
Euler--Maclaurin's formula (EMf)\ gives a representation of a sum in the form
of a definite integral. It is important to stress that the EMf is an exact
expression of the sum in integral form (for at least a differentiable
function): it is not an approximation. From the EMf we have\bigskip\ an
expression the sum in the form of an integral which is (see Appendix
%TCIMACRO{\TeXButton{\ref{DerForm}}{\ref{DerForm}}}%
%BeginExpansion
\ref{DerForm}%
%EndExpansion
)\
\begin{align}%
%TCIMACRO{\dsum \limits_{j=a}^{b}}%
%BeginExpansion
{\displaystyle\sum\limits_{j=a}^{b}}
%EndExpansion
g(j)  &  =%
%TCIMACRO{\dint \limits_{x=a}^{b}}%
%BeginExpansion
{\displaystyle\int\limits_{x=a}^{b}}
%EndExpansion
\left(  g(x)+\left(  x-\left\lfloor x\right\rfloor -\frac{1}{2}\right)
\frac{dg(x)}{dx}\right)  dx\label{EMF00}\\
&  +\frac{g(b)}{2}+\frac{g(a)}{2},\\
a\text{ and }b\text{ are integers, }a  &  <b.\nonumber
\end{align}


We will apply Euler--Maclaurin summation in this form, which is convenient for
our purposes.

In our case we have%
\begin{align*}
g(j)  &  =F(10^{j+\log(k+1)})-F(10^{j+\log(k)})\\
F(z)  &  =%
%TCIMACRO{\dint \limits_{-\infty}^{z}}%
%BeginExpansion
{\displaystyle\int\limits_{-\infty}^{z}}
%EndExpansion
f(t)dt.
\end{align*}


Depending on the values of $a$ and $b$, we have different representations
(\ref{EMF00}).

When $a=-\infty(z=0)$ and $b=\infty(z=\infty)$ we have%
\begin{align}
\rho(k)  &  =%
%TCIMACRO{\dint \limits_{t=0}^{\infty}}%
%BeginExpansion
{\displaystyle\int\limits_{t=0}^{\infty}}
%EndExpansion
f(t)P(k,t)dt,\label{EMF01}\\
\Omega(s)  &  =-%
%TCIMACRO{\dint \limits_{t=0}^{\infty}}%
%BeginExpansion
{\displaystyle\int\limits_{t=0}^{\infty}}
%EndExpansion
f(s)\left\lfloor \log\left(  t\right)  -s\right\rfloor dt. \label{EMF01a}%
\end{align}
where%
\begin{equation}
P(k,t)=\left\lfloor \log\left(  \frac{t}{k}\right)  \right\rfloor
-\left\lfloor \log\left(  \frac{t}{k+1}\right)  \right\rfloor . \label{EMF02}%
\end{equation}
For a positive integer $k$\ and a positive $t$\ we have
\begin{equation}
P(k,t)=\left\{
\begin{array}
[c]{cc}%
1, & 10^{m}k\leq t\leq10^{m}(k+1),\\
0, & t\notin\lbrack10^{m}k,10^{m}(k+1)],\text{integer }m.
\end{array}
\right.  \label{EMF02a}%
\end{equation}


Formula (\ref{EMF01}) is equivalent to%
\begin{align}
\rho(k,L)  &  =%
%TCIMACRO{\dsum \limits_{j=-\infty}^{\infty}}%
%BeginExpansion
{\displaystyle\sum\limits_{j=-\infty}^{\infty}}
%EndExpansion%
%TCIMACRO{\dint \limits_{10^{j-L}k}^{10^{j-L}(k+1)}}%
%BeginExpansion
{\displaystyle\int\limits_{10^{j-L}k}^{10^{j-L}(k+1)}}
%EndExpansion
f(t)dt,\label{EMF01b}\\
L  &  =\left\lfloor \log\left(  k\right)  \right\rfloor .\nonumber
\end{align}
In the case when the function $f(t)$ has no fixed points in the interior of
the interval $[t_{\min},t_{\max}]$ or $-\infty<j_{\min}$ or/and $j_{\max
}<\infty$ then we have a solution independent of $\left\lfloor \log\left(
k\right)  \right\rfloor $ (\ref{S2}). Conveniently, in the spirit of
(\ref{S2}), this case can be expressed as the sum of its logarithmic and
periodic members%

\begin{align}
\rho(k)  &  =\digamma\left(  \log(k+1)-\log(k)\right)  +Q_{1}(\log
(k+1))-Q_{1}(\log(k+1)),\label{EMF03}\\
&  Q_{1}(\log(k))=%
%TCIMACRO{\dint \limits_{s=0}^{\infty}}%
%BeginExpansion
{\displaystyle\int\limits_{s=0}^{\infty}}
%EndExpansion
f(s)\left(  \log\left(  \frac{s}{k}\right)  -\left\lfloor \log\left(  \frac
{s}{k}\right)  \right\rfloor \right)  ds,\label{EMF03a}\\
\digamma &  =%
%TCIMACRO{\dint \limits_{s=0}^{\infty}}%
%BeginExpansion
{\displaystyle\int\limits_{s=0}^{\infty}}
%EndExpansion
f(s)ds.
\end{align}


It is interesting to note that the first term in (\ref{EMF03}) has the form
for any admissible function $f(s)$, this follows from integration by parts.%
\begin{equation}%
%TCIMACRO{\dint \limits_{x=-\infty}^{\infty}}%
%BeginExpansion
{\displaystyle\int\limits_{x=-\infty}^{\infty}}
%EndExpansion%
%TCIMACRO{\dint \limits_{t=10^{x}k}^{10^{x}\left(  k+1\right)  }}%
%BeginExpansion
{\displaystyle\int\limits_{t=10^{x}k}^{10^{x}\left(  k+1\right)  }}
%EndExpansion
f(t)dt=(\log\left(  k+1\right)  -\log\left(  k\right)  )%
%TCIMACRO{\dint \limits_{t=0}^{\infty}}%
%BeginExpansion
{\displaystyle\int\limits_{t=0}^{\infty}}
%EndExpansion
f(t)dt. \label{EMF04}%
\end{equation}


The function $Q_{1}(s)$ is 1-periodic in its argument. It is natural to
consider the Fourier decomposition of this function. Then, after a few
operations (see Appendix
%TCIMACRO{\TeXButton{\ref{FourieExpan}}{\ref{FourieExpan}}}%
%BeginExpansion
\ref{FourieExpan}%
%EndExpansion
), we can write $\rho(k)$ as%

\begin{align}
\rho(k)  &  =\digamma(\log\left(  k+1\right)  -\log\left(  k\right)
)+\label{EMF05}\\
&
%TCIMACRO{\dsum \limits_{n=1}^{\infty}}%
%BeginExpansion
{\displaystyle\sum\limits_{n=1}^{\infty}}
%EndExpansion
\Lambda(n)\left(  \sin\left(  ns_{1}+\varphi(n)\right)  -\sin\left(
ns+\varphi(n)\right)  \right)  ,\nonumber\\
s_{1}  &  =2\pi\log(k+1),s_{1}=2\pi\log(k),\digamma=%
%TCIMACRO{\dint \limits_{x=0}^{\infty}}%
%BeginExpansion
{\displaystyle\int\limits_{x=0}^{\infty}}
%EndExpansion
f(x)dx
\end{align}%
\begin{equation}
\Lambda(n)\exp(\imath\varphi(n))=\frac{1}{n\pi}%
%TCIMACRO{\dint \limits_{x=0}^{\infty}}%
%BeginExpansion
{\displaystyle\int\limits_{x=0}^{\infty}}
%EndExpansion
x^{-\imath Y}f(x)dx,Y=\frac{2\pi n}{\ln(10)}. \label{EMF07}%
\end{equation}
In many practical cases, the Fourier coefficients decrease rapidly and then it
is possible to confine one's attention to only a few (even the first one) of
the first ones.

\section{Discrete Sets}%

%TCIMACRO{\TeXButton{\label{Diocret Distributions}}{\label
%{Diocret Distributions}}}%
%BeginExpansion
\label{Diocret Distributions}%
%EndExpansion


Consider a set of $N$ positive numbers. It is convenient to introduce an
enumeration of the elements of this set (the order is not important). Then%
\begin{equation}
f(z,N)=\frac{1}{N}%
%TCIMACRO{\dsum \limits_{j=1}^{N}}%
%BeginExpansion
{\displaystyle\sum\limits_{j=1}^{N}}
%EndExpansion
\delta\left(  z-a_{j}\right)  . \label{Disc1}%
\end{equation}


From (\ref{EM5})%
\begin{equation}
P(k,x)=\left\lfloor \log\left(  \frac{x}{k}\right)  \right\rfloor
-\left\lfloor \log\left(  \frac{x}{k+1}\right)  \right\rfloor . \label{DiscP}%
\end{equation}


\noindent we have%
\[
\rho(k,N)=\frac{1}{N}%
%TCIMACRO{\dsum \limits_{j=1}^{N}}%
%BeginExpansion
{\displaystyle\sum\limits_{j=1}^{N}}
%EndExpansion
P(k,a_{j}).
\]
And hence we obtain%
\begin{align}
\Omega(s)  &  =-\frac{1}{N}%
%TCIMACRO{\dsum \limits_{j=1}^{N}}%
%BeginExpansion
{\displaystyle\sum\limits_{j=1}^{N}}
%EndExpansion
\left\lfloor \log\left(  a_{j}\right)  -s\right\rfloor
+const,\label{OmegaDisc}\\
\Omega(1)-\Omega(0)  &  =1.
\end{align}


This is the general solution. But in some cases $\rho(k,N)$ can monotonically
go to the NBL or oscillate around the NLB. In such a case, it is more
convenient to use the form (\ref{A_B_Fourier}) and (\ref{Fouri3}). Then%

\begin{equation}
A_{n}(N)+\imath B_{n}(N)=\frac{1}{n\pi}%
%TCIMACRO{\dint \limits_{x=0}^{\infty}}%
%BeginExpansion
{\displaystyle\int\limits_{x=0}^{\infty}}
%EndExpansion
x^{-\imath Y}f(x,N)dx,Y=\frac{2\pi n}{\ln(10)},\nonumber
\end{equation}
ant taking in consideration (\ref{Disc1}) we have%
\begin{align}
A_{n}(N)+\imath B_{n}(N)  &  =\frac{1}{n\pi N}%
%TCIMACRO{\dsum \limits_{j=1}^{N}}%
%BeginExpansion
{\displaystyle\sum\limits_{j=1}^{N}}
%EndExpansion
(a_{j})^{-\imath Y}\label{Disc3a}\\
&  =\Lambda(n,N)\exp(\imath\varphi(n,N)),Y=\frac{2\pi n}{\ln(10)},
\end{align}
and%

\begin{align}
\rho(k,N)  &  =\log\left(  k+1\right)  -\log\left(  k\right)  +\label{Disc3}\\
&
%TCIMACRO{\dsum \limits_{n=1}^{\infty}}%
%BeginExpansion
{\displaystyle\sum\limits_{n=1}^{\infty}}
%EndExpansion
\left(
\begin{array}
[c]{c}%
A_{n}\left(  \sin\left(  ns_{1}\right)  -\sin\left(  ns\right)  \right) \\
+B_{n}(\cos\left(  ns_{1}\right)  -\cos\left(  ns\right)  )
\end{array}
\right)  \text{ or }\nonumber\\
\rho(k,N)  &  =\log\left(  k+1\right)  -\log\left(  k\right)  +%
%TCIMACRO{\dsum \limits_{n=1}^{\infty}}%
%BeginExpansion
{\displaystyle\sum\limits_{n=1}^{\infty}}
%EndExpansion
\Lambda(n,N)\left(  \sin\left(  ns_{1}+\varphi(n,N)\right)  -\sin\left(
ns+\varphi(n,N)\right)  \right)  ,\\
s_{1}  &  =2\pi\log\left(  k+1\right)  ,s=2\pi\log\left(  k\right)  .\nonumber
\end{align}


Let us introduce a criterion for assessing the contribution of the oscillatory
term%
\begin{equation}
V(k,n,N)\ =\left\vert \frac{%
%TCIMACRO{\dsum \limits_{j=1}^{n}}%
%BeginExpansion
{\displaystyle\sum\limits_{j=1}^{n}}
%EndExpansion
\Lambda(j,N)\left(  \sin\left(  js_{1}+\varphi(j,N)\right)  -\sin\left(
js+\varphi(j,N)\right)  \right)  }{\log\left(  1+\frac{1}{k}\right)  \digamma
}-1\right\vert ,\digamma=1.
\end{equation}
if $V\ll1$ then this is the case of the NBL. In many cases we need only a few
terms in the sum ( $n=1,2,3,\ldots\frac{{}}{{}}$).

Let us consider two examples.

\begin{itemize}
\item $a_{j}=\alpha^{j}$%
\begin{equation}
\frac{1}{n\pi N}%
%TCIMACRO{\dsum \limits_{j=1}^{N}}%
%BeginExpansion
{\displaystyle\sum\limits_{j=1}^{N}}
%EndExpansion
(a_{j})^{-\imath Y}=\frac{1}{n\pi N}\frac{1-\exp(-N\beta\imath)}{\exp
(\beta\imath)-1},\beta=2\pi n\log(\alpha). \label{Disc4}%
\end{equation}
If $\alpha=10^{r}$ where $r=\pm1,\pm2,\ldots$, then the RHS is
\begin{align*}
\beta &  =2\pi nr,\Lambda(n,N)=\frac{1}{n\pi},B_{n}(N)=0,\\
\rho(k,N)  &  =\left\{
\begin{array}
[c]{cc}%
1, & k=1,\\
0, & k\neq1,k<10
\end{array}
\right.  .
\end{align*}

\end{itemize}

If \ $\alpha\neq10^{r}$, then%
\begin{align*}
\Lambda(n,N)  &  =\frac{1}{\pi nN}\sqrt{\frac{1-\cos(\beta N)}{1-\cos(\beta)}%
},\\
\tan(\varphi(n,N))  &  =-\frac{\sin(\beta)+\sin(N\beta)}{\cos(\beta
)+\cos(N\beta)}%
\end{align*}


How can we see that $\rho(k,N)$ as $N\rightarrow\infty$ approaches the NBL,
with damped oscillations.

\begin{itemize}
\item $a_{j}=j^{\alpha}$ where $\alpha\neq0$ is real
\end{itemize}

%

\[
\frac{1}{n\pi N}%
%TCIMACRO{\dsum \limits_{j=1}^{N}}%
%BeginExpansion
{\displaystyle\sum\limits_{j=1}^{N}}
%EndExpansion
(a_{j})^{-\imath Y}=\frac{1}{n\pi N}%
%TCIMACRO{\dsum \limits_{j=1}^{N}}%
%BeginExpansion
{\displaystyle\sum\limits_{j=1}^{N}}
%EndExpansion
\exp(-\imath\beta\log(j)),\beta=2\pi n\alpha.
\]
When $N\rightarrow\infty$\ we can convert the summation to an integration%
\begin{align}
\frac{1}{n\pi N}%
%TCIMACRO{\dsum \limits_{j=1}^{N}}%
%BeginExpansion
{\displaystyle\sum\limits_{j=1}^{N}}
%EndExpansion
\exp(-\imath\beta\log(j))  &  \thickapprox\frac{1}{n\pi N}%
%TCIMACRO{\dint \limits_{x=1}^{N}}%
%BeginExpansion
{\displaystyle\int\limits_{x=1}^{N}}
%EndExpansion
\exp(-\imath\beta\log(x))dx\\
&  \thickapprox\frac{1}{n\pi}\exp(-\imath\beta\log(N))%
%TCIMACRO{\dint \limits_{1/N}^{1}}%
%BeginExpansion
{\displaystyle\int\limits_{1/N}^{1}}
%EndExpansion
\exp(-\imath\beta\log(y))dy\nonumber
\end{align}
from here we have that $\rho(k,N\rightarrow\infty)$\ \ is a sum of oscillating
functions from $\log(N)$ with periods $T_{n}=$\ $\frac{1}{n\left\vert
\alpha\right\vert }$

Numerical calculation give us the dependencies of $A_{n}(N)$ and $A_{n}%
(N)$\ on $N$

\bigskip%

%TCIMACRO{\TeXButton{B}{\begin{figure}[H] \centering}}%
%BeginExpansion
\begin{figure}[H] \centering
%EndExpansion%
%TCIMACRO{\FRAME{itbpFU}{5.4509in}{3.9487in}{0in}{\Qcb{$a_{j}=j^{-2},A_{n}(N)$
%- green line,~$B_{n}(N)$ - blue line}}{}{discj_2.png}%
%{\special{ language "Scientific Word";  type "GRAPHIC";
%maintain-aspect-ratio TRUE;  display "USEDEF";  valid_file "F";
%width 5.4509in;  height 3.9487in;  depth 0in;  original-width 5.393in;
%original-height 3.9003in;  cropleft "0";  croptop "1";  cropright "1";
%cropbottom "0";  filename 'Discj_2.png';file-properties "XNPEU";}} }%
%BeginExpansion
{\parbox[b]{5.4509in}{\begin{center}
\includegraphics[
natheight=3.900300in,
natwidth=5.393000in,
height=3.9487in,
width=5.4509in
]%
{C:/Users/Vladimir/Documents/BENFORD/AAA/FOR ProofReading INTERNET/From PRS 4  (ver 9)/Discj_2__1.png}%
\\
$a_{j}=j^{-2},A_{n}(N)$ - green line,~$B_{n}(N)$ - blue line
\end{center}}}
%EndExpansion
\caption{$a_{j}=j^{\alpha}$}\label{figureKey1}%
%TCIMACRO{\TeXButton{E}{\end{figure}}}%
%BeginExpansion
\end{figure}%
%EndExpansion


\bigskip

\noindent and

\bigskip%
%TCIMACRO{\TeXButton{B}{\begin{figure}[H] \centering}}%
%BeginExpansion
\begin{figure}[H] \centering
%EndExpansion%
%TCIMACRO{\FRAME{itbpFU}{5.4509in}{3.9487in}{0in}{\Qcb{$a_{j}=j^{-2}%
%,T_{1}=\frac{1}{2},\rho(k=1,N)-\log(2)$ vs $s=\log(N)$}}{}{discj-2n.png}%
%{\special{ language "Scientific Word";  type "GRAPHIC";
%maintain-aspect-ratio TRUE;  display "USEDEF";  valid_file "F";
%width 5.4509in;  height 3.9487in;  depth 0in;  original-width 5.393in;
%original-height 3.9003in;  cropleft "0";  croptop "1";  cropright "1";
%cropbottom "0";  filename 'DiscJ-2N.png';file-properties "XNPEU";}} }%
%BeginExpansion
{\parbox[b]{5.4509in}{\begin{center}
\includegraphics[
natheight=3.900300in,
natwidth=5.393000in,
height=3.9487in,
width=5.4509in
]%
{C:/Users/Vladimir/Documents/BENFORD/AAA/FOR ProofReading INTERNET/From PRS 4  (ver 9)/DiscJ-2N__2.png}%
\\
$a_{j}=j^{-2},T_{1}=\frac{1}{2},\rho(k=1,N)-\log(2)$ vs $s=\log(N)$%
\end{center}}}
%EndExpansion
\caption{$a_{j}=\alpha^{j}$}\label{figureKey2}%
%TCIMACRO{\TeXButton{E}{\end{figure}}}%
%BeginExpansion
\end{figure}%
%EndExpansion


\bigskip

\noindent and $V(k=1,n=1,N=10^{4},=.2793$, so the contribution of the
oscillatory term is substantial.

\section{Continuous Distributions}

\subsection{Examples of Exact Forms of DSLD}%

%TCIMACRO{\TeXButton{TofD}{\label{TofD}}}%
%BeginExpansion
\label{TofD}%
%EndExpansion


This section and the next explore simple examples of distributions \ and find
explicit forms of DSLD, giving a direction for

what's to come. They have a future in common: all these problems involve
explicit analytical distributions of the leading digits. \newline\ \ \ Their
treatments all use the idea of the solution of the basic equations
(\ref{MainFormulation},\ref{MainEq_0},\ref{MainEq_1},\ref{MainEq_2}).

\begin{itemize}
\item \ DSLD\textbf{\ (\ref{S1})} for \textbf{Power Distribution}
($b=1$)\textbf{. }One example can be
\begin{align*}
PDF  &  =\left\{
\begin{array}
[c]{cc}%
nz^{n-1}, & 0\leq z\leq1,0\leq n,\\
0, & \text{otherwise}%
\end{array}
\right. \\
F(z)  &  =\left\{
\begin{array}
[c]{cc}%
z^{n}, & 0\leq z\leq1,0\leq n,\\
0, & \text{otherwise}%
\end{array}
\right.  .
\end{align*}
Then we have, from (\ref{MainFormulation}),%
\[
\rho(k)=%
%TCIMACRO{\dsum \limits_{m=-\infty}^{m_{\max}}}%
%BeginExpansion
{\displaystyle\sum\limits_{m=-\infty}^{m_{\max}}}
%EndExpansion
\left(  F(10^{m}(k+1))-F(10^{m}k)\right)  .
\]
and from the condition%
\begin{align*}
10^{m_{\max}}k  &  <1,\\
m_{\max}  &  =-\lfloor\log(k)\rfloor-1.
\end{align*}%
\[
\rho(k,n)=%
%TCIMACRO{\dsum \limits_{m=-\infty}^{-\lfloor\log(k)\rfloor-1}}%
%BeginExpansion
{\displaystyle\sum\limits_{m=-\infty}^{-\lfloor\log(k)\rfloor-1}}
%EndExpansion
10^{mn}\left(  (k+1)^{n}-k^{n}\right)  .
\]%
\begin{align}
s  &  =\log(k)-\lfloor\log(k)\rfloor,\label{Model1}\\
s_{1}  &  =\log(k+1)-\lfloor\log(k)\rfloor,\nonumber\\
\rho(k,n)  &  =\frac{10^{ns_{1}}-10^{ns}}{10^{n}-1},\nonumber\\
\Omega(s)  &  =\frac{10^{ns}}{10^{n}-1},\Omega(1)-\Omega(0)=1.\nonumber
\end{align}
If $n=1$ then
\[
\rho(k,1)=\frac{10^{-\lfloor\log(k)\rfloor}}{9}.
\]
When $n\rightarrow0$ we have the NBL:%
\[
\rho(k,n\rightarrow0)=\log(k+1)-\log(k)+O(n).
\]

\end{itemize}

This distribution describes not just the first digits but it is good for a
collection of first digits. For example, for $k=123$,%
\begin{align*}
\rho(123,n)  &  =\frac{1.24^{n}-1.23^{n}}{10^{n}-1},\\
\rho(123,0.5)  &  =.00185,\\
\rho(123,1.0)  &  =.001111,\\
\rho(123,2.0)  &  =.00021.
\end{align*}


\begin{itemize}
\item Here we can consider similar \textbf{Pareto distribution} ($b=1$) with
\[
CDF=1-z^{-n},0<n,1\leq z\leq\infty.
\]
with
\begin{align}
s  &  =\log(k)-\lfloor\log(k)\rfloor,\label{Model2}\\
s_{1}  &  =\log(k+1)-\lfloor\log(k)\rfloor,\nonumber\\
\rho(k,n)  &  =\frac{10^{n}\left(  -10^{-ns_{1}}+10^{-ns}\right)  }{10^{n}%
-1},\nonumber\\
\Omega(s)  &  =-\frac{10^{n-ns}}{10^{n}-1},\Omega(1)-\Omega(0)=1..\nonumber
\end{align}
\ and
\[
\rho(k,1)=\frac{10^{1+\lfloor\log(k)\rfloor}}{9}\left(  \frac{1}{k}-\frac
{1}{k+1}\right)  ,
\]
with convergence to the NBL as $n\rightarrow0$
\[
\rho(k,n\rightarrow0)=\log(k+1)-\log(k)+O(n).
\]
From (\ref{Model1} and \ref{Model2}) we have for $\Omega(s)$%
\ from\ \ (\ref{S2c}) a transition to NBL with $n=0.$
\end{itemize}

\subsubsection{DSLD of the Ratio of Two Numbers from the UD}%

%TCIMACRO{\TeXButton{DofDRatio2UD}{\label{DofDRatio2UD}}}%
%BeginExpansion
\label{DofDRatio2UD}%
%EndExpansion


From Appendices
%TCIMACRO{\TeXButton{\ref{SofFE}}{\ref{SofFE}} }%
%BeginExpansion
\ref{SofFE}
%EndExpansion
and
%TCIMACRO{\TeXButton{\ref{R2PNUD}}{\ref{R2PNUD}} }%
%BeginExpansion
\ref{R2PNUD}
%EndExpansion
we have%

\[
F(z)=\left\{
\begin{array}
[c]{cc}%
\frac{z}{2} & 0\leq z\leq1,\\
1-\frac{1}{2z} & 1<z.
\end{array}
\right.
\]%
\[
\rho(k,0)=%
%TCIMACRO{\dsum \limits_{m=-\infty}^{-1}}%
%BeginExpansion
{\displaystyle\sum\limits_{m=-\infty}^{-1}}
%EndExpansion
\left(  F(10^{m}(k+1))-F(10^{m}k)\right)  ,1\leq k\leq9.
\]%
\begin{equation}
\rho(k,0)=\frac{1}{18}-\frac{5}{9}\left(  \frac{1}{k+1}-\frac{1}{k}\right)
,1\leq k\leq9.\label{Rati01}%
\end{equation}%
\begin{align*}
\Omega(s) &  =A10^{-s}+B10^{s},\\
A &  =-\frac{5}{9},B=\frac{1}{18},\\
\Omega(1)-\Omega(0) &  =1.
\end{align*}%
\begin{align*}
s_{1} &  =\log(k+1)-\lfloor\log(k)\rfloor,\\
s &  =\log(k)-\lfloor\log(k)\rfloor.
\end{align*}%
\[
\rho(k,L)=\Omega(s_{1})-\Omega(s),k\text{ \ any positive integer}%
\]%
\begin{equation}
\rho(k,L)=\frac{10^{-L}}{18}-\frac{5}{9}\left(  \frac{1}{k+1}-\frac{1}%
{k}\right)  10^{L},L=\lfloor\log(k)\rfloor,1\leq k\leq9.\label{Ratio2}%
\end{equation}%
%TCIMACRO{\TeXButton{B}{\begin{table}[!htbp] \centering}}%
%BeginExpansion
\begin{table}[!htbp] \centering
%EndExpansion%
\begin{tabular}
[c]{|c|c|c|c|c|c|c|c|c|c|}\hline\hline
${\small k}$ & ${\small 1}$ & ${\small 2}$ & ${\small 3}$ & ${\small 4}$ &
${\small 5}$ & ${\small 6}$ & ${\small 7}$ & ${\small 8}$ & ${\small 9}%
$\\\hline
${\small \rho(k,0)}$ & ${\small 0.3333}$ & ${\small 0.1481}$ &
${\small 0.1019}$ & ${\small 0.08333}$ & ${\small 0.07407}$ &
${\small 0.06878}$ & ${\small 0.065488}$ & ${\small 0.06327}$ &
${\small 0.06173}$\\\hline\hline
\end{tabular}
\caption{Table Caption 2}\label{TableKey2}%
%TCIMACRO{\TeXButton{E}{\end{table}}}%
%BeginExpansion
\end{table}%
%EndExpansion


\bigskip

We can find the distribution of the second \ digits:%
\begin{align}
\varrho_{2}(m)  &  =%
%TCIMACRO{\dsum \limits_{i=1}^{9}}%
%BeginExpansion
{\displaystyle\sum\limits_{i=1}^{9}}
%EndExpansion
\rho(10i+m,1),\nonumber\\
\varrho_{2}(m)  &  =%
%TCIMACRO{\dsum \limits_{i=1}^{9}}%
%BeginExpansion
{\displaystyle\sum\limits_{i=1}^{9}}
%EndExpansion
\left(  \frac{1}{180}+\frac{50}{9(10i+m)}-\frac{50}{9(10i+m+1)}\right)
,m=0,1,\ldots,9. \label{2dDig}%
\end{align}%
\[
\varrho_{2}(0)=.1294,\varrho_{2}(1)=.1191,\varrho_{2}(2)=.1109,\varrho
_{2}(5)=.09423,\varrho_{2}(9)=.08169.
\]
and the distribution of the third\ digits $p$:
\begin{align}
\varrho_{3}(p)  &  =%
%TCIMACRO{\dsum \limits_{i=1}^{9}}%
%BeginExpansion
{\displaystyle\sum\limits_{i=1}^{9}}
%EndExpansion%
%TCIMACRO{\dsum \limits_{j=0}^{9}}%
%BeginExpansion
{\displaystyle\sum\limits_{j=0}^{9}}
%EndExpansion
\rho(100i+10j+p,2),\nonumber\\
\varrho_{3}(p)  &  =%
%TCIMACRO{\dsum \limits_{i=1}^{9}}%
%BeginExpansion
{\displaystyle\sum\limits_{i=1}^{9}}
%EndExpansion%
%TCIMACRO{\dsum \limits_{j=0}^{9}}%
%BeginExpansion
{\displaystyle\sum\limits_{j=0}^{9}}
%EndExpansion
\left(  \frac{1}{1800}+\frac{500}{9(100i+10j+p)}-\frac{500}{9(100i+10j+p+1)}%
\right)  ,\label{3dDig}\\
p  &  =0,1,\ldots,9.
\end{align}%
\[
\varrho_{3}(0)=.10254,\varrho_{3}(9)=.09759.
\]
We have what is almost the uniform distribution with a density close to $0.1$.

\subsubsection{DSLD of the Production of Two Numbers from a UD}%

%TCIMACRO{\TeXButton{P2NfUD}{\label{P2NfUD}}}%
%BeginExpansion
\label{P2NfUD}%
%EndExpansion


From Appendix
%TCIMACRO{\TeXButton{\ref{P2RNUD}}{\ref{P2RNUD}}}%
%BeginExpansion
\ref{P2RNUD}%
%EndExpansion%
\[
f_{Z}(z,2)=-\ln(z),0\leq z\leq1.
\]
%

\[
F_{Z}(z,2)=z\left(  1-\ln(z)\right)  ,0\leq z\leq1.
\]%
\[
\rho(k,L=0,2)=%
%TCIMACRO{\dsum \limits_{m=-\infty}^{m=-1}}%
%BeginExpansion
{\displaystyle\sum\limits_{m=-\infty}^{m=-1}}
%EndExpansion
\left(  F_{Z}(10^{m}(k+1),2)-F_{Z}(10^{m}k),2\right)  ,1\leq k\leq9.
\]%
\begin{align*}
\Omega(s)  &  =10^{s}\left(  A+Bs\right)  ,\\
A  &  =\frac{1}{9}+\frac{10}{81}\ln(10),B=-\frac{\ln(10)}{9}.
\end{align*}
%

\begin{align*}
s_{1} &  =\log(k+1)-\lfloor\log(k)\rfloor,\\
s &  =\log(k)-\lfloor\log(k)\rfloor.
\end{align*}%
\begin{align*}
\rho(k,L,2) &  =\Omega(s_{1})-\Omega(s),\\
\Omega(1)-\Omega(0) &  =1,k\text{ \ any positive integer.}%
\end{align*}%
\begin{equation}
\rho(k,L=0,2)=\frac{1}{9}\left(  k\ln(k)-(k+1)\ln(k+1)+1+\frac{10}{9}%
\ln(10)\right)  ,1\leq k\leq9.\label{Prod1}%
\end{equation}%
%TCIMACRO{\TeXButton{B}{\begin{table}[!htbp] \centering}}%
%BeginExpansion
\begin{table}[!htbp] \centering
%EndExpansion%
\begin{tabular}
[c]{|c|c|c|c|c|c|c|c|c|c|}\hline\hline
${\small k}$ & ${\small 1}$ & ${\small 2}$ & ${\small 3}$ & ${\small 4}$ &
${\small 5}$ & ${\small 6}$ & ${\small 7}$ & ${\small 8}$ & ${\small 9}%
$\\\hline
${\small \rho(k,0,2)}$ & ${\small 0.24135}$ & ${\small 0.18319}$ &
${\small 0.14551}$ & ${\small 0.11736}$ & ${\small 0.09500}$ &
${\small 0.07640}$ & ${\small 0.06050}$ & ${\small 0.04660}$ &
${\small 0.03410}$\\\hline\hline
\end{tabular}
\caption{Table Caption 3}\label{TableKey3}%
%TCIMACRO{\TeXButton{E}{\end{table}}}%
%BeginExpansion
\end{table}%
%EndExpansion%
\begin{align}
\rho(k,L,2) &  =\frac{10^{\left(  -L\right)  }}{9}\left(  k\ln(k)-(k+1)\ln
(k+1)+1+\frac{\left(  9L+10\right)  }{9}\ln(10)\right)  ,\label{Prod2}\\
10^{L} &  \leq k\leq10^{L+1}-1,L=0,1,\ldots\nonumber
\end{align}


\subsubsection{DSLD of the Production of Three Numbers from a UD}%

%TCIMACRO{\TeXButton{\label{Pro3}}{\label{Pro3}}}%
%BeginExpansion
\label{Pro3}%
%EndExpansion


From Appendix
%TCIMACRO{\TeXButton{\ref{PRNUD}}{\ref{PRNUD}} }%
%BeginExpansion
\ref{PRNUD}
%EndExpansion
we have
\[
f_{Z}(z,3)=\frac{\ln(z)^{2}}{2},0\leq z\leq1.
\]
%

\[
F_{Z}(z,3)=\frac{z}{2}\left(  \ln(z)^{2}-2\ln(z)+2\right)  ,0\leq z\leq1.
\]%
\begin{align*}
\rho(k,0)  &  =\frac{1}{18}\left(  (k+1)\ln(k+1)^{2}-k\ln(k)^{2}\right) \\
&  -\left(  \frac{1}{9}+\frac{10}{81}\ln(10)\right)  \left(  (k+1)\ln
(k+1)-k\ln(k)\right) \\
&  +\frac{55}{729}\left(  \ln(10)\right)  ^{2}+\frac{10}{81}\ln(10)+\frac
{1}{9}%
\end{align*}
%

%TCIMACRO{\TeXButton{B}{\begin{table}[!htbp] \centering}}%
%BeginExpansion
\begin{table}[!htbp] \centering
%EndExpansion%
\begin{tabular}
[c]{|c|c|c|c|c|c|c|c|c|c|}\hline\hline
${\small k}$ & ${\small 1}$ & ${\small 2}$ & ${\small 3}$ & ${\small 4}$ &
${\small 5}$ & ${\small 6}$ & ${\small 7}$ & ${\small 8}$ & ${\small 9}%
$\\\hline
${\small \rho(k,0)}$ & ${\small 0.30066}$ & ${\small 0.18817}$ &
${\small 0.13194}$ & ${\small 0.09864}$ & ${\small 0.07699}$ &
${\small 0.06290}$ & ${\small 0.05260}$ & ${\small 0.04630}$ &
${\small 0.04179}$\\\hline\hline
\end{tabular}
\caption{Table Caption 4}\label{TableKey4}%
%TCIMACRO{\TeXButton{E}{\end{table}}}%
%BeginExpansion
\end{table}%
%EndExpansion
%

\begin{align*}
\Omega(s)  &  =10^{s}\left(  A+Bs+Cs^{2}\right)  ,\\
C  &  =\frac{\ln(10)^{2}}{18},B=-\frac{\ln(10)}{81}(9+10\ln(10)),\\
A  &  =\frac{1}{9}+\frac{10}{81}\ln(10)+\frac{55}{729}\ln(10)^{2}\\
\Omega(1)-\Omega(0)  &  =1.
\end{align*}%
\begin{align*}
s_{1}  &  =\log(k+1)-\lfloor\log(k)\rfloor,\\
s  &  =\log(k)-\lfloor\log(k)\rfloor.
\end{align*}%
\[
\rho(k)=\Omega(s_{1})-\Omega(s),k\text{ \ any positive integer}%
\]


\subsubsection{DSLD of the Production of $n$ Numbers from a UD}%

%TCIMACRO{\TeXButton{PnNfUD}{\label{PnNfUD}}}%
%BeginExpansion
\label{PnNfUD}%
%EndExpansion


From Appendix
%TCIMACRO{\TeXButton{\ref{PRNUD}}{\ref{PRNUD}}}%
%BeginExpansion
\ref{PRNUD}%
%EndExpansion%
\begin{align*}
z  &  =%
%TCIMACRO{\dprod \limits_{i=1}^{n}}%
%BeginExpansion
{\displaystyle\prod\limits_{i=1}^{n}}
%EndExpansion
x_{i},\\
z  &  \leq x_{1}\leq1,\\
x_{n}  &  =z\left(
%TCIMACRO{\dprod \limits_{i=1}^{n-1}}%
%BeginExpansion
{\displaystyle\prod\limits_{i=1}^{n-1}}
%EndExpansion
x_{i}\right)  ^{-1},\\
z\left(
%TCIMACRO{\dprod \limits_{i=1}^{j-1}}%
%BeginExpansion
{\displaystyle\prod\limits_{i=1}^{j-1}}
%EndExpansion
x_{i}\right)  ^{-1}  &  \leq x_{j}\leq1,j=2,\ldots,n-1,\\
J  &  =\frac{\partial(x_{1},\ldots,x_{n})}{\partial(x_{1},\ldots,z)}%
=\frac{\partial x_{n}}{\partial z}=\left(
%TCIMACRO{\dprod \limits_{i=1}^{n-1}}%
%BeginExpansion
{\displaystyle\prod\limits_{i=1}^{n-1}}
%EndExpansion
x_{i}\right)  ^{-1}.
\end{align*}%
\[
f_{Z}(z,n)=%
%TCIMACRO{\dint \limits_{x_{1}=z}^{1}}%
%BeginExpansion
{\displaystyle\int\limits_{x_{1}=z}^{1}}
%EndExpansion%
%TCIMACRO{\dint \limits_{x_{2}=z/x_{1}}^{1}}%
%BeginExpansion
{\displaystyle\int\limits_{x_{2}=z/x_{1}}^{1}}
%EndExpansion
\ldots%
%TCIMACRO{\dint \limits_{x_{n-1}=z/\left(  x_{1}\ldots x_{n-2}\right)  }^{1}}%
%BeginExpansion
{\displaystyle\int\limits_{x_{n-1}=z/\left(  x_{1}\ldots x_{n-2}\right)  }%
^{1}}
%EndExpansion
J^{-1}dx_{n-1}\ldots dx_{2}dx_{1},2\leq n.
\]%
\[
f_{Z}(z,n)=\frac{(-\ln(z))^{(n-1)}}{(n-1)!},0\leq z\leq1.
\]%
\[
F_{Z}(z,n)=\frac{\Gamma(n,-\ln(z))}{\Gamma(n)},0\leq z\leq1.
\]
where%
\begin{align*}
\Gamma(a,x)  &  =%
%TCIMACRO{\dint \limits_{x}^{\infty}}%
%BeginExpansion
{\displaystyle\int\limits_{x}^{\infty}}
%EndExpansion
t^{a-1}\exp(-t)dt,\\
\Gamma(a)  &  =%
%TCIMACRO{\dint \limits_{0}^{\infty}}%
%BeginExpansion
{\displaystyle\int\limits_{0}^{\infty}}
%EndExpansion
t^{a-1}\exp(-t)dt=(a-1)!.
\end{align*}
Then as $n\rightarrow+\infty$, we have, from Appendix
%TCIMACRO{\TeXButton{\ref{PRNUD}}{\ref{PRNUD}}}%
%BeginExpansion
\ref{PRNUD}%
%EndExpansion
%

\begin{align*}
\Omega\left(  s\right)   &  =s+const,\\
\rho(k,n  &  \rightarrow+\infty)=\log(k+1)-\log(k).
\end{align*}


This is the distribution of the NBL. The table presents the values of
$\rho(k,n\rightarrow+\infty)$ for some parameters%

%TCIMACRO{\TeXButton{B}{\begin{table}[!htbp] \centering}}%
%BeginExpansion
\begin{table}[!htbp] \centering
%EndExpansion%
\begin{tabular}
[c]{|c|c|c|c|c|c|c|c|c|}\hline\hline
${\small n}$ & ${\small 2}$ & ${\small 3}$ & ${\small 4}$ & ${\small 5}$ &
${\small 6}$ & ${\small 7}$ & ${\small 8}$ & ${\small 9}$\\\hline
${\small \rho(1,n)}$ & ${\small .24135}$ & ${\small .30066}$ &
${\small .30764}$ & ${\small .30279}$ & ${\small .30068}$ & ${\small .30074}$
& ${\small .30100}$ & ${\small .30106}$\\\hline
${\small \rho(9,n)}$ & ${\small .03418}$ & ${\small .04167}$ &
${\small .04585}$ & ${\small .04619}$ & ${\small .04585}$ & ${\small .04573}$
& ${\small .04573}$ & ${\small .04573}$\\\hline\hline
\end{tabular}
\caption{Table Caption 5}\label{TableKey5}%
%TCIMACRO{\TeXButton{E}{\end{table}}}%
%BeginExpansion
\end{table}%
%EndExpansion
%

\[
\log(2)=0.30103,1-\log(9)=0.045757.
\]


\section{Analysis of DSLD of Other Continuous Distributions}%

%TCIMACRO{\TeXButton{OtherContDist}{\label{OtherContDist}}}%
%BeginExpansion
\label{OtherContDist}%
%EndExpansion


\subsection{DSLD for LogNormal Distribution}%

%TCIMACRO{\TeXButton{LogNormDis}{\label{LogNormDist}}}%
%BeginExpansion
\label{LogNormDist}%
%EndExpansion%
\begin{align*}
f(z)  &  =\frac{1}{\sigma z\sqrt{2\pi}}\exp\left(  -\frac{\left(  \ln
(z)-\mu\right)  ^{2}}{2\sigma^{2}}\right)  ,\\
F(z)  &  =\frac{1}{2}+\frac{1}{2}\operatorname{erf}\left(  \frac{\left(
\ln(z)-\mu\right)  \sqrt{2}}{2\sigma}\right)  ,0\leq z<\infty.
\end{align*}
Then%
\begin{equation}
Q_{1}\left(  s,\sigma\right)  =const+%
%TCIMACRO{\dsum \limits_{n=1}^{\infty}}%
%BeginExpansion
{\displaystyle\sum\limits_{n=1}^{\infty}}
%EndExpansion
\left(  A_{n}\sin\left(  ns\right)  +B_{n}\cos\left(  ns\right)  \right)  .
\label{LogNorm1}%
\end{equation}%
\begin{align*}
\digamma &  =1,\\
A_{n}  &  =\frac{1}{n\pi}\exp\left(  -\frac{2\pi^{2}\sigma^{2}}{\ln(10)^{2}%
}n^{2}\right)  ,B_{n}=0.
\end{align*}%
\begin{align}
\rho(k,\mu,\sigma &  \rightarrow0)=\log(k+1)-\log(k)+Q_{1}\left(  s_{1}%
,\sigma\right)  -Q_{1}\left(  s,\sigma\right)  .\label{LogNorm2}\\
&  s_{1}=2\pi\left(  \log(1+k)-\frac{\mu}{\ln(10)}\right)  ,s=2\pi\left(
\log(k)-\frac{\mu}{\ln(10)}\right)  .\nonumber
\end{align}


From this we obtain that $\rho(k,\mu,\sigma)$ is a periodic function of $\mu
$\ \ with period $\ln(10).$

\bigskip As $\sigma\rightarrow0$ ($\mu=0$) we have%
\[
\rho(k)=\Omega(\log(k+1)-\left\lfloor \log(k)\right\rfloor )-\Omega
(\log(k)-\left\lfloor \log(k)\right\rfloor )
\]%
\[
\rho(k)=\left\{
\begin{array}
[c]{cc}%
\frac{1}{2}, & \log(k)=\left\lfloor \log(k)\right\rfloor ,\\
\frac{1}{2}, & \log(k+1)=\left\lfloor \log(k)\right\rfloor +1.
\end{array}
\right.
\]
Where%
\begin{align*}
\Omega(x)  &  =\left\{
\begin{array}
[c]{cc}%
-\frac{1}{2}, & x=0,\\
\frac{1}{2}, & x=1,\\
0, & \text{other}.
\end{array}
\right.  ,\\
\Omega(1)-\Omega(0)  &  =1.
\end{align*}


As a result of the summation,%
\begin{equation}
Q_{1}\left(  s,\sigma\right)  =A_{0}+\arctan\left(  \frac{\beta\sin(2\pi
s)}{1-\beta\cos(2\pi s)}\right)  ,\beta=\exp\left(  -\frac{2\pi^{2}\sigma^{2}%
}{\ln(10)^{2}}\right)  .
\end{equation}


If we compare the results of our approach and the numerical results of
stochastic modeling
%TCIMACRO{\TeXButton{citeFormann}{\cite{formann0}}}%
%BeginExpansion
\cite{formann0}%
%EndExpansion
. \ We can see that despite the fact that it was used in the numerical
modeling of $10^{7}$ samples, the accuracy of the approximation is quite poor.
The approximation errors lie in the range of $15\%-100\%$.

\subsection{DSLD for a First Family of Distributions}%

%TCIMACRO{\TeXButton{DfED}{\label{DfWD}}}%
%BeginExpansion
\label{DfWD}%
%EndExpansion


Here we will look at one type of density function, that underlying
distributions that describe a number of important and frequently used
distributions
%TCIMACRO{\TeXButton{\cite{walck} }{\cite{walck} }}%
%BeginExpansion
\cite{walck}
%EndExpansion
.%
\[
\digamma=%
%TCIMACRO{\dint \limits_{z=0}^{\infty}}%
%BeginExpansion
{\displaystyle\int\limits_{z=0}^{\infty}}
%EndExpansion
f(z)dz.
\]


The model distribution is a continuous probability distribution with
probability density function given by%
\begin{equation}
f(z)=\left\{
\begin{array}
[c]{cc}%
0, & z<0\\
\digamma\beta\left(  \frac{z}{a}\right)  ^{\alpha}\exp\left(  -\left(
\frac{z}{a}\right)  ^{\beta}\right)  a^{(-1)}\Gamma\left(  \frac{\alpha
+1}{\beta}\right)  ^{-1}, & 0\leq z,0<a,0<\alpha,0<\beta.
\end{array}
\right.  \label{defWD}%
\end{equation}


Because from general form we get
\begin{equation}
\rho(k,a,\alpha,\beta)=%
%TCIMACRO{\dsum \limits_{m=-\infty}^{\infty}}%
%BeginExpansion
{\displaystyle\sum\limits_{m=-\infty}^{\infty}}
%EndExpansion
\left(  F(\frac{10^{m}(k+1)}{a}-F\left(  \frac{10^{m}k}{a}\right)  \right)  .
\label{WA_rho}%
\end{equation}


From the leading term on the right hand side of (\ref{WA_rho}) we have%

\begin{align*}
&  m+\log(k)-\log(a)\\
&  =m+\lfloor\log(k)\rfloor-\lfloor\log(a)\rfloor\\
&  +\log(k)-\lfloor\log(k)\rfloor-\log(a)+\lfloor\log(a)\rfloor
\end{align*}
and we have \ (\ref{WA_rho}) with an infinite range of summation. Now we can
introduce a new dummy variable $m^{\prime}$
\[
m=m^{\prime}-\lfloor\log(k)\rfloor+\lfloor\log(a)\rfloor.
\]
After some calculation we have%
\begin{align*}
\rho(k,a,\alpha,\beta)  &  =\log(k+1)-\log(k)+\\
&
%TCIMACRO{\dsum \limits_{n=1}^{\infty}}%
%BeginExpansion
{\displaystyle\sum\limits_{n=1}^{\infty}}
%EndExpansion
A_{n}(\alpha,\beta)\left(  \sin\left(  ns_{1}\right)  -\sin\left(  ns\right)
\right)  +\\
&
%TCIMACRO{\dsum \limits_{n=1}^{\infty}}%
%BeginExpansion
{\displaystyle\sum\limits_{n=1}^{\infty}}
%EndExpansion
B_{n}(\alpha,\beta)\left(  \cos\left(  ns_{1}\right)  -\cos\left(  ns\right)
\right)  ,\\
s_{1}  &  =2\pi\log\left(  \frac{k+1}{a}\right)  ,s=2\pi\log\left(  \frac
{k}{a}\right)  .
\end{align*}
\bigskip

Where%
\begin{align}
A_{n}(\alpha,\beta)+iB_{n}(\alpha,\beta)  &  =\digamma\frac{1}{\pi n}%
\Gamma\left(  \frac{\alpha}{\beta}+\frac{1}{\beta}-Y\imath\right)
\Gamma\left(  \frac{\alpha}{\beta}+\frac{1}{\beta}\right)  ^{-1},\\
Y  &  =\frac{2\pi n}{\beta\ln(10)},
\end{align}


From this we obtain that $\rho(k,a,\alpha,\beta)$\ is a periodic function of
$\log\left(  a\right)  $ with period $1$.

A useful property of the $\Gamma$ function is%
\begin{equation}
\frac{\left\vert \Gamma\left(  x+\imath y\right)  \right\vert }{\left\vert
\Gamma\left(  x\right)  \right\vert }\leq1,\text{ for any real }x\text{ and
}y\text{. } \label{Gamma1}%
\end{equation}


\noindent then%
\begin{equation}
\sqrt{A_{n}(\alpha,\beta)^{2}+B_{n}(\alpha,\beta)^{2}}\leq\frac{\digamma}{\pi
n}. \label{AB1}%
\end{equation}


\subsection{Examples}

\begin{itemize}
\item \textbf{Exponential Distribution}
\end{itemize}

%

\[
f(z,\lambda)=\frac{1}{\lambda}\exp\left(  -\frac{z}{\lambda}\right)
,0<\lambda,0\leq z.
\]%
\begin{align*}
\digamma &  =1,\alpha=0,\beta=1,\\
A_{n}+iB_{n}  &  =\frac{1}{\pi n}\Gamma\left(  1-\imath Y\right)
\Gamma\left(  1\right)  ^{-1},\\
\Lambda(n)^{2}  &  =2\left(  n\ln(10)\sinh\left(  \frac{2\pi^{2}n}{\ln
(10)}\right)  \right)  ^{-1},\\
s_{1}  &  =2\pi\log\left(  \frac{k+1}{\lambda}\right)  ,s=2\pi\log\left(
\frac{k}{\lambda}\right)  .
\end{align*}%
\begin{align*}
\Lambda(1)  &  =.01813,A_{1}=.01308,B_{1}=-.01256,\\
\Lambda(2)  &  =1.768\times10^{-4},A_{2}=2.484\times10^{-5},B_{2}%
=1.749\times10^{-4}.
\end{align*}


\begin{itemize}
\item \textbf{Weibull Distribution}
\[
f(z,\nu)=\frac{\beta}{a}\left(  \frac{z}{a}\right)  ^{\beta-1}\exp\left(
-\left(  \frac{z}{a}\right)  ^{\beta}\right)  ,0\leq z.
\]%
\begin{align*}
\digamma &  =1,\alpha=\beta-1,0<\beta,\\
A_{n}(\beta)-iB_{n}(\beta)  &  =\frac{2i}{\beta\ln(10)}\Gamma\left(
1+\imath\frac{Y}{\beta}\right)  ,\\
\Lambda(n,\beta)^{2}  &  =\frac{2}{\beta\ln(10)n}\sinh\left(  \frac{2\pi^{2}%
n}{\beta\ln(10)}\right)  ^{-1},\\
s_{1}  &  =2\pi\log\left(  \frac{k+1}{a}\right)  ,s=2\pi\log\left(  \frac
{k}{a}\right)  .
\end{align*}%
\begin{align*}
\Lambda(1,2)  &  =.1093,A_{1}(2)=.1068,B_{1}(2)=.02362,\\
\Lambda(2,2)  &  =9.065\times10^{-3},A_{2}(2)=.006552,B_{2}(2)=-.006283.
\end{align*}%
\begin{align*}
\rho(k,a,n)  &  =\log(k+1)-\log(k)\\
&  +%
%TCIMACRO{\dsum \limits_{j=1}^{n}}%
%BeginExpansion
{\displaystyle\sum\limits_{j=1}^{n}}
%EndExpansion
A_{j}(\sin(js_{1})-\sin(js))\\
&  +%
%TCIMACRO{\dsum \limits_{j=1}^{n}}%
%BeginExpansion
{\displaystyle\sum\limits_{j=1}^{n}}
%EndExpansion
B_{j}(\cos(js_{1})-\cos(js)),\\
s_{1}  &  =2\pi\frac{\log(k+1)}{a},s=2\pi\frac{\log(k)}{a}.
\end{align*}


\noindent with $1\ll\beta$ we have%
\begin{align*}
A_{n}(\beta)  &  =\frac{1}{\pi n}+o\left(  \frac{1}{\beta}\right)  ,\\
B_{n}(\beta)  &  =\frac{2\gamma}{\beta\ln(10)}+o\left(  \frac{1}{\beta^{2}%
}\right)  ,\gamma=.57721
\end{align*}


\item \textbf{ChiSquare Distribution}%
\[
f(z,\nu)=\left(  \frac{z}{2}\right)  ^{\frac{\nu}{2}-1}\exp\left(  -\frac
{z}{2}\right)  \left(  2\Gamma\left(  \frac{\nu}{2}\right)  \right)
^{-1},0\leq z.
\]%
\[
\digamma=1,a=2,\alpha=\frac{\nu}{2}-1,\beta=1,
\]%
\begin{align}
A_{n}(\nu)-iB_{n}(\nu)  &  =\frac{1}{\pi n}\Gamma\left(  \frac{\nu}{2}%
+Y\imath\right)  \Gamma\left(  \frac{\nu}{2}\right)  ^{-1},\\
Y  &  =\frac{2\pi n}{\ln(10)},\\
s_{1}  &  =2\pi\log\left(  \frac{k+1}{2}\right)  ,s=2\pi\log\left(  \frac
{k}{2}\right)  .
\end{align}%
\begin{align*}
A_{1}(7)  &  =-.1041,B_{1}(7)=.01788,\\
A_{1}(7)  &  =.0005160,B_{1}(7)=-.004204.
\end{align*}%
\begin{align*}
\rho(k,a,n)  &  =\log(k+1)-\log(k)\\
&  +%
%TCIMACRO{\dsum \limits_{j=1}^{n}}%
%BeginExpansion
{\displaystyle\sum\limits_{j=1}^{n}}
%EndExpansion
A_{j}(\sin(js_{1})-\sin(js))\\
&  +%
%TCIMACRO{\dsum \limits_{j=1}^{n}}%
%BeginExpansion
{\displaystyle\sum\limits_{j=1}^{n}}
%EndExpansion
B_{j}(\cos(js_{1})-\cos(js)),\\
s_{1}  &  =2\pi\frac{\log(k+1)}{2},s=2\pi\frac{\log(k)}{2}.
\end{align*}


With $1\ll\nu$
\begin{align*}
A_{n}(\nu)  &  =\frac{1}{\pi n}\cos\left(  2\pi n\log\left(  \frac{\nu}%
{2}\right)  \right)  ,\\
B_{n}(\nu)  &  =-\frac{1}{\pi n}\sin\left(  2\pi n\log\left(  \frac{\nu}%
{2}\right)  \right)  .\\
Q_{1}(k,1  &  \ll\nu)=%
%TCIMACRO{\dsum \limits_{n=1}^{\infty}}%
%BeginExpansion
{\displaystyle\sum\limits_{n=1}^{\infty}}
%EndExpansion
\frac{1}{\pi n}\sin\left(  2\pi n\log\left(  \frac{k}{\nu}\right)  \right)
\end{align*}%
\begin{align*}%
%TCIMACRO{\dsum \limits_{n=1}^{\infty}}%
%BeginExpansion
{\displaystyle\sum\limits_{n=1}^{\infty}}
%EndExpansion
\frac{1}{\pi n}\sin\left(  2\pi nX\right)   &  =\frac{1}{\pi}\arctan\left(
\frac{\sin(2\pi X)}{1-\cos(2\pi X)}\right) \\
&  =\frac{1}{2}-(X-\left\lfloor X\right\rfloor ).
\end{align*}


\item \textbf{Exponential Distribution}%
\[
f(z,\nu)=\frac{1}{a}\exp\left(  -\frac{z}{a}\right)  ,0\leq z.
\]%
\begin{align}
\digamma &  =1,\beta=1,\alpha=0,\\
A_{n}-iB_{n} &  =\frac{2i}{\ln(10)}\Gamma\left(  Y\imath\right)  ,\nonumber\\
A_{n}{}^{2}+B_{n}{}^{2} &  =\frac{2}{n\ln(10)\sinh(\pi Y)}.\nonumber\\
Y &  =\frac{2\pi n}{\ln(10)}.\nonumber
\end{align}%
\begin{align}
A_{1} &  =.01308,B_{1}=-.01256,\\
A_{2} &  =-.00002436,B_{2}=.000175.\nonumber
\end{align}%
\begin{align*}
\rho(k,a,n) &  =\log(k+1)-\log(k)\\
&  +%
%TCIMACRO{\dsum \limits_{j=1}^{n}}%
%BeginExpansion
{\displaystyle\sum\limits_{j=1}^{n}}
%EndExpansion
A_{j}(\sin(js_{1})-\sin(js))\\
&  +%
%TCIMACRO{\dsum \limits_{j=1}^{n}}%
%BeginExpansion
{\displaystyle\sum\limits_{j=1}^{n}}
%EndExpansion
B_{j}(\cos(js_{1})-\cos(js)),\\
s_{1} &  =2\pi\frac{\log(k+1)}{a},s=2\pi\frac{\log(k)}{a}.
\end{align*}%
%TCIMACRO{\TeXButton{B}{\begin{figure}[H] \centering}}%
%BeginExpansion
\begin{figure}[H] \centering
%EndExpansion%
%TCIMACRO{\FRAME{itbpFU}{5.4509in}{3.9487in}{0in}{\Qcb{$\ln(10)k(\rho
%(k)-\log(k+1)+\log(k))$ vs $s$=$\log(k)$}}{}{expdist1.png}%
%{\special{ language "Scientific Word";  type "GRAPHIC";
%maintain-aspect-ratio TRUE;  display "USEDEF";  valid_file "F";
%width 5.4509in;  height 3.9487in;  depth 0in;  original-width 5.393in;
%original-height 3.9003in;  cropleft "0";  croptop "1";  cropright "1";
%cropbottom "0";  filename 'ExpDist1.png';file-properties "XNPEU";}} }%
%BeginExpansion
{\parbox[b]{5.4509in}{\begin{center}
\includegraphics[
natheight=3.900300in,
natwidth=5.393000in,
height=3.9487in,
width=5.4509in
]%
{C:/Users/Vladimir/Documents/BENFORD/AAA/FOR ProofReading INTERNET/From PRS 4  (ver 9)/ExpDist1__3.png}%
\\
$\ln(10)k(\rho(k)-\log(k+1)+\log(k))$ vs $s$=$\log(k)$%
\end{center}}}
%EndExpansion
\caption{Exponential Distribution}\label{figureKey3}%
%TCIMACRO{\TeXButton{E}{\end{figure}}}%
%BeginExpansion
\end{figure}%
%EndExpansion


\item \textbf{Gamma Distribution}%
\[
f(z,a,\delta)=\frac{1}{a}\left(  \frac{z}{a}\right)  ^{\delta-1}\exp\left(
-\frac{z}{a}\right)  \Gamma\left(  \delta\right)  ^{-1},0\leq z.
\]%
\[
\digamma=1,\alpha=\delta-1,\beta=1,0<\delta,
\]%
\begin{align}
A_{n}(\delta)-iB_{n}(\delta) &  =\frac{1}{\pi n}\Gamma\left(  \delta
+Yi\right)  \Gamma\left(  \delta\right)  ^{-1},\\
A_{n}(\delta)^{2}+B_{n}(\delta)^{2} &  =\frac{1}{n^{2}\pi^{2}}\Gamma\left(
\delta+Yi\right)  \Gamma\left(  \delta-Yi\right)  \Gamma\left(  \delta\right)
^{-2},.\\
Y &  =\frac{2\pi n}{\ln(10)},\\
s_{1} &  =2\pi\log\left(  \frac{k+1}{a}\right)  ,s=2\pi\log\left(  \frac{k}%
{a}\right)  .
\end{align}%
\begin{align}
A_{1}(2) &  =-.02117,B_{1}(2)=-.04829,\\
A_{2}(2) &  =.00093,B_{2}(2)=.0003105.
\end{align}


We have a non-monotonic behavior of $\rho(k,a,\delta)$ as a function of
$\delta$
\end{itemize}

%

%TCIMACRO{\TeXButton{B}{\begin{table}[!htbp] \centering}}%
%BeginExpansion
\begin{table}[!htbp] \centering
%EndExpansion%
\begin{tabular}
[c]{|c|c|c|c|c|c|c|c|c|c|}\hline\hline
$\delta$ & ${\small 1}$ & ${\small 2}$ & ${\small 3}$ & ${\small 4}$ &
${\small 5}$ & ${\small 6}$ & ${\small 7}$ & ${\small 8}$ & ${\small 9}%
$\\\hline
${\small \rho(1,1,\delta)}$ & ${\small .32960}$ & ${\small .3432}$ &
${\small .2468}$ & ${\small .1342}$ & ${\small .0782}$ & ${\small .0829}$ &
${\small .1342}$ & ${\small .2205}$ & {\small .}$3309$\\\hline
${\small \rho(9,1,\delta)}$ & ${\small .048923}$ & ${\small .038394}$ &
${\small .020912}$ & ${\small .016467}$ & ${\small .027004}$ &
${\small .04886}$ & ${\small .07675}$ & ${\small .10370}$ & ${\small .12285}%
$\\\hline\hline
\end{tabular}
\caption{Table Caption 6}\label{TableKey6}%
%TCIMACRO{\TeXButton{E}{\end{table}}}%
%BeginExpansion
\end{table}%
%EndExpansion


\bigskip

We can prove that as $\delta\rightarrow+\infty$, the function $\rho
(k,a,\delta)$ has $1$-periodic behavior in $\log(\delta).$%

%TCIMACRO{\TeXButton{B}{\begin{figure}[H] \centering}}%
%BeginExpansion
\begin{figure}[H] \centering
%EndExpansion%
%TCIMACRO{\FRAME{itbpFU}{5.4509in}{3.9487in}{0in}{\Qcb{$\rho(k=1,a=1,\delta
%)/\log(2)$ vs $\ s=\log(\delta)$}}{}{gamman.png}%
%{\special{ language "Scientific Word";  type "GRAPHIC";
%maintain-aspect-ratio TRUE;  display "USEDEF";  valid_file "F";
%width 5.4509in;  height 3.9487in;  depth 0in;  original-width 5.393in;
%original-height 3.9003in;  cropleft "0";  croptop "1";  cropright "1";
%cropbottom "0";  filename 'GammaN.png';file-properties "XNPEU";}} }%
%BeginExpansion
{\parbox[b]{5.4509in}{\begin{center}
\includegraphics[
natheight=3.900300in,
natwidth=5.393000in,
height=3.9487in,
width=5.4509in
]%
{C:/Users/Vladimir/Documents/BENFORD/AAA/FOR ProofReading INTERNET/From PRS 4  (ver 9)/GammaN__4.png}%
\\
$\rho(k=1,a=1,\delta)/\log(2)$ vs $\ s=\log(\delta)$%
\end{center}}}
%EndExpansion
\caption{Gamma Distributio}\label{figureKey4}%
%TCIMACRO{\TeXButton{E}{\end{figure}}}%
%BeginExpansion
\end{figure}%
%EndExpansion


\bigskip Similar analyses are possible for other distributions, like the
$\chi^{2}$-distribution, Student's $t$-distribution, \ the Fisher
\ $F$-distribution, and others.

\begin{itemize}
\item \textbf{Normal Distribution} with $\mu=0$%
\[
f(z,\sigma)=\frac{1}{\sigma\sqrt{2\pi}}\exp\left(  -\frac{z^{2}}{2\sigma^{2}%
}\right)  ,0\leq z.
\]%
\begin{equation}
\digamma=\frac{1}{2},a=\sigma\sqrt{2},\alpha=0,\beta=2, \label{Nomal0}%
\end{equation}%
\begin{align}
A_{n}-iB_{n}  &  =\frac{\digamma}{\pi n}\Gamma\left(  \frac{1}{2}%
+Y\imath\right)  \Gamma\left(  \frac{1}{2}\right)  ^{-1},Y=\frac{\pi n}%
{\ln(10)},\\
\Lambda(n)^{2}  &  =\left(  4\pi^{2}n^{2}\cosh\left(  \frac{\pi^{2}n}{\ln
(10)}\right)  \right)  ^{-1},\nonumber\\
s_{1}  &  =2\pi\log\left(  \frac{k+1}{\sigma\sqrt{2}}\right)  ,s=2\pi
\log\left(  \frac{k}{\sigma\sqrt{2}}\right)  .\nonumber
\end{align}%
\begin{align*}
\rho(k,\sigma,n)  &  =\frac{1}{2}\log\left(  1+\frac{1}{k}\right)  +%
%TCIMACRO{\dsum \limits_{j=1}^{n}}%
%BeginExpansion
{\displaystyle\sum\limits_{j=1}^{n}}
%EndExpansion
A_{n}\left(  \sin\left(  js_{1}\right)  -\sin\left(  js\right)  \right) \\
&  +%
%TCIMACRO{\dsum \limits_{j=1}^{n}}%
%BeginExpansion
{\displaystyle\sum\limits_{j=1}^{n}}
%EndExpansion
B_{n}\left(  \cos\left(  js_{1}\right)  -\cos\left(  js\right)  \right)  .
\end{align*}
From this we have that $\rho(k,\mu)$ is a periodic function of $\log\left(
\sigma\right)  $ with period $1.$%
\begin{align*}
\Lambda(1)  &  =.02640,A_{1}=.01621,B_{1}=.02081,\\
\Lambda(2)  &  =1.548\times10^{-3},A_{2}=1,546\times10^{-3},B_{2}%
=-4.002\times10^{-5},\\
\Lambda(3)  &  =1.210\times10^{-4}.
\end{align*}
We can compare the values of $\rho(k,\sigma,n)$, depending on how many terms
of the following are taken into account.%
\[
\frac{\rho(1,1,0)}{\rho(1,1,\infty)}-1=-.1627,\frac{\rho(1,1,1)}%
{\rho(1,1,\infty)}-1=-1.634\times10^{-2},\frac{\rho(1,1,2)}{\rho(1,1,\infty
)}-1=1.202\times10^{-6}%
\]

\end{itemize}

\subsection{A Second Family of Distributions}

Another class of distributions
%TCIMACRO{\TeXButton{\cite{walck} }{\cite{walck}} }%
%BeginExpansion
\cite{walck}
%EndExpansion
is%
\begin{equation}
f(z)=\left\{
\begin{array}
[c]{cc}%
0, & z<0\\%
\begin{array}
[c]{c}%
\digamma a^{(-1)}\beta\left(  \frac{z}{a}\right)  ^{\alpha}\left(  1+\left(
\frac{z}{a}\right)  ^{\beta}\right)  ^{-\delta}\Gamma\left(  \delta\right)
\Gamma\left(  \delta-\frac{\alpha}{\beta}-\frac{1}{\beta}\right)  ^{-1}\\
\Gamma\left(  \frac{\alpha}{\beta}+\frac{1}{\beta}\right)  ^{-1},
\end{array}
&
\begin{array}
[c]{c}%
0\leq z,0<a,0<\alpha,\\
0<\beta,0<\delta-\frac{\alpha}{\beta}-\frac{1}{\beta}.
\end{array}
\end{array}
\right.
\end{equation}%
\begin{align*}
\rho(k,a,\alpha,\beta,\delta)  &  =\digamma\left(  \log(k+1)-\log(k)\right)
+\\
&
%TCIMACRO{\dsum \limits_{n=1}^{\infty}}%
%BeginExpansion
{\displaystyle\sum\limits_{n=1}^{\infty}}
%EndExpansion
A_{n}(\alpha,\beta,\delta)\left(  \sin\left(  ns_{1}\right)  -\sin\left(
ns\right)  \right)  +\\
&
%TCIMACRO{\dsum \limits_{n=1}^{\infty}}%
%BeginExpansion
{\displaystyle\sum\limits_{n=1}^{\infty}}
%EndExpansion
B_{n}(\alpha,\beta,\delta)\left(  \cos\left(  ns_{1}\right)  -\cos\left(
ns\right)  \right)  ,\\
s_{1}  &  =2\pi n\log\left(  \frac{k+1}{a}\right)  ,s=2\pi n\log\left(
\frac{k}{a}\right)
\end{align*}
From this we have that $\rho(k,a,\alpha,\beta,\delta)$ is a periodic function
of $\log\left(  a\right)  $\ with period $1.$

Where%
\begin{align}
A_{n}(\alpha,\beta,\delta)-iB_{n}(\alpha,\beta,\delta)  &  =\digamma\frac
{1}{\pi n}\Gamma\left(  \delta-\frac{\alpha}{\beta}-\frac{1}{\beta}-Yi\right)
\Gamma\left(  \frac{\alpha}{\beta}+\frac{1}{\beta}+Yi\right) \nonumber\\
&  \Gamma\left(  \delta-\frac{\alpha}{\beta}-\frac{1}{\beta}\right)
^{-1}\Gamma\left(  \frac{\alpha}{\beta}+\frac{1}{\beta}\right)  ^{-1},\\
Y  &  =\frac{2\pi n}{\beta\ln(10)},i^{2}=-1,\nonumber
\end{align}


If
\[
\delta=\frac{2(\alpha+1)}{\beta}%
\]


\noindent then
\begin{align*}
A_{n}(\alpha,\beta)  &  =\digamma\frac{1}{\pi n}\Gamma\left(  \frac{\alpha
}{\beta}+\frac{1}{\beta}-Yi\right)  \Gamma\left(  \frac{\alpha}{\beta}%
+\frac{1}{\beta}+Yi\right) \\
&  \Gamma\left(  \frac{\alpha}{\beta}+\frac{1}{\beta}\right)  ^{-1}%
\Gamma\left(  \frac{\alpha}{\beta}+\frac{1}{\beta}\right)  ^{-1}\\
B_{n}(\alpha,\beta)  &  =0.
\end{align*}


\begin{itemize}
\item \textbf{Cauchy Distribution} with $\mu=0$%
\[
f(z,a)=\frac{1}{\pi a}\frac{1}{1+\left(  \frac{z}{a}\right)  ^{2}},0<a.
\]%
\[
\digamma=\frac{1}{2},\alpha=0,\beta=2,\delta=1
\]%
\begin{align*}
A_{n}  &  =\frac{\digamma}{\pi n}\Gamma\left(  \frac{1}{2}-Y\imath\right)
\Gamma\left(  \frac{1}{2}+Y\imath\right)  \Gamma\left(  \frac{1}{2}\right)
^{-2}\\
&  =\digamma\left(  \pi n\cosh\left(  \frac{\pi^{2}n}{\ln(10)}\right)
\right)  ^{-1},Y=\frac{\pi n}{\ln(10)},\\
B_{n}  &  =0.
\end{align*}%
\begin{equation}
A_{1}=.004380,A_{2}=.000063012.
\end{equation}
\bigskip

\item \textbf{$F$-Ratio Distribution}%
\[
f(z)=\left(  \frac{\nu}{\omega}\right)  ^{\frac{\nu}{2}}z^{\frac{\nu}{2}%
-1}\left(  1+\frac{z\nu}{\omega}\right)  ^{-\delta}\Gamma\left(
\delta\right)  \Gamma\left(  \delta-\frac{\alpha}{\beta}-\frac{1}{\beta
}\right)  ^{-1},\delta=\frac{\nu}{2}+\frac{\omega}{2},0<\nu,0<\omega.
\]%
\[
\digamma=1,a=\frac{\omega}{\nu},\alpha=\frac{\nu}{2}-1,\beta=1,\delta
=\frac{\nu}{2}+\frac{\omega}{2}%
\]%
\begin{align}
A_{n}(\nu,\omega)-iB_{n}(\nu,\omega)  &  =\frac{1}{\pi n}\Gamma\left(
\frac{\omega}{2}-Y\imath\right)  \Gamma\left(  \frac{\nu}{2}+Y\imath\right) \\
&  \Gamma\left(  \frac{\omega}{2}\right)  ^{-1}\Gamma\left(  \frac{\nu}%
{2}\right)  ^{-1},\nonumber\\
Y  &  =\frac{2\pi n}{\ln(10)},\nonumber\\
s_{1}  &  =2\pi\log\left(  \frac{\nu(k+1)}{\omega}\right)  ,s=2\pi\log\left(
\frac{\nu k}{\omega}\right)  .\nonumber
\end{align}


If $\omega=\nu$, we have%
\begin{align*}
A_{n}(\nu,\nu)  &  =\frac{1}{\pi n}\Gamma\left(  \frac{\nu}{2}-Yi\right)
\Gamma\left(  \frac{\nu}{2}+Yi\right)  \Gamma\left(  \frac{\nu}{2}\right)
^{-2},\\
B_{n}(\nu,\nu)  &  =0,Y=\frac{2\pi n}{\ln(10)}.
\end{align*}%
\begin{equation}
A_{1}(3,3)=.003705,A_{2}(3,3)=.00000136.
\end{equation}

\end{itemize}

One example of this type of distribution is the ratio of two positive random
numbers ($\alpha=0,\beta=2,\delta=1$).

\subsubsection{DSLD for the Ratio of two Positive Numbers}%

%TCIMACRO{\TeXButton{DofLD2P}{\label{DofLD2P}}}%
%BeginExpansion
\label{DofLD2P}%
%EndExpansion


If we take any two positive numbers $x$ and $y$, then what is the distribution
of the leading significant digits of $\frac{y}{x}$?

Introduce new variables $R$ and $\varphi$ (where $0<\varphi<\frac{\pi}{2}%
,0<R$)
\begin{align*}
x  &  =R\cos(\varphi),y=R\sin(\varphi),\\
\frac{y}{x}  &  =\tan(\varphi).
\end{align*}


Then for the leading digit $k$ we have%
\[
10^{m}k<\tan(\varphi_{k})<10^{m}(k+1),1\leq k\leq9,-\infty<m<\infty,m\text{ is
integer.}%
\]
and so%
\[
\arctan(10^{m}k)<\varphi_{k}<\arctan(10^{m}(k+1)).
\]
%

\begin{equation}
\varphi_{k}=%
%TCIMACRO{\dsum \limits_{m=-\infty}^{\infty}}%
%BeginExpansion
{\displaystyle\sum\limits_{m=-\infty}^{\infty}}
%EndExpansion
\left(  \arctan(10^{m}(k+1))-\arctan(10^{m}k)\right)  . \label{Prob0}%
\end{equation}
$\rho(k)$ of leading digit $k$ is%
\begin{equation}
\rho(k)=\frac{2}{\pi}\varphi_{k}. \label{Prob1}%
\end{equation}
%

%TCIMACRO{\TeXButton{B}{\begin{table}[!htbp] \centering}}%
%BeginExpansion
\begin{table}[!htbp] \centering
%EndExpansion%
\begin{tabular}
[c]{|c|c|c|c|c|c|c|c|c|c|}\hline\hline
${\small k}$ & ${\small 1}$ & ${\small 2}$ & ${\small 3}$ & ${\small 4}$ &
${\small 5}$ & ${\small 6}$ & ${\small 7}$ & ${\small 8}$ & ${\small 9}%
$\\\hline
${\small \rho(k)}$ & ${\small .3092}$ & ${\small .1691}$ & ${\small .1184}$ &
${\small .09361}$ & ${\small .07903}$ & ${\small .06824}$ & ${\small .06023}$
& ${\small .05394}$ & ${\small .04811}$\\\hline\hline
\end{tabular}
\caption{Table Caption 7}\label{TableKey7}%
%TCIMACRO{\TeXButton{E}{\end{table}}}%
%BeginExpansion
\end{table}%
%EndExpansion


\bigskip It is clear that this distribution is connected with the Cauchy
distribution $\left(  PDF=\frac{1}{\pi}\frac{1}{1+z^{2}}\right)  $

Here we present some areas of $\frac{2}{\pi}\left(  \arctan(10^{m}%
(k+1))-\arctan(10^{m}k)\right)  $ for different $k$\ and $m$:\ %

%TCIMACRO{\TeXButton{B}{\begin{figure}[H] \centering}}%
%BeginExpansion
\begin{figure}[H] \centering
%EndExpansion%
%TCIMACRO{\FRAME{itbpFU}{4.3474in}{0.4238in}{0in}{\Qcb{$k=1$ and
%\ $m=-2,-1,0,1$}}{}{ratio2_k_1.png}{\special{ language "Scientific Word";
%type "GRAPHIC";  maintain-aspect-ratio TRUE;  display "USEDEF";
%valid_file "F";  width 4.3474in;  height 0.4238in;  depth 0in;
%original-width 4.2964in;  original-height 0.3943in;  cropleft "0";
%croptop "1";  cropright "1";  cropbottom "0";
%filename 'Ratio2_k_1.png';file-properties "XNPEU";}} }%
%BeginExpansion
{\parbox[b]{4.3474in}{\begin{center}
\includegraphics[
natheight=0.394300in,
natwidth=4.296400in,
height=0.4238in,
width=4.3474in
]%
{C:/Users/Vladimir/Documents/BENFORD/AAA/FOR ProofReading INTERNET/From PRS 4  (ver 9)/Ratio2_k_1__5.png}%
\\
$k=1$ and \ $m=-2,-1,0,1$%
\end{center}}}
%EndExpansion%
%TCIMACRO{\FRAME{itbpFU}{4.3474in}{0.4238in}{0in}{\Qcb{$k=2$ and
%\ $m=-2,-1,0,1,2$}}{}{ratio2_k_2.png}{\special{ language "Scientific Word";
%type "GRAPHIC";  maintain-aspect-ratio TRUE;  display "USEDEF";
%valid_file "F";  width 4.3474in;  height 0.4238in;  depth 0in;
%original-width 4.2964in;  original-height 0.3943in;  cropleft "0";
%croptop "1";  cropright "1";  cropbottom "0";
%filename 'Ratio2_k_2.png';file-properties "XNPEU";}} }%
%BeginExpansion
{\parbox[b]{4.3474in}{\begin{center}
\includegraphics[
natheight=0.394300in,
natwidth=4.296400in,
height=0.4238in,
width=4.3474in
]%
{C:/Users/Vladimir/Documents/BENFORD/AAA/FOR ProofReading INTERNET/From PRS 4  (ver 9)/Ratio2_k_2__6.png}%
\\
$k=2$ and \ $m=-2,-1,0,1,2$%
\end{center}}}
%EndExpansion%
%TCIMACRO{\FRAME{itbpFU}{4.3474in}{0.4238in}{0in}{\Qcb{$k=3$
%for\ $m=-2,-1,0,1,2$}}{}{ratio2_k_3.png}%
%{\special{ language "Scientific Word";  type "GRAPHIC";
%maintain-aspect-ratio TRUE;  display "USEDEF";  valid_file "F";
%width 4.3474in;  height 0.4238in;  depth 0in;  original-width 4.2964in;
%original-height 0.3943in;  cropleft "0";  croptop "1";  cropright "1";
%cropbottom "0";  filename 'Ratio2_k_3.png';file-properties "XNPEU";}} }%
%BeginExpansion
{\parbox[b]{4.3474in}{\begin{center}
\includegraphics[
natheight=0.394300in,
natwidth=4.296400in,
height=0.4238in,
width=4.3474in
]%
{C:/Users/Vladimir/Documents/BENFORD/AAA/FOR ProofReading INTERNET/From PRS 4  (ver 9)/Ratio2_k_3__7.png}%
\\
$k=3$ for\ $m=-2,-1,0,1,2$%
\end{center}}}
%EndExpansion
\caption{The Ratio of two Positive Numbers}\label{figureKey5}%
%TCIMACRO{\TeXButton{E}{\end{figure}}}%
%BeginExpansion
\end{figure}%
%EndExpansion




The ratio of the oscillatory part to $\log(1+1/k)$\ in \bigskip$\rho(k)$ has
the form%

%TCIMACRO{\TeXButton{B}{\begin{figure}[H] \centering}}%
%BeginExpansion
\begin{figure}[H] \centering
%EndExpansion%
%TCIMACRO{\FRAME{itbpFU}{5.5798in}{4.0421in}{0in}{\Qcb{$\frac{\rho(k)}%
%{\log(1+1/k)}-1$ vs s=$k$}}{}{at1.png}{\special{ language "Scientific Word";
%type "GRAPHIC";  maintain-aspect-ratio TRUE;  display "USEDEF";
%valid_file "F";  width 5.5798in;  height 4.0421in;  depth 0in;
%original-width 5.393in;  original-height 3.9003in;  cropleft "0";
%croptop "1";  cropright "1";  cropbottom "0";
%filename 'AT1.png';file-properties "XNPEU";}} }%
%BeginExpansion
{\parbox[b]{5.5798in}{\begin{center}
\includegraphics[
natheight=3.900300in,
natwidth=5.393000in,
height=4.0421in,
width=5.5798in
]%
{C:/Users/Vladimir/Documents/BENFORD/AAA/FOR ProofReading INTERNET/From PRS 4  (ver 9)/AT1__8.png}%
\\
$\frac{\rho(k)}{\log(1+1/k)}-1$ vs s=$k$%
\end{center}}}
%EndExpansion
\caption{The ratio of the oscillatory to non-oscilattory  parts}\label{figureKey8}%
%TCIMACRO{\TeXButton{E}{\end{figure}}}%
%BeginExpansion
\end{figure}%
%EndExpansion


Then we have%
\begin{align}
\rho(k,N)  &  =\log(k+1)-\log(k)+\nonumber\\
&
%TCIMACRO{\dsum \limits_{n=1}^{N}}%
%BeginExpansion
{\displaystyle\sum\limits_{n=1}^{N}}
%EndExpansion
A_{n}\left(  \sin(2\pi n\log(k+1))-\sin(2\pi n\log(k))\right)  ,\label{2NumbR}%
\\
A_{n}  &  =\left(  \pi n\cosh\left(  \frac{n\pi^{2}}{\ln(10)}\right)  \right)
^{-1}.\nonumber
\end{align}


If we calculate the difference from the NBL ($k=1$), we get%
\[
100\%\left(  \frac{0{\small .3092}}{0.3010}-1\right)  =2.75\%.
\]
The full solution is $\rho(k,\infty).$ The one-term approximation of this
solution has only a small deviation from the exact solution
\begin{align*}
\rho(k,1)  &  =\log(k^{-1}+1)+\left(  \pi\cosh\left(  \frac{\pi^{2}}{\ln
(10)}\right)  \right)  ^{-1}\left(  \sin(2\pi\log(k+1))-\sin(2\pi
\log(k))\right)  ,\\
\frac{\rho(k,1)}{\rho(k,\infty)}-1  &  =O\left(  10^{-4}\right)  .
\end{align*}


Now we consider the ratio of two random variables $z=\frac{y}{x}$\ selected
from normal distributions.

\begin{itemize}
\item \textbf{Normal Distributions} ($0$%
%TCIMACRO{\TEXTsymbol{<}}%
%BeginExpansion
$<$%
%EndExpansion
$\sigma_{x},0<\sigma_{y},\mu_{x}=0,\mu_{y}=0$) with PDF%
\begin{align*}
f(z)  &  =%
%TCIMACRO{\dint \limits_{x=0}^{\infty}}%
%BeginExpansion
{\displaystyle\int\limits_{x=0}^{\infty}}
%EndExpansion
xf_{norm}(xz,0,\sigma_{y})f_{norm}(x,0,\sigma_{x})dx\\
&  =\frac{\sigma_{x}\sigma_{y}}{2\pi\left(  \sigma_{x}^{2}z^{2}+\sigma_{y}%
^{2}\right)  },\\
\digamma &  =%
%TCIMACRO{\dint \limits_{z=0}^{\infty}}%
%BeginExpansion
{\displaystyle\int\limits_{z=0}^{\infty}}
%EndExpansion
f(z)dz=\frac{1}{4}.
\end{align*}
We have $1/4$ because we consider only the first quadrant ($0<x,0<y$).
\end{itemize}

Then%
\begin{align*}
\rho(k,\sigma_{x},\sigma_{y})  &  =\digamma\left(
\begin{array}
[c]{c}%
\log(1+k)-\log(k)+\\%
%TCIMACRO{\dsum \limits_{n=1}^{\infty}}%
%BeginExpansion
{\displaystyle\sum\limits_{n=1}^{\infty}}
%EndExpansion
A_{n}\left(  \sin(ns_{1}\right)  -\sin(ns))
\end{array}
\right)  ,\\
A_{n}  &  =\left(  \pi n\cosh\left(  \frac{n\pi^{2}}{\ln(10)}\right)  \right)
^{-1},\\
s_{1}  &  =2\pi\log\left(  \frac{\sigma_{x}(k+1)}{\sigma_{y}}\right)
,s=2\pi\log\left(  \frac{\sigma_{x}k}{\sigma_{y}}\right)  .
\end{align*}%
\[
A_{1}=8.7556\times10^{-3},A_{2}=6.0230\times10^{-5}.
\]
From this we obtain a DSLD close to the NBL, multiplied by $\digamma$.

Now we consider the ratio of two random variables $z=\frac{y}{x}$\ selected
from exponential distributions.

\begin{itemize}
\item \textbf{Exponential Distribution} ($0$%
%TCIMACRO{\TEXTsymbol{<}}%
%BeginExpansion
$<$%
%EndExpansion
$\lambda_{x},0<\lambda_{y}$)%
\begin{align*}
f_{x}(x)  &  =\frac{1}{\lambda_{x}}\exp\left(  -\frac{x}{\lambda_{x}}\right)
,f_{y}(y)=\frac{1}{\lambda_{y}}\exp\left(  -\frac{y}{\lambda_{y}}\right)  .\\
f(z)  &  =%
%TCIMACRO{\dint \limits_{x=0}^{\infty}}%
%BeginExpansion
{\displaystyle\int\limits_{x=0}^{\infty}}
%EndExpansion
xf_{y}(xz)f_{x}(x)dx\\
&  =\frac{\lambda_{x}\lambda_{y}}{\left(  \lambda_{x}z+\lambda_{y}\right)
^{2}},\\
\digamma &  =%
%TCIMACRO{\dint \limits_{z=0}^{\infty}}%
%BeginExpansion
{\displaystyle\int\limits_{z=0}^{\infty}}
%EndExpansion
f(z)dz=1.
\end{align*}
Then%
\begin{align*}
\rho(k,\lambda_{x},\lambda_{y})  &  =\left(
\begin{array}
[c]{c}%
\log(1+k)-\log(k)+\\%
%TCIMACRO{\dsum \limits_{n=1}^{\infty}}%
%BeginExpansion
{\displaystyle\sum\limits_{n=1}^{\infty}}
%EndExpansion
A_{n}\left(  \sin(ns_{1}\right)  -\sin(ns))
\end{array}
\right)  ,\\
A_{n}  &  =\frac{2\pi}{\ln(10)}\sinh\left(  \frac{n^{2}\pi}{\ln(10)}\right)
^{-1},\\
s_{1}  &  =2\pi\log\left(  \frac{\lambda_{x}(k+1)}{\lambda_{y}}\right)
,s=2\pi\log\left(  \frac{\lambda_{x}k}{\lambda_{y}}\right)  .
\end{align*}%
\[
A_{1}=1.0326\times10^{-3},A_{2}=1.9540\times10^{-7}.
\]
From this we obtain a DSLD close to the NBL.

Now we consider the ratio of two random variables $z=\frac{y}{x}$\ selected
from LogNormal distributions.

\item \textbf{Ratio } of two positive numbers with \textbf{LogNormal
distributions}%
\begin{align*}
z  &  =\frac{y}{x},\\
f(z)  &  =%
%TCIMACRO{\dint \limits_{x=0}^{\infty}}%
%BeginExpansion
{\displaystyle\int\limits_{x=0}^{\infty}}
%EndExpansion
xf_{y}(xz,\mu_{y},\sigma_{y})f_{x}(x,\mu_{x},\sigma_{x})dx\\
&  =f(z,\mu,\sigma),\\
\mu &  =\mu_{x}-\mu_{y},\sigma^{2}=\sigma_{x}^{2}+\sigma_{y}^{2},\\
\digamma &  =%
%TCIMACRO{\dint \limits_{z=0}^{\infty}}%
%BeginExpansion
{\displaystyle\int\limits_{z=0}^{\infty}}
%EndExpansion
f(z)dz=1.
\end{align*}
and the PDF for $z$ has a LogNormal form, see (\ref{LogNorm1}).
\end{itemize}

\subsection{DSLD for Normal Distribution}%

%TCIMACRO{\TeXButton{NormD}{\label{NormD}}}%
%BeginExpansion
\label{NormD}%
%EndExpansion
Because the normal distribution is common in many applications, let us
consider it in more detail.
\begin{align*}
f(z,\mu,\sigma)  &  =\frac{1}{\sigma\sqrt{2\pi}}\exp\left(  -\frac{(z-\mu
)^{2}}{2\sigma^{2}}\right)  ,\\
F(z,\mu,\sigma)  &  =\frac{1}{2}+\frac{1}{2}\operatorname{erf}\left(
\frac{z-\mu}{\sigma\sqrt{2}}\right)  ,\\
\digamma &  =%
%TCIMACRO{\dint \limits_{z=0}^{\infty}}%
%BeginExpansion
{\displaystyle\int\limits_{z=0}^{\infty}}
%EndExpansion
f(z)dz=1-F(0,\mu,\sigma)\\
&  =\frac{1}{2}+\frac{1}{2}\operatorname{erf}\left(  \frac{\mu}{\sigma\sqrt
{2}}\right)  .
\end{align*}
If $\mu\neq0$, we have $\left(  \xi=\frac{\mu}{\sigma\sqrt{2}}\right)  $
\begin{align*}
G(s,\mu,\sigma)  &  =C\left(  \xi\right)  s+Q_{1}\left(  s,\xi\right)  ,\\
C\left(  \xi\right)   &  =\frac{1}{2}+\frac{1}{2}\operatorname{erf}\left(
\xi\right)  ,\\
Q_{1}\left(  s,\xi\right)   &  =Q_{1}\left(  s+1,\xi\right)  ,\\
s  &  =\log(k)-\log(\sigma\sqrt{2}),s_{1}=\log(k+1)-\log(\sigma\sqrt{2}),\\
\rho(k,\mu,\sigma)  &  =A\left(  \xi\right)  \left(  \log(k+1)-\log(k)\right)
+Q_{1}\left(  s_{1},\xi\right)  -Q_{1}\left(  s,\xi\right)  .
\end{align*}
The function $\rho(k,\mu,\sigma)$ is invariant under the transformations
$\mu\rightarrow10^{n}\mu,\sigma\rightarrow10^{n}\sigma$ with integer $n$.

When $\mu=0$ see \ref{Nomal0}. In the general case, we will use the convenient
form (\ref{EM5})%

\begin{equation}
\rho(k)=%
%TCIMACRO{\dint \limits_{z=0}^{\infty}}%
%BeginExpansion
{\displaystyle\int\limits_{z=0}^{\infty}}
%EndExpansion
f(z)\left(  \left\lfloor \log\left(  \frac{z}{k}\right)  \right\rfloor
-\left\lfloor \log\left(  \frac{z}{k+1}\right)  \right\rfloor \right)  dz.
\label{Nomal1}%
\end{equation}


In some cases we can find asymptomatic formulas.

\begin{itemize}
\item If $0<\mu,\mu=O(1),\frac{\sigma}{\mu}$\bigskip$\ll1,\digamma=1$

It is convenient to make the change%
\begin{align*}
z  &  =\mu+\sigma\sqrt{2}t,\\
\rho(k)  &  =\frac{1}{\sqrt{\pi}}%
%TCIMACRO{\dint \limits_{t=-\frac{\mu}{\sigma\sqrt{2}}}^{\infty}}%
%BeginExpansion
{\displaystyle\int\limits_{t=-\frac{\mu}{\sigma\sqrt{2}}}^{\infty}}
%EndExpansion
\exp\left(  -t^{2}\right)  \left(  \left\lfloor \log\left(  \frac{\mu
+\sigma\sqrt{2}t}{k}\right)  \right\rfloor -\left\lfloor \log\left(  \frac
{\mu+\sigma\sqrt{2}t}{k+1}\right)  \right\rfloor \right)  dt.
\end{align*}

\end{itemize}

As $\frac{\sigma}{\mu}\rightarrow0$, we have%
\begin{equation}
\rho(k)=\left\{
\begin{array}
[c]{cc}%
\frac{1}{2} & \log\left(  \frac{\mu}{k}\right)  -\left\lfloor \log\left(
\frac{\mu}{k}\right)  \right\rfloor =0,\\
\frac{1}{2} & \log\left(  \frac{\mu}{k+1}\right)  -\left\lfloor \log\left(
\frac{\mu}{k+1}\right)  \right\rfloor =0\\
1 & \left\lfloor \log\left(  \frac{\mu}{k}\right)  \right\rfloor -\left\lfloor
\log\left(  \frac{\mu}{k+1}\right)  \right\rfloor >0\\
0 & \text{other}.
\end{array}
\right.  \label{NormD1}%
\end{equation}%
\begin{align*}
\Omega(s)  &  =\left\{
\begin{array}
[c]{cc}%
-\frac{1}{2}, & s=0,\\
\frac{1}{2}, & s=1,\\
1, & 1<s,\\
0, & \text{other}.
\end{array}
\right.  ,\\
\Omega(1)-\Omega(0)  &  =1.
\end{align*}%
\begin{align*}
\rho(k)  &  =\Omega(s_{1})-\Omega(s),\\
s  &  =\log\left(  \frac{k}{\mu}\right)  -\left\lfloor \log\left(  \frac
{k}{\mu}\right)  \right\rfloor ,\\
s_{1}  &  =\log\left(  \frac{k+1}{\mu}\right)  -\left\lfloor \log\left(
\frac{k}{\mu}\right)  \right\rfloor .
\end{align*}
\newline

\begin{itemize}
\item In the general case, the formulas for $A_{n}$ and $B_{n}$ too bulky and
it is better to use a numerical method for calculations.
\end{itemize}

Let find the distribution of the leading digits for the deviation from $\mu$,
$z=x-\mu$%
\[
\digamma=\frac{1}{2}+\frac{1}{2}\operatorname{erf}\left(  \frac{\mu\sqrt{2}%
}{2\sigma}\right)  .
\]%
\begin{align*}
A_{n}-\imath B_{n} &  =\frac{1}{n\pi^{\frac{3}{2}}}%
%TCIMACRO{\dint \limits_{z=0}^{\infty}}%
%BeginExpansion
{\displaystyle\int\limits_{z=0}^{\infty}}
%EndExpansion
x^{iY}\exp\left(  -(x-\xi)^{2}\right)  dx,\\
Y &  =\frac{2n\pi}{\ln(10)}.
\end{align*}
As an example, we consider the set of parameters as in
%TCIMACRO{\TeXButton{cite1}{\cite{formann0}} }%
%BeginExpansion
\cite{formann0}
%EndExpansion
$\mu=1.1,\sigma=0.25$ then%
\[
\digamma=\frac{1}{2}+\frac{1}{2}\operatorname{erf}\left(  3.1113\right)
=.99999.
\]%
%TCIMACRO{\TeXButton{B}{\begin{table}[!htbp] \centering}}%
%BeginExpansion
\begin{table}[!htbp] \centering
%EndExpansion%
\begin{tabular}
[c]{|c|c|c|c|c|c|c|c|c|c|}\hline\hline
${\small k}$ & ${\small 1}$ & ${\small 2}$ & ${\small 3}$ & ${\small 4}$ &
${\small 5}$ & ${\small 6}$ & ${\small 7}$ & ${\small 8}$ & ${\small 9}%
$\\\hline
${\small \rho(n=\infty)}$ & ${\small .6550}$ & ${\small .0010}$ &
${\small .0200}$ & ${\small .0050}$ & ${\small .01500}$ & ${\small .0320}$ &
${\small .0600}$ & ${\small .0970}$ & ${\small .1330}$\\\hline
${\small \rho(n=4)}$ & {\small .}$6540$ & ${\small .0021}$ & ${\small .0240}$
& ${\small .0046}$ & ${\small .01372}$ & ${\small .0347}$ & ${\small .0583}$ &
${\small .0952}$ & ${\small .1348}$\\\hline\hline
\end{tabular}
\caption{Table Caption 8}\label{TableKey8}%
%TCIMACRO{\TeXButton{E}{\end{table}}}%
%BeginExpansion
\end{table}%
%EndExpansion


\bigskip

The structure of the DSLD depends on the parameters of the underlying
distribution, $\mu=1.1$ and $\sigma=0.25$ in this case, which is equivalent to
$\mu=11$ and $\sigma=2.5$, with sample numbers mainly lying in the interval
$\mu\pm3\sigma$ $[3.5,18.5]$ For $\mu=100,\sigma=15$ (which is equivalent to
$\mu=10$, $\sigma=1.5,[5.5,14.5]$), we have

\bigskip%

%TCIMACRO{\TeXButton{B}{\begin{table}[!htbp] \centering}}%
%BeginExpansion
\begin{table}[!htbp] \centering
%EndExpansion%
\begin{tabular}
[c]{|c|c|c|c|c|c|c|c|c|c|}\hline\hline
${\small k}$ & ${\small 1}$ & ${\small 2}$ & ${\small 3}$ & ${\small 4}$ &
${\small 5}$ & ${\small 6}$ & ${\small 7}$ & ${\small 8}$ & ${\small 9}%
$\\\hline
${\small \rho(n=\infty)}$ & ${\small .5000}$ & ${\small .0000}$ &
${\small .0000}$ & ${\small .0000}$ & ${\small .00380}$ & ${\small .01900}$ &
${\small .0680}$ & ${\small .1620}$ & ${\small .2480}$\\\hline
${\small \rho(n=6)}$ & ${\small .4980}$ & ${\small -0.0009}$ & ${\small .0007}%
$ & ${\small .00183}$ & ${\small .00082}$ & ${\small .02162}$ &
${\small .0663}$ & ${\small .1616}$ & ${\small .2497}$\\\hline\hline
\end{tabular}
\caption{Table Caption 9}\label{TableKey9}%
%TCIMACRO{\TeXButton{E}{\end{table}}}%
%BeginExpansion
\end{table}%
%EndExpansion


\bigskip

Although there is a logarithmic term in the representation of the DSLD, the
distribution is not monotonous or tending to the NBL.

From (\ref{NormD1}) we can see that $\rho(k)$ is a $1$-periodic function of
$\log(k)$ (for $k$ positive) and $\log(\sigma)$%
\begin{align}
\rho(k)  &  =\frac{1}{2}\left(  \log(k+1)-\log(k)\right) \label{NormD2}\\
&  +Q_{1}(\log(k+1)-\lfloor\log(k)\rfloor+\lfloor\log(\sigma)\rfloor
-\log(\sigma))\nonumber\\
&  -Q_{1}(\log(k)-\lfloor\log(k)\rfloor+\lfloor\log(\sigma)\rfloor-\log
(\sigma)),\nonumber\\
k  &  =1,2,\ldots.\nonumber
\end{align}
%

%TCIMACRO{\TeXButton{B}{\begin{table}[!htbp] \centering}}%
%BeginExpansion
\begin{table}[!htbp] \centering
%EndExpansion%
\begin{tabular}
[c]{|c|c|c|c|c|c|c|c|c|c|}\hline\hline
${\small k}$ & ${\small 1}$ & ${\small 2}$ & ${\small 3}$ & ${\small 4}$ &
${\small 5}$ & ${\small 6}$ & ${\small 7}$ & ${\small 8}$ & ${\small 9}%
$\\\hline
${\small \rho(\sigma=1)}$ & ${\small .17987}$ & ${\small .06434}$ &
${\small .04330}$ & ${\small .04050}$ & ${\small .03860}$ & ${\small .03668}$
& ${\small .03446}$ & ${\small .03227}$ & ${\small .02980}$\\\hline\hline
\end{tabular}
\caption{Table Caption 10}\label{TableKey10}%
%TCIMACRO{\TeXButton{E}{\end{table}}}%
%BeginExpansion
\end{table}%
%EndExpansion


\bigskip

\noindent and%

%TCIMACRO{\TeXButton{B}{\begin{table}[!htbp] \centering}}%
%BeginExpansion
\begin{table}[!htbp] \centering
%EndExpansion%
\begin{tabular}
[c]{|c|c|c|c|c|c|c|c|c|c|}\hline\hline
${\small \sigma}$ & ${\small 1}$ & ${\small 2}$ & ${\small 3}$ & ${\small 4}$
& ${\small 5}$ & ${\small 6}$ & ${\small 7}$ & ${\small 8}$ & ${\small 9}%
$\\\hline
${\small \rho(k=1)}$ & ${\small .17987}$ & ${\small .17200}$ &
${\small .13213}$ & ${\small .11000}$ & ${\small .10770}$ & ${\small .11911}$
& ${\small .13640}$ & ${\small .15400}$ & ${\small .16874}$\\\hline\hline
\end{tabular}
\caption{Table Caption 11}\label{TableKey11}%
%TCIMACRO{\TeXButton{E}{\end{table}}}%
%BeginExpansion
\end{table}%
%EndExpansion
\bigskip%
\begin{align*}
G(s) &  =\frac{s}{2}+0.026418\sin(2\pi s)-0.000938\cos(2\pi s)\\
&  -0.000536\sin(4\pi s)-0.001459\cos(4\pi s).
\end{align*}
%

\begin{align*}
\rho(k) &  \approx G(\log(k+1)-\lfloor\log(k)\rfloor+\lfloor\log
(\sigma)\rfloor-\log(\sigma))\\
&  -G(\log(k)-\lfloor\log(k)\rfloor+\lfloor\log(\sigma)\rfloor-\log
(\sigma)),\\
k &  =1,2,\ldots.
\end{align*}


\bigskip Here are some illustrations of the ratio $\rho(k,\sigma,\mu
)/\log(1+1/k)$%

%TCIMACRO{\TeXButton{B}{\begin{figure}[H] \centering}}%
%BeginExpansion
\begin{figure}[H] \centering
%EndExpansion%
%TCIMACRO{\FRAME{itbpFU}{5.4509in}{3.9487in}{0in}{\Qcb{$\rho(k=1,\sigma
%,\mu=0)/\log(2)-1$ vs $\ s=\sigma.$}}{}{at1.png}%
%{\special{ language "Scientific Word";  type "GRAPHIC";
%maintain-aspect-ratio TRUE;  display "USEDEF";  valid_file "F";
%width 5.4509in;  height 3.9487in;  depth 0in;  original-width 5.393in;
%original-height 3.9003in;  cropleft "0";  croptop "1";  cropright "1";
%cropbottom "0";  filename 'AT1.png';file-properties "XNPEU";}} }%
%BeginExpansion
{\parbox[b]{5.4509in}{\begin{center}
\includegraphics[
natheight=3.900300in,
natwidth=5.393000in,
height=3.9487in,
width=5.4509in
]%
{C:/Users/Vladimir/Documents/BENFORD/AAA/FOR ProofReading INTERNET/From PRS 4  (ver 9)/AT1__9.png}%
\\
$\rho(k=1,\sigma,\mu=0)/\log(2)-1$ vs $\ s=\sigma.$%
\end{center}}}
%EndExpansion
\caption{Normal Distribution, $k=1$ and $\mu=0$}\label{figureKey9}%
%TCIMACRO{\TeXButton{E}{\end{figure}}}%
%BeginExpansion
\end{figure}%
%EndExpansion


\bigskip

\noindent and $(\sigma=1,\mu=1)$ \bigskip%

%TCIMACRO{\TeXButton{B}{\begin{figure}[H] \centering}}%
%BeginExpansion
\begin{figure}[H] \centering
%EndExpansion%
%TCIMACRO{\FRAME{itbpFU}{5.4509in}{3.9487in}{0in}{\Qcb{$\rho(k,\sigma
%=1,\mu=1)/(\digamma\log(1+1/k))-1$ vs $\ k$}}{}{normm1s1.png}%
%{\special{ language "Scientific Word";  type "GRAPHIC";
%maintain-aspect-ratio TRUE;  display "USEDEF";  valid_file "F";
%width 5.4509in;  height 3.9487in;  depth 0in;  original-width 5.393in;
%original-height 3.9003in;  cropleft "0";  croptop "1";  cropright "1";
%cropbottom "0";  filename 'NormM1S1.png';file-properties "XNPEU";}} }%
%BeginExpansion
{\parbox[b]{5.4509in}{\begin{center}
\includegraphics[
natheight=3.900300in,
natwidth=5.393000in,
height=3.9487in,
width=5.4509in
]%
{C:/Users/Vladimir/Documents/BENFORD/AAA/FOR ProofReading INTERNET/From PRS 4  (ver 9)/NormM1S1__10.png}%
\\
$\rho(k,\sigma=1,\mu=1)/(\digamma\log(1+1/k))-1$ vs $\ k$%
\end{center}}}
%EndExpansion
\caption{Normal Distribution, $\sigma=1$ and $\mu=0$}\label{figureKey10}%
%TCIMACRO{\TeXButton{E}{\end{figure}}}%
%BeginExpansion
\end{figure}%
%EndExpansion


\noindent and $(\sigma=0.1,\mu=1)$

\bigskip%

%TCIMACRO{\TeXButton{B}{\begin{figure}[H] \centering}}%
%BeginExpansion
\begin{figure}[H] \centering
%EndExpansion%
%TCIMACRO{\FRAME{itbpFU}{5.4509in}{3.9487in}{0in}{\Qcb{$\rho(k,\sigma
%=0.1,\mu=1)/(\digamma\log(1+1/k))-1$ vs $\ k$}}{}{normm1s01.png}%
%{\special{ language "Scientific Word";  type "GRAPHIC";
%maintain-aspect-ratio TRUE;  display "USEDEF";  valid_file "F";
%width 5.4509in;  height 3.9487in;  depth 0in;  original-width 5.393in;
%original-height 3.9003in;  cropleft "0";  croptop "1";  cropright "1";
%cropbottom "0";  filename 'NormM1S01.png';file-properties "XNPEU";}} }%
%BeginExpansion
{\parbox[b]{5.4509in}{\begin{center}
\includegraphics[
natheight=3.900300in,
natwidth=5.393000in,
height=3.9487in,
width=5.4509in
]%
{C:/Users/Vladimir/Documents/BENFORD/AAA/FOR ProofReading INTERNET/From PRS 4  (ver 9)/NormM1S01__11.png}%
\\
$\rho(k,\sigma=0.1,\mu=1)/(\digamma\log(1+1/k))-1$ vs $\ k$%
\end{center}}}
%EndExpansion
\caption{Normal Distribution, $\sigma=0.1$ and $\mu=1$}\label{figureKey11}%
%TCIMACRO{\TeXButton{E}{\end{figure}}}%
%BeginExpansion
\end{figure}%
%EndExpansion


\bigskip

\noindent in this case, the contribution of the oscillating term is up to $8$
times greater than that of the logarithmic.%

%TCIMACRO{\TeXButton{B}{\begin{figure}[H] \centering}}%
%BeginExpansion
\begin{figure}[H] \centering
%EndExpansion%
%TCIMACRO{\FRAME{itbpFU}{5.4509in}{3.9487in}{0in}{\Qcb{$\rho(k,\sigma
%=1,\mu=-1)/(\digamma\log(1+1/k))-1$ vs $\ k$}}{}{normm-1s1.png}%
%{\special{ language "Scientific Word";  type "GRAPHIC";
%maintain-aspect-ratio TRUE;  display "USEDEF";  valid_file "F";
%width 5.4509in;  height 3.9487in;  depth 0in;  original-width 5.393in;
%original-height 3.9003in;  cropleft "0";  croptop "1";  cropright "1";
%cropbottom "0";  filename 'NormM-1S1.png';file-properties "XNPEU";}} }%
%BeginExpansion
{\parbox[b]{5.4509in}{\begin{center}
\includegraphics[
natheight=3.900300in,
natwidth=5.393000in,
height=3.9487in,
width=5.4509in
]%
{C:/Users/Vladimir/Documents/BENFORD/AAA/FOR ProofReading INTERNET/From PRS 4  (ver 9)/NormM-1S1__12.png}%
\\
$\rho(k,\sigma=1,\mu=-1)/(\digamma\log(1+1/k))-1$ vs $\ k$%
\end{center}}}
%EndExpansion
\caption{Normal Distribution, $\sigma=1$ and $\mu=-1$}\label{figureKey12}%
%TCIMACRO{\TeXButton{E}{\end{figure}}}%
%BeginExpansion
\end{figure}%
%EndExpansion


\bigskip

An easy kind of approximation has the form%
\begin{align*}
\rho(k,\sigma) &  \approx\frac{1}{2}\log\left(  \frac{k+1}{k}\right)  \\
&  +\frac{1}{12\pi}\left(  \sin\left(  2\pi(\log(k+1)-\log(\sigma))\right)
-\sin\left(  2\pi(\log(k)-\log(\sigma))\right)  \right)  ,\\
k &  =1,2,\ldots.
\end{align*}


For negative numbers $x$, we have%
\[
10^{m}(k-1)\leq x\leq10^{m}k
\]%
\begin{align}
\rho(k)  &  =\frac{1}{2}\left(  \log(\left\vert k\right\vert +1)-\log
((\left\vert k\right\vert )\right)  +Q_{1}(\log(\left\vert k\right\vert
+1)-\lfloor\log((\left\vert k\right\vert )\rfloor+\lfloor\log(\sigma
)\rfloor-\log(\sigma))\\
&  -Q_{1}(\log((\left\vert k\right\vert )-\lfloor\log((\left\vert k\right\vert
)\rfloor+\lfloor\log(\sigma)\rfloor-\log(\sigma)),\nonumber\\
k  &  =-1,-2,\ldots.\nonumber
\end{align}


It is important to emphasize that in spite of the possible argument about its
insignificant value, the role of the functions $Q_{1}$ can be substantial. So
if we look at the dependence on $\sigma$, the difference can reach more than
$50\%$.%
\begin{align*}
\rho(k  &  =1,\sigma=1.327,\mu=0)=.19287,\\
\rho(k  &  =1,\sigma=4.613,\mu=0)=.10657,\\
\frac{\rho(1,1.327)+\rho(1,4.613)}{2}  &  =.14972,\\
100\%\frac{2(\rho(1,1.327)-\rho(1,4.613))}{\rho(1,1.327)+\rho(1,4.613)}  &
=57.6\%.
\end{align*}
When $\mu\neq0$, we have%
\begin{align*}
G(s,\mu,\sigma)  &  =A\left(  \frac{\mu}{\sigma}\right)  s+Q_{1}\left(
s,\frac{\mu}{\sigma}\right)  ,\\
A\left(  \frac{\mu}{\sigma}\right)   &  =\frac{1}{2}+\frac{1}{2}%
\operatorname{erf}\left(  \frac{\mu\sqrt{2}}{2\sigma}\right)  ,\\
Q_{1}\left(  s,\frac{\mu}{\sigma}\right)   &  =Q_{1}\left(  s+1,\frac{\mu
}{\sigma}\right)  ,\\
s  &  =\log(k)-\log(\sigma),s_{1}=\log(k+1)-\log(\sigma),\\
\rho(k,\sigma,\mu)  &  =A\left(  \frac{\mu}{\sigma}\right)  \left(
\log(k+1)-\log(k)\right)  +Q_{1}\left(  s_{1},\frac{\mu}{\sigma}\right)
-Q_{1}\left(  s,\frac{\mu}{\sigma}\right)  .
\end{align*}


\subsection{Truncated Normal Distribution}

The truncated normal distribution is often used in different application.
\begin{align*}
f(z,\sigma,\mu,a,b)  &  =\left\{
\begin{array}
[c]{cc}%
C\exp\left(  -\frac{(z-\mu)^{2}}{2\sigma^{2}}\right)  , & a\leq z\leq b\\
0, & z<a,b<z
\end{array}
\right.  ,\\
C  &  =\frac{1}{\sigma}\sqrt{\frac{2}{\pi}}\left(  \operatorname{erf}\left(
\frac{b-\mu}{\sigma\sqrt{2}}\right)  -\operatorname{erf}\left(  \frac{a-\mu
}{\sigma\sqrt{2}}\right)  \right)  ^{-1},-\infty\leq a<b\leq\infty.
\end{align*}
And%
\begin{align*}
CDF  &  =C\left(  \operatorname{erf}\left(  \frac{z-\mu}{\sigma\sqrt{2}%
}\right)  -\operatorname{erf}\left(  \frac{a-\mu}{\sigma\sqrt{2}}\right)
\right) \\
\digamma &  =C\left(  \operatorname{erf}\left(  \frac{\gamma_{+}-\mu}%
{\sigma\sqrt{2}}\right)  -\operatorname{erf}\left(  \frac{\gamma_{-}-\mu
}{\sigma\sqrt{2}}\right)  \right)  ,\\
\gamma_{+}  &  =\max(b,0),\gamma_{-}=\max(a,0).
\end{align*}
An expression for $\rho(z,\sigma,\mu,a,b)$ can be found from (\ref{EM0a}).

\section{Approximation}%

%TCIMACRO{\TeXButton{\label{Approximation}}{\label{Approximation}}}%
%BeginExpansion
\label{Approximation}%
%EndExpansion


\bigskip Other presentation os $s$ and $s_{1}$%
\[
s=\log(k)-\left\lfloor \log\left(  k\right)  \right\rfloor ,s_{1}%
=\log(k+1)-\left\lfloor \log\left(  k\right)  \right\rfloor ,
\]%
\[
s=\log(k)+\left\lfloor -\log\left(  k\right)  \right\rfloor ,s_{1}%
=\log(k+1)+\left\lfloor -\log\left(  k+1\right)  \right\rfloor .
\]
Because%
\[
\left\lfloor -x\right\rfloor =-\left\lfloor x\right\rfloor +\left\{
\begin{array}
[c]{cc}%
0, & x\in%
%TCIMACRO{\U{2124} }%
%BeginExpansion
\mathbb{Z}
%EndExpansion
,\\
-1, & x\notin%
%TCIMACRO{\U{2124} }%
%BeginExpansion
\mathbb{Z}
%EndExpansion
,
\end{array}
\right.
\]
Then%
\[
\left\lfloor -\log\left(  k+1\right)  \right\rfloor =-\left\lfloor \log\left(
k\right)  \right\rfloor -1,
\]
and%

\[
\left\lfloor -\log\left(  k+1\right)  \right\rfloor =\left\lfloor -\log\left(
k\right)  \right\rfloor .
\]


\begin{itemize}
\item Discrete $\Omega$%
\begin{align*}
-\Omega(\log(k),a)  &  =\left\lfloor \log\left(  \frac{a}{k}\right)
\right\rfloor -\log\left(  \frac{a}{k}\right)  +\log\left(  a\right)
-\log\left(  k\right)  ,\\
&  =-\left(  \log\left(  \frac{a}{k}\right)  -\left\lfloor \log\left(
\frac{a}{k}\right)  \right\rfloor \right)  +\log\left(  a\right)  -\log\left(
k\right)
\end{align*}%
\[
\Omega(s,a)=s+\frac{1}{N}%
%TCIMACRO{\dsum \limits_{i=1}^{N}}%
%BeginExpansion
{\displaystyle\sum\limits_{i=1}^{N}}
%EndExpansion
\left(  \log\left(  a_{i}\right)  -s-\left\lfloor \log\left(  a_{i}\right)
-s\right\rfloor \right)  -\frac{1}{N}%
%TCIMACRO{\dsum \limits_{i=1}^{N}}%
%BeginExpansion
{\displaystyle\sum\limits_{i=1}^{N}}
%EndExpansion
\log\left(  a_{i}\right)  .
\]%
\begin{align*}
\{x\}  &  =x-\left\lfloor x\right\rfloor \\
&  =\left\{
\begin{array}
[c]{cc}%
x, & x\in%
%TCIMACRO{\U{2124} }%
%BeginExpansion
\mathbb{Z}
%EndExpansion
,\\%
%TCIMACRO{\dsum \limits_{j=1}^{\infty}}%
%BeginExpansion
{\displaystyle\sum\limits_{j=1}^{\infty}}
%EndExpansion
\frac{1}{\pi j}\sin(2\pi jx), & x\notin%
%TCIMACRO{\U{2124} }%
%BeginExpansion
\mathbb{Z}
%EndExpansion
.
\end{array}
\right.  .
\end{align*}
We set the non-essential additive constant to zero.%
\[
\Omega(s,a)=s+\frac{1}{N}%
%TCIMACRO{\dsum \limits_{i=1}^{N}}%
%BeginExpansion
{\displaystyle\sum\limits_{i=1}^{N}}
%EndExpansion
\{\log\left(  a_{i}\right)  -s\}+const.
\]%
\begin{align*}
\frac{1}{N}%
%TCIMACRO{\dsum \limits_{i=1}^{N}}%
%BeginExpansion
{\displaystyle\sum\limits_{i=1}^{N}}
%EndExpansion
\{\log\left(  a_{i}\right)  -s\}  &  =%
%TCIMACRO{\dsum \limits_{j=1}^{\infty}}%
%BeginExpansion
{\displaystyle\sum\limits_{j=1}^{\infty}}
%EndExpansion
\Lambda_{j}\frac{\sin(2\pi js+\varphi_{j})}{\pi j},\\
\Lambda_{j}\exp(\imath\varphi_{j})  &  =\frac{1}{\pi jN}%
%TCIMACRO{\dsum \limits_{i=1}^{N}}%
%BeginExpansion
{\displaystyle\sum\limits_{i=1}^{N}}
%EndExpansion
\exp(-2\pi j\imath\left\{  \log\left(  a_{i}\right)  \right\}  ).
\end{align*}


\item Continues $\Omega$%
\begin{align*}
-\Omega(\log(k),a)  &  =%
%TCIMACRO{\dint \limits_{z=0}^{\infty}}%
%BeginExpansion
{\displaystyle\int\limits_{z=0}^{\infty}}
%EndExpansion
f(z)\left\lfloor \log\left(  \frac{z}{k}\right)  \right\rfloor dz\\
&  =%
%TCIMACRO{\dint \limits_{z=0}^{\infty}}%
%BeginExpansion
{\displaystyle\int\limits_{z=0}^{\infty}}
%EndExpansion
f(z)\left(  \left\lfloor \log\left(  \frac{z}{k}\right)  \right\rfloor
-\log\left(  \frac{z}{k}\right)  +\log\left(  z\right)  -\log\left(  k\right)
\right)  dz,\\
&  =-%
%TCIMACRO{\dint \limits_{z=0}^{\infty}}%
%BeginExpansion
{\displaystyle\int\limits_{z=0}^{\infty}}
%EndExpansion
f(z)\left\{  \log\left(  \frac{z}{k}\right)  \right\}  dz-\log\left(
k\right)
%TCIMACRO{\dint \limits_{z=0}^{\infty}}%
%BeginExpansion
{\displaystyle\int\limits_{z=0}^{\infty}}
%EndExpansion
f(z)dz.
\end{align*}%
\[
\Omega(s)=s%
%TCIMACRO{\dint \limits_{z=0}^{\infty}}%
%BeginExpansion
{\displaystyle\int\limits_{z=0}^{\infty}}
%EndExpansion
f(z)dz.+%
%TCIMACRO{\dint \limits_{z=0}^{\infty}}%
%BeginExpansion
{\displaystyle\int\limits_{z=0}^{\infty}}
%EndExpansion
f(z)\left(  \log\left(  z\right)  -s-\left\lfloor \log\left(  z\right)
-s\right\rfloor \right)  dz+const.
\]
We set the non-essential additive constant to zero.%
\[
\Omega(s)=s%
%TCIMACRO{\dint \limits_{z=0}^{\infty}}%
%BeginExpansion
{\displaystyle\int\limits_{z=0}^{\infty}}
%EndExpansion
f(z)dz.+%
%TCIMACRO{\dint \limits_{z=0}^{\infty}}%
%BeginExpansion
{\displaystyle\int\limits_{z=0}^{\infty}}
%EndExpansion
f(z)\left\{  \log\left(  z\right)  -s\right\}  dz.
\]

\end{itemize}

%

\begin{align*}%
%TCIMACRO{\dint \limits_{z=0}^{\infty}}%
%BeginExpansion
{\displaystyle\int\limits_{z=0}^{\infty}}
%EndExpansion
f(z)\left\{  \log\left(  z\right)  -s\right\}  dz  &  =%
%TCIMACRO{\dsum \limits_{j=1}^{\infty}}%
%BeginExpansion
{\displaystyle\sum\limits_{j=1}^{\infty}}
%EndExpansion
\Lambda_{j}\sin(2\pi js+\varphi_{j}),\\
\Lambda_{j}\exp(\imath\varphi_{j})  &  =\frac{1}{\pi j}%
%TCIMACRO{\dint \limits_{z=0}^{\infty}}%
%BeginExpansion
{\displaystyle\int\limits_{z=0}^{\infty}}
%EndExpansion
f(z)\exp(-2\pi j\imath\left\{  \log\left(  z\right)  \right\}  )dz\\
&  =\frac{1}{\pi j}%
%TCIMACRO{\dint \limits_{z=0}^{\infty}}%
%BeginExpansion
{\displaystyle\int\limits_{z=0}^{\infty}}
%EndExpansion
f(z)z^{\left(  -\frac{2\pi j\imath}{\log\left(  10\right)  }\right)
}dz,j=1,2,\ldots.
\end{align*}


\begin{itemize}
\item Connection with van der Corput's methods.

\item From (\ref{Ratio2})
\end{itemize}

%

\[
\Lambda_{n}\exp(\varphi_{n}\imath)=\frac{1}{2\pi n}\left(
%TCIMACRO{\dint \limits_{z=0}^{\ 1}}%
%BeginExpansion
{\displaystyle\int\limits_{z=0}^{\ 1}}
%EndExpansion
z^{(-Y\imath)}dz+%
%TCIMACRO{\dint \limits_{z=1}^{\ \infty}}%
%BeginExpansion
{\displaystyle\int\limits_{z=1}^{\ \infty}}
%EndExpansion
z^{(-2-Y\imath)}dz\right)  ,Y=\frac{2\pi n}{\ln(10)},n=1,2,\ldots
\]%
\[
\Lambda_{n}=\frac{1}{\pi n}\frac{1}{(1+Y^{2})},\varphi_{n}=0.
\]%
\begin{align}
&  \frac{10^{-L}}{18}-10^{L}\frac{5}{9}\left(  \frac{1}{k+1}\frac{1}%
{k}\right)  \label{Ratio3}\\
&  \approx\log\left(  1+\frac{1}{k}\right)  +0.0377\times\left(  \sin\left(
2\pi\log(1+k)\right)  -\sin\left(  2\pi\log(k)\right)  \right)  ,\nonumber\\
L &  =\left\lfloor \log(k)\right\rfloor .\nonumber
\end{align}%
\begin{align*}
k &  =1,0.3333\approx0.3368,\\
k &  =15,0.02867\approx0.03047.
\end{align*}


\begin{itemize}
\item From (\ref{Prod2})
\end{itemize}

%

\[
\Lambda_{n}\exp(\varphi_{n}\imath)=\frac{1}{\pi n}%
%TCIMACRO{\dint \limits_{z=0}^{1}}%
%BeginExpansion
{\displaystyle\int\limits_{z=0}^{1}}
%EndExpansion
-z^{(-Y\imath)}\ln(z)dz,Y=\frac{2\pi n}{\ln(10)},n=1,2,\ldots
\]%
\[
\Lambda_{n}=\frac{1}{\pi n}\frac{1}{(1+Y^{2})},\varphi_{n}=\arctan\left(
\frac{2Y}{1-Y^{2}}\right)  .
\]%
\begin{align}
&  \frac{10^{-L}}{9}\left(  k\ln(k)-(k+1)\ln(k+1)+1+\frac{(9L+10)}{9}%
\ln(10)\right)  \label{Prod3}\\
&  \approx\log\left(  1+\frac{1}{k}\right)  +0.0377\times\left(  \sin\left(
2\pi\log(1+k)+2.4391\right)  -\sin\left(  2\pi\log(k)+2.4391\right)  \right)
.\nonumber
\end{align}%
\begin{align*}
k &  =1,0.24135\approx0.24171,\\
k &  =15,0.02353\approx0.0222.
\end{align*}


\begin{itemize}
\item Estimation coefficients in $n!$

\item Approximation of function $P(k,x,n)$
\begin{align}
P(k,x,n)  &  =\left\{
\begin{array}
[c]{cc}%
s-s_{1}, & s\in%
%TCIMACRO{\U{2124} }%
%BeginExpansion
\mathbb{Z}
%EndExpansion
,s_{1}\in%
%TCIMACRO{\U{2124} }%
%BeginExpansion
\mathbb{Z}
%EndExpansion
,\\
s-s_{1}+\frac{1}{2}-%
%TCIMACRO{\dsum \limits_{j=1}^{n}}%
%BeginExpansion
{\displaystyle\sum\limits_{j=1}^{n}}
%EndExpansion
\frac{\sin(2\pi js_{1})}{\pi j}, & s\in%
%TCIMACRO{\U{2124} }%
%BeginExpansion
\mathbb{Z}
%EndExpansion
,s_{1}\notin%
%TCIMACRO{\U{2124} }%
%BeginExpansion
\mathbb{Z}
%EndExpansion
,\\
s-s_{1}-\frac{1}{2}+%
%TCIMACRO{\dsum \limits_{j=1}^{n}}%
%BeginExpansion
{\displaystyle\sum\limits_{j=1}^{n}}
%EndExpansion
\frac{\sin(2\pi js)}{\pi j}, & s\notin%
%TCIMACRO{\U{2124} }%
%BeginExpansion
\mathbb{Z}
%EndExpansion
,s_{1}\in%
%TCIMACRO{\U{2124} }%
%BeginExpansion
\mathbb{Z}
%EndExpansion
,\\
s-s_{1}+%
%TCIMACRO{\dsum \limits_{j=1}^{n}}%
%BeginExpansion
{\displaystyle\sum\limits_{j=1}^{n}}
%EndExpansion
\frac{\sin(2\pi js)-\sin(2\pi js_{1})}{\pi j}, & s\notin%
%TCIMACRO{\U{2124} }%
%BeginExpansion
\mathbb{Z}
%EndExpansion
,s_{1}\notin%
%TCIMACRO{\U{2124} }%
%BeginExpansion
\mathbb{Z}
%EndExpansion
.
\end{array}
\right\} \label{RHOa}\\
s  &  =\log\left(  \frac{x}{k}\right)  ,s_{1}=\log\left(  \frac{x}%
{k+1}\right)  .\nonumber
\end{align}


\item
\[
z=y+x,
\]
where $y\approx$ $N(\mu,\sigma)$ and $x\approx$ $U(0,\varepsilon)$ then%
\[
f(z,\mu,\sigma,\varepsilon)=\frac{1}{2\varepsilon}\left(  \operatorname{erf}%
\left(  \frac{z-\mu}{\sigma\sqrt{2}}\right)  -\operatorname{erf}\left(
\frac{z-\mu-\varepsilon}{\sigma\sqrt{2}}\right)  \right)  .
\]

\end{itemize}

The function $\rho(k)$ weakly depends on relatively small positive changes in
the values of the set. This robustness makes it possible to impose additional
small perturbations and to construct an approximate formula for DSLD.

\begin{itemize}
\item Vector of coefficiens $A_{i}$ and $B_{i}$ is $w$ witn $2n+1(0\leq n)$
components \ \
\[
w=\left[  A_{0},A_{1},B_{1},A_{2},B_{2},\ldots\right]  ^{\top}.
\]
and $G$ with $m(0\leq m)$ is vector of experemental values of some $\rho(k)$%
\[
G=\left[  G_{1},G_{2},G_{3},G_{4},G_{5},\ldots\right]  ^{\top}.
\]%
\begin{align*}
M_{i,1}  &  =\log\left(  1+\frac{1}{k}\right)  ,1\leq i\leq m,\\
M_{i,j}  &  =\sin\left(  2\pi j\log(1+k)\right)  -\sin\left(  2\pi
j\log(k)\right)  ,j\text{ even,}2\leq j,\\
M_{i,j}  &  =\cos\left(  2\pi j\log(1+k)\right)  -\cos\left(  2\pi
j\log(k)\right)  ,j\text{ odd,}3\leq j.
\end{align*}
Then estimation of vector $w$\ \ is%
\[
w=\left(  M^{\top}.M\right)  ^{-1}.M^{\top}.G.
\]

\end{itemize}

\subsection{Discrete Distributions}%

\begin{align*}
\rho(k) &  =\log\left(  1+\frac{1}{k}\right)  +%
%TCIMACRO{\dsum \limits_{j=1}^{m}}%
%BeginExpansion
{\displaystyle\sum\limits_{j=1}^{m}}
%EndExpansion
U(k,a,j),m\in%
%TCIMACRO{\U{2124} }%
%BeginExpansion
\mathbb{Z}
%EndExpansion
^{\ast},\\
U(k,a,j) &  =-\frac{1}{\pi jN}%
%TCIMACRO{\dsum \limits_{i=1}^{N}}%
%BeginExpansion
{\displaystyle\sum\limits_{i=1}^{N}}
%EndExpansion
\left(  \sin\left(  2\pi j\log\left(  \frac{a_{i}}{1+k}\right)  \right)
-\sin\left(  2\pi j\log\left(  \frac{a_{i}}{k}\right)  \right)  \right)  .
\end{align*}%
\begin{align*}
s &  =2\pi\log(k),s_{1}=2\pi\log(1+k),\\
S_{j}(N) &  =\frac{1}{\pi jN}%
%TCIMACRO{\dsum \limits_{i=1}^{N}}%
%BeginExpansion
{\displaystyle\sum\limits_{i=1}^{N}}
%EndExpansion
\left(  a_{i}\right)  ^{-\frac{2\pi j}{\ln(10)}\imath}\\
&  =\frac{1}{\pi jN}%
%TCIMACRO{\dsum \limits_{i=1}^{N}}%
%BeginExpansion
{\displaystyle\sum\limits_{i=1}^{N}}
%EndExpansion
\exp\left(  -2\pi j\imath\log(a_{i}\right)  )\\
&  =\frac{1}{\pi jN}%
%TCIMACRO{\dsum \limits_{i=1}^{N}}%
%BeginExpansion
{\displaystyle\sum\limits_{i=1}^{N}}
%EndExpansion
\exp\left(  -2\pi j\imath\left\{  \log\left(  a_{i}\right)  \right\}  \right)
.
\end{align*}%
\[
S_{j}(N)=R\exp(\imath\varphi),
\]
Different presentations of $U(k,a,j)$
\begin{align*}
U(k,a,j) &  =\operatorname{Re}(S_{j}(N))\left(  \sin(js_{1})-\sin(js)\right)
\\
&  +\operatorname{Im}(S_{j}(N))\left(  \cos(js_{1})-\cos(js)\right)  .
\end{align*}%
\[
U(k,a,j)=R\left(  \sin(js_{1}+\varphi)-\sin(js+\varphi)\right)  .
\]%
\begin{align*}
&
%TCIMACRO{\dsum \limits_{i}}%
%BeginExpansion
{\displaystyle\sum\limits_{i}}
%EndExpansion
\left(  \sin(A_{i}-s)-\sin(A_{i}-s_{1})\right)  \\
&  =%
%TCIMACRO{\dsum \limits_{i}}%
%BeginExpansion
{\displaystyle\sum\limits_{i}}
%EndExpansion
\left(  \cos(A_{i})(\sin(s_{1})-\sin(s))-\sin(A_{i})(\cos(s_{1})-\cos
(s))\right)
\end{align*}%
\begin{align*}
\left(
%TCIMACRO{\dsum \limits_{i}}%
%BeginExpansion
{\displaystyle\sum\limits_{i}}
%EndExpansion
\exp(-\imath A_{i})\right)   &  =R\exp(\imath\varphi),\\%
%TCIMACRO{\dsum \limits_{i}}%
%BeginExpansion
{\displaystyle\sum\limits_{i}}
%EndExpansion
\cos(A_{i}) &  =\operatorname{Re}\left(
%TCIMACRO{\dsum \limits_{i}}%
%BeginExpansion
{\displaystyle\sum\limits_{i}}
%EndExpansion
\exp(-\imath A_{i})\right)  ,\\
-%
%TCIMACRO{\dsum \limits_{i}}%
%BeginExpansion
{\displaystyle\sum\limits_{i}}
%EndExpansion
\sin(A_{i}) &  =\operatorname{Im}\left(
%TCIMACRO{\dsum \limits_{i}}%
%BeginExpansion
{\displaystyle\sum\limits_{i}}
%EndExpansion
\exp(-\imath A_{i})\right)  .
\end{align*}
%

\[
U(k,a,j)=\frac{R}{\pi jN}\left(  \sin\left(  2\pi j\log\left(  1+k\right)
+\varphi\right)  -\sin\left(  2\pi j\log\left(  k\right)  +\varphi\right)
\right)  .
\]


\subsection{The Fast Fourier Transform.}%

\begin{align*}%
%TCIMACRO{\dint \limits_{z=0}^{\infty}}%
%BeginExpansion
{\displaystyle\int\limits_{z=0}^{\infty}}
%EndExpansion
f(z)dz  &  =\ln(10)%
%TCIMACRO{\dint \limits_{t=-\infty}^{\infty}}%
%BeginExpansion
{\displaystyle\int\limits_{t=-\infty}^{\infty}}
%EndExpansion
g(t)dt,\\
\frac{1}{\pi n}%
%TCIMACRO{\dint \limits_{z=0}^{\infty}}%
%BeginExpansion
{\displaystyle\int\limits_{z=0}^{\infty}}
%EndExpansion
\exp(-2\pi n\log(z)\imath)f(z)dz  &  =\frac{1}{\pi n}%
%TCIMACRO{\dint \limits_{t=-\infty}^{\infty}}%
%BeginExpansion
{\displaystyle\int\limits_{t=-\infty}^{\infty}}
%EndExpansion
\exp(-2\pi nt\imath)g(t)dt,\\
g(t)  &  =\ln(10)10^{t}f(10^{t}).
\end{align*}
Forward Discrete Fourier Transform (DFT):%
\[
X_{n}=%
%TCIMACRO{\dsum \limits_{j=0}^{N-1}}%
%BeginExpansion
{\displaystyle\sum\limits_{j=0}^{N-1}}
%EndExpansion
\exp\left(  -2\pi n\frac{j}{N}\imath\right)  x_{j}.
\]
Inverse Discrete Fourier Transform (IDFT):%
\[
x_{n}=\frac{1}{N}%
%TCIMACRO{\dsum \limits_{j=0}^{N-1}}%
%BeginExpansion
{\displaystyle\sum\limits_{j=0}^{N-1}}
%EndExpansion
\exp\left(  2\pi n\frac{j}{N}\imath\right)  X_{j}.
\]


\section{Discussion}%

%TCIMACRO{\TeXButton{]label{Discussion}}{\label{Discussion}}}%
%BeginExpansion
\label{Discussion}%
%EndExpansion


\ \ \ \ \ This article develops an approach to the manifestation of a certain
type of DSLD. We have shown that the DSLD does not have an arbitrary form;
rather, it takes one of three forms. In some cases the DSLD can be expressed
in closed form or approximated with a high degree of accuracy. We propose an
integral representation of the DSLD convenient for analytical and numerical analysis.

The form of an DSLD depends not only on the type of underlying probability
distribution, but also on the parameters of this distribution. So when we
change the parameters, the form of the DSLD changes.

We stressed at the outset of our discussion the universality of the NBL, that
the introduction of this distribution attracted the attention of researchers
to this type of problem.

As we can see in Sections
%TCIMACRO{\TeXButton{\ref{P2NfUD} and \ref{Pro3}}{\ref{P2NfUD} and \ref{Pro3}}
%}%
%BeginExpansion
\ref{P2NfUD} and \ref{Pro3}
%EndExpansion
the values for the frequency distributions are close to the NBL, but as seen
from the explicit formulas, definitely not the NBL.

The literature on NBL has adopted a very strong and, at the same time, rather
restrictive principle, namely, to consider as insignificant deviations that
are not described in the law. This principle, strictly followed by many
researchers, has led to the impression of the universality of this law, but
there are other opinions
%TCIMACRO{\TeXButton{CITE}{\cite{arnold0},
%\cite{arnold1},
%\cite{bergerhill}}}%
%BeginExpansion
\cite{arnold0},
\cite{arnold1},
\cite{bergerhill}%
%EndExpansion
. The NBL\ has an empirical nature, and in many cases has no procedure for an
estimation of the error measurements of experimental data.

Many researchers have drawn conclusions on distributions based on the analysis
of a relatively small number of samples, without the evaluation of
approximation errors. It should also be noted here that DSLD\textbf{\ }%
necessarily depends on properties of both the underlying distribution and the
area interval.

These two factors constitute an essential and intrinsic part of the structure,
and neither can be neglected when constructing an DSLD. So they are, in this
sense, inseparable, and must always appear together. The contributions of the
log term and the oscillating term depend on the underlying probability
distribution. In some cases, the contribution of the oscillating term may be
relatively small, or it can be overwhelming. In many cases, the question
arises, what is the measure of the accuracy of the approximations. Sometimes
it is rather arbitrary, and may lead to the `universality' of the NBL. As we
have seen from the analysis of the normal distribution (with $0<\mu$), the
presence of a log term does not lead to a distribution resembling the NBL.

We will consider two simple distributions to illustrate the variation of the
DSLD depending on the parameters. The behavior of the DSLD is a function of
the properties of the underlying distribution and the values of some parameters.

\begin{enumerate}
\item If we are risen endless area for sampling, we have, in general, a
solution as the sum of the difference of the logarithms (NBL's style, perhaps
with an extra multiplicative factor) and a difference of periodic functions
(\ref{S2},\ref{S2a}). Moreover, the presence of periodic functions is
important because they make up $30\%-40\%$ of the value of the DSLD. The
natural emergence of periodic functions is defined by a transformation group
and infinite boundaries.

\item The presence of finite boundaries leads mainly to distributions of the
third type (\ref{S3})%
\begin{equation}
\rho(k)=\Omega(\log(k+1)-\lfloor\log(k)\rfloor)-\Omega(\log(k)-\lfloor
\log(k)\rfloor).
\end{equation}
As shown above in the case of the production of $n$ numbers obeying a UD, with
an increasing number of factors, the main contribution gives the members in
the inner region and the edge interval effectively does not play a significant
role. But even small boundary effects eliminate the presence of a periodic
function. These types of underlying distributions are perhaps the best
candidates for the NBL.
\end{enumerate}

Knowing the possible structure of the function $\rho(k)$, we can offer a
procedure to make a Monte Carlo approximation with experimental data. Because
with $1\ll\log(k))$\ we have from (\ref{AsymptK})%
\[
\rho(k)=\frac{1}{\ln(10)10^{s}}\left(  A+\frac{dQ_{1}(s)}{ds}\right)
+o\left(  10^{-s}\right)  ,s=\log(k).
\]
\ \ For example, for (\ref{S2},\ref{S2a}), we can get from $1\ll\log(k)$ (in
many cases $3\leq\log(k)$) after statistical experimental density function
$\rho(1\ll\log(k))=\eta(s)),s=\log(k).$

Then make the approximation
\begin{align*}
10^{s}\ln(10)f(s) &  \approx a_{0}+a_{1}\sin(2\pi s)+b_{1}\cos(2\pi
s)+\ldots,S\leq s\leq S+1.\\
G(s) &  =%
%TCIMACRO{\dint \limits_{\tau=0}^{s}}%
%BeginExpansion
{\displaystyle\int\limits_{\tau=0}^{s}}
%EndExpansion
10^{\tau}\ln(10)\eta(\tau)d\tau.
\end{align*}
So%
\[
G(s)=const+a_{0}s-\frac{a_{1}}{2\pi}\cos(2\pi s)+\frac{b_{1}}{2\pi}\sin(2\pi
s)+\ldots
\]
and%
\begin{align*}
\rho(k) &  \approx G(s_{1})-G(s),k\in%
%TCIMACRO{\U{2124} }%
%BeginExpansion
\mathbb{Z}
%EndExpansion
^{\ast},\\
s_{1} &  =\log(k+1)-\lfloor\log(k)\rfloor.s=\log(k)-\lfloor\log(k)\rfloor.
\end{align*}
In many cases, taking $S=4$ gives pretty good results.

\section{Conclusions}

The main results of this paper are as follows.

\begin{enumerate}
\item A general form for the DSLD is presented (\ref{S3}).

\item It is shown that in a number of cases this general solution takes the
simpler forms (\ref{S1}) and (\ref{S2}).

\item It is proved that instead of the NBL it is natural to look for, in
permissible cases, a DSLD of the form (\ref{S2}).

\item The relation between the DSLD and the density of the underlying
distribution is presented in the form of an integral relation.

\item The function $P(k,x)$ (\ref{EMF02}) is introduced, which is convenient
for isolating the interval of the leading digits.

\item There has been presented some exact solutions and their relation to the
general form of the DSLD.

\item In a number of cases, a Fourier representation is found that can be used
to find approximate solutions by cutting short its infinite series. In many
practical cases, only the first few terms are sufficient. These approximate
analytical formulas provide a high degree of accuracy.

\item It is shown that knowing the results of a numerical experiment for
sufficiently large leading numbers, it is possible to recover the solution for
their entire range.

\item In many cases, the function $\Omega(s)$ has a quasipolynomial form (with
constants $\alpha_{i}$\ and $\beta_{ij}$):
\end{enumerate}

%

\[
\Omega(s)=%
%TCIMACRO{\dsum \limits_{i}}%
%BeginExpansion
{\displaystyle\sum\limits_{i}}
%EndExpansion
10^{\alpha_{i}s}%
%TCIMACRO{\dsum \limits_{j}}%
%BeginExpansion
{\displaystyle\sum\limits_{j}}
%EndExpansion
\beta_{ij}s^{j}%
\]
We have found the forms of the DSLD for a set of underlying distributions and
presented closed forms for the DSLD\textbf{\ } for a few of them. We have
introduced an integral representation of the DSLD\textbf{\ }as a function of
the underlying distribution.

From these results, we conclude that the logarithmic terms are native terms
with sums with other oscillating terms. In many cases, their contribution is
essential. Regarding the NBL, it is in many cases a better or worse
approximation of more complicated forms of distributions, and can be exact
only in a few rare cases, which neglect the oscillating terms.

We found some solutions in the form of an infinite series which with high
accuracy can be substituted by the sum of its first few terms. We are
confident that our approach can be applied to the analysis of many applied
problems. In some cases, it may be preferable to use this approach, but not,
for instance, Monte Carlo methods, which are known for their properties of
slow convergence.

Here we can formulate a necessary condition for the realization of the NBL%

\begin{equation}
\left\vert \frac{1}{\pi n}%
%TCIMACRO{\dint \limits_{x=0}^{+\infty}}%
%BeginExpansion
{\displaystyle\int\limits_{x=0}^{+\infty}}
%EndExpansion
x^{Y\imath}f(x)dx\right\vert \ll%
%TCIMACRO{\dint \limits_{x=0}^{\infty}}%
%BeginExpansion
{\displaystyle\int\limits_{x=0}^{\infty}}
%EndExpansion
f(x)dx,Y=\frac{2\pi n}{\ln(10)},n\in%
%TCIMACRO{\U{2115} }%
%BeginExpansion
\mathbb{N}
%EndExpansion
.\label{NBLcond}%
\end{equation}
From a practical point of view, the ratio of an (oscillating part)/(log part)
should be $o(10^{-3}).$

This approach can be applied for different radices (bases), and then%
\begin{align*}
\rho(k,B)  &  =\Omega(\log_{B}(k+1)-\lfloor\log_{B}(k)\rfloor)-\Omega\left(
\log_{B}(k)-\lfloor\log_{B}(k)\rfloor\right)  ),\\
\log_{B}(k)  &  =\frac{\ln(k)}{\ln(B)},1<B,B\text{ positive integer.}%
\end{align*}
%

%TCIMACRO{\TeXButton{Appendices}{\section*{APPENDICES}}}%
%BeginExpansion
\section*{APPENDICES}%
%EndExpansion
%

%TCIMACRO{\TeXButton{Define Title of Appendix}%
%{Now we will look in more detail at some of the technical aspects}}%
%BeginExpansion
Now we will look in more detail at some of the technical aspects%
%EndExpansion
%

%TCIMACRO{\TeXButton{Beginning of Appendix}{\appendix}}%
%BeginExpansion
\appendix
%EndExpansion


\section{Solution of the Functional Equations}%

%TCIMACRO{\TeXButton{\label{SofFE}}{\label{SofFE}}}%
%BeginExpansion
\label{SofFE}%
%EndExpansion


Let's temporarily forget about probabilities and consider (\ref{eq1}%
,\ref{eq2}$)$ as functional equations with specific symmetry properties. For
the PDF of the leading digits we have%

\begin{align*}
k &  \in\lbrack10^{L},10^{L+1}-1],L\in%
%TCIMACRO{\U{2124} }%
%BeginExpansion
\mathbb{Z}
%EndExpansion
^{\ast},\\
L &  =\lfloor\log(k)\rfloor,0\leq\rho(k,L)\leq1.
\end{align*}%
\begin{equation}%
%TCIMACRO{\dsum \limits_{k=10^{L}}^{10^{L+1}-1}}%
%BeginExpansion
{\displaystyle\sum\limits_{k=10^{L}}^{10^{L+1}-1}}
%EndExpansion
\rho(k,L)=\digamma.\label{eq1}%
\end{equation}%
\begin{equation}
\rho(k,0)=%
%TCIMACRO{\dsum \limits_{i=0}^{10^{L}-1}}%
%BeginExpansion
{\displaystyle\sum\limits_{i=0}^{10^{L}-1}}
%EndExpansion
\rho(10^{L}k+i,L),L\in%
%TCIMACRO{\U{2124} }%
%BeginExpansion
\mathbb{Z}
%EndExpansion
^{\ast}.\label{eq2}%
\end{equation}


\subsection{Partial solutions}%

%TCIMACRO{\TeXButton{PartialSolutions}{\label{PartialSolutions}}}%
%BeginExpansion
\label{PartialSolutions}%
%EndExpansion
Using equations (\ref{eq1},\ref{eq2}) we can find the partial solutions. Now
we will consider different types of solutions of the equations.

\begin{itemize}
\item $\rho(k)=f(L)$
\end{itemize}

After substitution in (\ref{eq2}) we obtain%
\begin{align*}
f(0)  &  =10^{L}f(L),\\
f(L)  &  =C_{0}10^{-L}.
\end{align*}
From equation (\ref{eq1}) we have%
\begin{align}
f(L)  &  =10^{-L}\frac{\digamma}{9},\label{SolL}\\
\rho(k)  &  =\frac{\digamma}{9}10^{-\lfloor\log(k)\rfloor}.\nonumber
\end{align}


\begin{itemize}
\item $\rho(k,L)=f(k)$
\end{itemize}

Then%
\begin{equation}
f(k)=%
%TCIMACRO{\dsum \limits_{i=0}^{10^{L}-1}}%
%BeginExpansion
{\displaystyle\sum\limits_{i=0}^{10^{L}-1}}
%EndExpansion
f(10^{L}k+i),L=0,1,\ldots. \label{SolEq2}%
\end{equation}


Bearing in mind (\ref{MainFormulation}), we will consider solutions of the
form%
\begin{equation}
f(k)=g(k+1)-g(k). \label{Sdiff1}%
\end{equation}
Then for $L=1$
\begin{align*}
g(k+1)-g(k)  &  =g(10(k+1))-g(10k),\\
g(10k)-g(k)  &  =g(10(k+1))-g(k+1).
\end{align*}
then
\begin{align*}
g(10k)-g(k)  &  =g(10(k+1))-g(k+1)=A=const,\\
k  &  \in%
%TCIMACRO{\U{2124} }%
%BeginExpansion
\mathbb{Z}
%EndExpansion
.
\end{align*}
We get the functional equation
\begin{equation}
g(10k)-g(k)=A. \label{funcEq1}%
\end{equation}


Let us solve the functional equation (\ref{funcEq1}). Introduce the parameter
$\tau$ and two unknown functions $\psi(\tau)$ and $\varphi(\tau)$%

\begin{align}
k  &  =\varphi(\tau),10k=\varphi(\tau+1),\label{Fun1}\\
g(k)  &  =\psi(\tau),g(10k)=\psi(\tau+1).\nonumber
\end{align}%
\begin{align}
10\varphi(\tau)  &  =\varphi(\tau+1),\label{Fun2}\\
\psi(\tau+1)-\psi(\tau)  &  =A.\nonumber
\end{align}
The solutions of the difference equations are%
\begin{align*}
\varphi(\tau)  &  =10^{\tau}P_{1}(\tau),P_{1}(\tau+1)=P_{1}(\tau),\forall
\tau,\\
\psi(\tau)  &  =A\tau+Q_{1}(\tau),Q_{1}(\tau+1)=Q_{1}(\tau),\forall\tau.
\end{align*}
where $P_{1}(\tau)$ and $Q_{1}(\tau)\ $\ are arbitrary functions with period
$1$. Then we have a parametric solution of (\ref{funcEq1})%
\begin{align}
k  &  =10^{\tau}P_{1}(\tau),\label{Param1}\\
g(k)  &  =A\tau+Q_{1}(\tau).\nonumber
\end{align}
In the partial case when
\[
P_{1}(\tau)=1,Q_{1}(\tau)=0,A=1.
\]
we have the NBL distribution%
\begin{equation}
\rho(k)=\log(k+1)-\log(k).
\end{equation}


If put $P_{1}(\tau)=const$ we have%
\begin{align*}
\tau &  =\log(k)+c,\\
g(k)  &  =A\log(k)+Ac+Q_{1}(\log(k)+c),
\end{align*}
%

\begin{equation}
\rho(k)=A\left(  \log(k+1)-\log(k)\right)  +Q_{1}(\log(k+1)+c)-Q_{1}%
(\log(k)+c). \label{SolK0}%
\end{equation}


From (\ref{eq1})
\begin{align}%
%TCIMACRO{\dsum \limits_{k=10^{L}}^{10^{L+1}-1}}%
%BeginExpansion
{\displaystyle\sum\limits_{k=10^{L}}^{10^{L+1}-1}}
%EndExpansion
\rho(k)  &  =\digamma,\\
A+Q_{1}(L+1+c)-Q_{1}(L+c)  &  =\digamma,\nonumber\\
A  &  =\digamma.\nonumber
\end{align}%
\begin{equation}
\rho(k)=\digamma\left(  \log(k+1)-\log(k)\right)  +Q_{1}(\log(k+1)+c)-Q_{1}%
(\log(k)+c). \label{SolK}%
\end{equation}


If $1\ll k$, we have%
\begin{equation}
\rho(k\rightarrow\infty)=\frac{1}{\ln(10)}\frac{1}{k}+\frac{1}{\ln(10)}%
\frac{1}{k}\left.  \frac{dQ_{1}(s)}{ds}\right\vert _{s=\log(k)+c}+o\left(
\frac{1}{k}\right)  . \label{AsymptK}%
\end{equation}
If we have some approximation (from experimental data and/or from numerical
modeling) $f(s)$ where $\hat{s}\leq s\leq\hat{s}+1$ and $3\leq\hat{s}$,

\noindent then%
\begin{align*}
G(s) &  =-s+\ln(10)%
%TCIMACRO{\dint \limits^{s}}%
%BeginExpansion
{\displaystyle\int\limits^{s}}
%EndExpansion
10^{t}f(t)dt,\\
\rho(k) &  \approx-\log\left(  \frac{k+1}{k}\right)  +\ln(10)%
%TCIMACRO{\dint \limits_{s}^{s_{1}}}%
%BeginExpansion
{\displaystyle\int\limits_{s}^{s_{1}}}
%EndExpansion
10^{t}f(t)dt,\\
s_{1} &  =\log\left(  k+1\right)  +c,s=\log\left(  k\right)  +c,k=1,2,\ldots.
\end{align*}


\begin{itemize}
\item $\rho(k,L)$
\end{itemize}

After some algebra similar to an earlier item, we find%
\begin{equation}
\rho(k)=\Omega(\log(k+1)-\lfloor\log(k)\rfloor)-\Omega(\log(k)-\lfloor
\log(k)\rfloor). \label{SolK_L}%
\end{equation}


\noindent where $\Omega(s)$ is a function of $s$ subject to the restriction%
\[
0\leq\rho(k,L)\leq1.
\]
Direct substitution of (\ref{SolK_L}) in (\ref{eq2}) give%
\begin{align}
&  \Omega(\log(k+1))-\Omega(\log(k))\label{identical2}\\
&  =%
%TCIMACRO{\dsum \limits_{i=0}^{10^{L}-1}}%
%BeginExpansion
{\displaystyle\sum\limits_{i=0}^{10^{L}-1}}
%EndExpansion
\left(  \Omega(\log(10^{L}k+i+1)-L)-\Omega(\log(10^{L}k+i)-L)\right)
\nonumber\\
&  =\Omega(\log(k+1))-\Omega(\log(k)),L=0,1,\ldots.\nonumber
\end{align}


\noindent and from (\ref{eq1})%
\begin{align}
&
%TCIMACRO{\dsum \limits_{i=10^{L}}^{10^{L+1}-1}}%
%BeginExpansion
{\displaystyle\sum\limits_{i=10^{L}}^{10^{L+1}-1}}
%EndExpansion
\left(  \Omega(\log(i+1)-L)-\Omega(\log(i)-L)\right) \label{identical1}\\
&  =\Omega(1)-\Omega(0)=\digamma,\nonumber\\
L  &  =0,1,\ldots.\nonumber
\end{align}


For the distributions of the digits in the second and third positions we have%
\begin{align*}
\rho_{2}(m)  &  =%
%TCIMACRO{\dsum \limits_{i=1}^{9}}%
%BeginExpansion
{\displaystyle\sum\limits_{i=1}^{9}}
%EndExpansion
\rho(10i+m),\\
\rho_{3}(m)  &  =%
%TCIMACRO{\dsum \limits_{i=1}^{9}}%
%BeginExpansion
{\displaystyle\sum\limits_{i=1}^{9}}
%EndExpansion%
%TCIMACRO{\dsum \limits_{j=0}^{9}}%
%BeginExpansion
{\displaystyle\sum\limits_{j=0}^{9}}
%EndExpansion
\rho(100i+10j+m),m=0,1,\ldots,9.
\end{align*}
For a solution of (\ref{SolL}) we have%
\[
\Omega(s)=C10^{s}.
\]
and for (\ref{SolK0})%
\[
\Omega(s)=C\log(s)+Q_{1}(s).
\]


\section{From Sum to Integral in (\ref{MainFormulation})}%

%TCIMACRO{\TeXButton{EulerMaclaurin}{\label{EulerMaclaurin}}}%
%BeginExpansion
\label{EulerMaclaurin}%
%EndExpansion
Euler--Maclaurin's formula (\textbf{EMf})%
%TCIMACRO{\TeXButton{\cite_abramowitz}{\cite{knuth} ,\cite{wabramowitzstegun}
%and \cite{knuth2}}}%
%BeginExpansion
\cite{knuth} ,\cite{wabramowitzstegun}
and \cite{knuth2}%
%EndExpansion
\ gives a representation of sums in form of definite integrals. It is
important to stress that the \textbf{EMf} is an exact representation of the
sum in integral form (at least for differentiable functions, it is not an
approximation). From \textbf{EMf} we have a representation of a sum in the
form of an integral by (see Appendix
%TCIMACRO{\TeXButton{\ref{DerForm}}{\ref{DerForm}}}%
%BeginExpansion
\ref{DerForm}%
%EndExpansion
)\
\begin{equation}%
%TCIMACRO{\dsum \limits_{j=a}^{b}}%
%BeginExpansion
{\displaystyle\sum\limits_{j=a}^{b}}
%EndExpansion
g(j)=%
%TCIMACRO{\dint \limits_{x=a}^{b}}%
%BeginExpansion
{\displaystyle\int\limits_{x=a}^{b}}
%EndExpansion
\left(  g(x)+\left(  x-\left\lfloor x\right\rfloor -\frac{1}{2}\right)
\frac{dg(x)}{dx}\right)  dx+\frac{g(b)}{2}+\frac{g(a)}{2},a<b. \label{EM00}%
\end{equation}


We will apply Euler--Maclaurin summation in this form, which is convenient for
our purposes.

In our case we have%
\begin{align*}
g(j)  &  =F(10^{j+\log(k+1)})-F(10^{j+\log(k)})\\
F(z)  &  =%
%TCIMACRO{\dint \limits_{-\infty}^{z}}%
%BeginExpansion
{\displaystyle\int\limits_{-\infty}^{z}}
%EndExpansion
f(t)dt.
\end{align*}
We can, without loss of generality, put infinite limits of integration, since
it is always possible to represent the density distribution $\tilde{f}(t)$ as%
\begin{equation}
f(t)=\left\{
\begin{array}
[c]{cc}%
0, & t_{\max}<t,\\
\tilde{f}(t) & t\in\lbrack t_{\min},t_{\max}]\\
0 & t<t_{\min}.
\end{array}
\right.  \label{EM00x}%
\end{equation}


Let us estimate the sums and integrals in (\ref{MainFormulation}).

\subsection{Infinite limits}%

%TCIMACRO{\TeXButton{InfinitLimi}{\label{InfinitLimit}}}%
%BeginExpansion
\label{InfinitLimit}%
%EndExpansion
We will consider the case where $j_{\min_{{}}}=-$ $\infty$ and $j_{\max
}=\infty.$ From the Euler--Maclaurin formula, for convention sum to integral
we have%
\begin{align}%
%TCIMACRO{\dsum \limits_{j=-\infty}^{\infty}}%
%BeginExpansion
{\displaystyle\sum\limits_{j=-\infty}^{\infty}}
%EndExpansion
g(j)  &  =%
%TCIMACRO{\dint \limits_{x=-\infty}^{\infty}}%
%BeginExpansion
{\displaystyle\int\limits_{x=-\infty}^{\infty}}
%EndExpansion
\left(  g(x)+\frac{dg(x)}{dx}\left(  x-\left\lfloor x\right\rfloor -\frac
{1}{2}\right)  \right)  dx,g(\pm\infty)=0,\label{EM1}\\
S  &  =I+R,\nonumber\\
I  &  =%
%TCIMACRO{\dint \limits_{x=-\infty}^{\infty}}%
%BeginExpansion
{\displaystyle\int\limits_{x=-\infty}^{\infty}}
%EndExpansion
g(x)dx,R=%
%TCIMACRO{\dint \limits_{x=-\infty}^{\infty}}%
%BeginExpansion
{\displaystyle\int\limits_{x=-\infty}^{\infty}}
%EndExpansion
\left(  \frac{dg(x)}{dx}\left(  x-\left\lfloor x\right\rfloor -\frac{1}%
{2}\right)  \right)  dx.\nonumber
\end{align}


With
\begin{align*}
S  &  =%
%TCIMACRO{\dsum \limits_{j=-\infty}^{\infty}}%
%BeginExpansion
{\displaystyle\sum\limits_{j=-\infty}^{\infty}}
%EndExpansion
\left(  F(10^{j}(k+1))-F(10^{j}k)\right)  ,\\
F(z)  &  =%
%TCIMACRO{\dint \limits_{x=-\infty}^{z}}%
%BeginExpansion
{\displaystyle\int\limits_{x=-\infty}^{z}}
%EndExpansion
f(x)dx,f(\pm\infty)=0.
\end{align*}
and we have%
\begin{align}
I  &  =%
%TCIMACRO{\dint \limits_{x=-\infty}^{\infty}}%
%BeginExpansion
{\displaystyle\int\limits_{x=-\infty}^{\infty}}
%EndExpansion
\left(  F(10^{x}(k+1))-F(10^{x}k)\right)  dx\label{EM2}\\
&  =%
%TCIMACRO{\dint \limits_{x=-\infty}^{\infty}}%
%BeginExpansion
{\displaystyle\int\limits_{x=-\infty}^{\infty}}
%EndExpansion%
%TCIMACRO{\dint \limits_{t=\alpha}^{t=\beta}}%
%BeginExpansion
{\displaystyle\int\limits_{t=\alpha}^{t=\beta}}
%EndExpansion
f(t)dtdx,\nonumber\\
\alpha &  =10^{x}k,\beta=10^{x}(k+1).\nonumber
\end{align}
After integration by part we have%
\[
I=-%
%TCIMACRO{\dint \limits_{x=-\infty}^{\infty}}%
%BeginExpansion
{\displaystyle\int\limits_{x=-\infty}^{\infty}}
%EndExpansion
x\left(  f(\beta)\frac{d\beta}{dx}-f(\alpha)\frac{d\alpha}{dx}\right)  dx.
\]
and%
\begin{align*}%
%TCIMACRO{\dint \limits_{x=-\infty}^{\infty}}%
%BeginExpansion
{\displaystyle\int\limits_{x=-\infty}^{\infty}}
%EndExpansion
xf(\alpha)\frac{d\alpha}{dx}dx  &  =%
%TCIMACRO{\dint \limits_{s=0}^{\infty}}%
%BeginExpansion
{\displaystyle\int\limits_{s=0}^{\infty}}
%EndExpansion
\left(  \log(s)-\log(k)\right)  f(s)ds\\
&  =-\log(k)%
%TCIMACRO{\dint \limits_{s=0}^{\infty}}%
%BeginExpansion
{\displaystyle\int\limits_{s=0}^{\infty}}
%EndExpansion
f(s)ds+%
%TCIMACRO{\dint \limits_{\alpha=0}^{\infty}}%
%BeginExpansion
{\displaystyle\int\limits_{\alpha=0}^{\infty}}
%EndExpansion
\log(s)f(s)ds.
\end{align*}
Then the integral part is%
\begin{equation}
I=\left(  \log(k+1)-\log(k)\right)
%TCIMACRO{\dint \limits_{s=0}^{\infty}}%
%BeginExpansion
{\displaystyle\int\limits_{s=0}^{\infty}}
%EndExpansion
f(s)ds. \label{Iform}%
\end{equation}
For the evaluation of the second term in (\ref{EM1}), we have%
\[
R=%
%TCIMACRO{\dint \limits_{x=-\infty}^{\infty}}%
%BeginExpansion
{\displaystyle\int\limits_{x=-\infty}^{\infty}}
%EndExpansion
\left(  f(\beta)\frac{d\beta}{dx}-f(\alpha)\frac{d\alpha}{dx}\right)  \left(
x-\left\lfloor x\right\rfloor -\frac{1}{2}\right)  dx
\]
and%
\begin{align*}
&
%TCIMACRO{\dint \limits_{x=-\infty}^{\infty}}%
%BeginExpansion
{\displaystyle\int\limits_{x=-\infty}^{\infty}}
%EndExpansion
f(\alpha)\frac{d\alpha}{dx}\left(  x-\left\lfloor x\right\rfloor -\frac{1}%
{2}\right)  dx\\
&  =%
%TCIMACRO{\dint \limits_{s=0}^{\infty}}%
%BeginExpansion
{\displaystyle\int\limits_{s=0}^{\infty}}
%EndExpansion
f(s)\left(  \log\left(  \frac{s}{k}\right)  -\left\lfloor \log\left(  \frac
{s}{k}\right)  \right\rfloor -\frac{1}{2}\right)  ds,
\end{align*}
then the residual part is%
\begin{align}
R  &  =-%
%TCIMACRO{\dint \limits_{s=0}^{\infty}}%
%BeginExpansion
{\displaystyle\int\limits_{s=0}^{\infty}}
%EndExpansion
f(s)\left(  \log(k+1)-\log(k)-\left\lfloor \log\left(  \frac{s}{k+1}\right)
\right\rfloor +\left\lfloor \log\left(  \frac{s}{k}\right)  \right\rfloor
\right)  ds\label{Rform}\\
&  =-\left(  \log(k+1)-\log(k)\right)
%TCIMACRO{\dint \limits_{s=0}^{\infty}}%
%BeginExpansion
{\displaystyle\int\limits_{s=0}^{\infty}}
%EndExpansion
f(s)ds+%
%TCIMACRO{\dint \limits_{t=0}^{\infty}}%
%BeginExpansion
{\displaystyle\int\limits_{t=0}^{\infty}}
%EndExpansion
f(t)\left(  \left\lfloor \log\left(  \frac{t}{k}\right)  \right\rfloor
-\left\lfloor \log\left(  \frac{t}{k+1}\right)  \right\rfloor \right)
dt.\nonumber
\end{align}
We obtain the general form of $\rho(k)$ as%
\begin{equation}
\rho(k)=%
%TCIMACRO{\dint \limits_{s=0}^{\infty}}%
%BeginExpansion
{\displaystyle\int\limits_{s=0}^{\infty}}
%EndExpansion
f(t)\left(  \left\lfloor \log\left(  \frac{t}{k}\right)  \right\rfloor
-\left\lfloor \log\left(  \frac{t}{k+1}\right)  \right\rfloor \right)  dt.
\label{EM5}%
\end{equation}


From this we have an integral representation of $\Omega(s)$ (provided the
integral exists)%
\begin{equation}
\Omega(s)=-%
%TCIMACRO{\dint \limits_{t=0}^{\infty}}%
%BeginExpansion
{\displaystyle\int\limits_{t=0}^{\infty}}
%EndExpansion
f(t)\left\lfloor \log\left(  t\right)  -s\right\rfloor dt+const
\label{OmegaInt}%
\end{equation}
\bigskip

\bigskip Another form is%
\begin{align}
\rho(k)  &  =\digamma\left(  \log(k+1)-\log(k)\right)  +\label{RMain}\\
&
%TCIMACRO{\dint \limits_{s=0}^{\infty}}%
%BeginExpansion
{\displaystyle\int\limits_{s=0}^{\infty}}
%EndExpansion
f(s)\left(  \log\left(  \frac{s}{k+1}\right)  -\left\lfloor \log\left(
\frac{s}{k+1}\right)  \right\rfloor \right)  ds-%
%TCIMACRO{\dint \limits_{s=0}^{\infty}}%
%BeginExpansion
{\displaystyle\int\limits_{s=0}^{\infty}}
%EndExpansion
f(s)\left(  \log\left(  \frac{s}{k}\right)  -\left\lfloor \log\left(  \frac
{s}{k}\right)  \right\rfloor \right)  ds,\nonumber\\
\digamma &  =%
%TCIMACRO{\dint \limits_{s=0}^{\infty}}%
%BeginExpansion
{\displaystyle\int\limits_{s=0}^{\infty}}
%EndExpansion
f(s)ds.
\end{align}
If we introduce the $1$-periodic function $Q_{1}(s)$
\begin{align}
Q_{1}(\log(k))  &  =%
%TCIMACRO{\dint \limits_{s=0}^{\infty}}%
%BeginExpansion
{\displaystyle\int\limits_{s=0}^{\infty}}
%EndExpansion
f(t)\left(  \log\left(  \frac{t}{k}\right)  -\left\lfloor \log\left(  \frac
{t}{k}\right)  \right\rfloor \right)  dt,\label{EM7}\\
Q_{1}(s+1)  &  =Q_{1}(s).\nonumber
\end{align}%
\begin{align}
Q_{1}(\log(k))  &  =%
%TCIMACRO{\dint \limits_{x=-\infty}^{\infty}}%
%BeginExpansion
{\displaystyle\int\limits_{x=-\infty}^{\infty}}
%EndExpansion
\ln(10)10^{x}kf(10^{x}k)\left(  x-\left\lfloor x\right\rfloor \right)  dx,\\
Q_{1}(s+1)  &  =Q_{1}(s).\nonumber
\end{align}


\noindent and because%
\[
0\leq\left(  \log\left(  \frac{s}{k}\right)  -\left\lfloor \log\left(
\frac{s}{k}\right)  \right\rfloor \right)  \leq1
\]
then%
\[
\left\vert Q_{1}(\log(k))\right\vert \leq\digamma.
\]
Then we can rewrite (\ref{RMain}) as%

\begin{equation}
S=\rho(k)=\digamma\left(  \log\left(  k+1\right)  -\log\left(  k\right)
\right)  +Q_{1}(\log(k+1))-Q_{1}(\log(k)). \label{MainEM0}%
\end{equation}


\subsection{Fourier Expansion}

\
%TCIMACRO{\TeXButton{FourieExpan}{\label{FourieExpan}}}%
%BeginExpansion
\label{FourieExpan}%
%EndExpansion
\ Let us consider the $1$-periodic function $Q_{1}(s)$ (\ref{EM7})
\begin{equation}
Q_{1}(s)=%
%TCIMACRO{\dint \limits_{x=0}^{\infty}}%
%BeginExpansion
{\displaystyle\int\limits_{x=0}^{\infty}}
%EndExpansion
f(x)\left(  \log\left(  x\right)  -s-\left\lfloor \log\left(  x\right)
-s\right\rfloor \right)  dx. \label{Fouri1}%
\end{equation}


We will consider the Fourier expansion%
\[
Q_{1}(s)=const+%
%TCIMACRO{\dsum \limits_{j=1}^{\infty}}%
%BeginExpansion
{\displaystyle\sum\limits_{j=1}^{\infty}}
%EndExpansion
\left(  A_{j}\sin\left(  2\pi js\right)  +B_{j}\cos\left(  2\pi js\right)
\right)  .
\]
Because%
\begin{align*}
a_{n}(c)  &  =%
%TCIMACRO{\dint \limits_{s=0}^{1}}%
%BeginExpansion
{\displaystyle\int\limits_{s=0}^{1}}
%EndExpansion
\left(  c-s-\left\lfloor c-s\right\rfloor \right)  \sin\left(  2\pi ns\right)
ds\left(
%TCIMACRO{\dint \limits_{s=0}^{1}}%
%BeginExpansion
{\displaystyle\int\limits_{s=0}^{1}}
%EndExpansion
\sin\left(  2\pi ns\right)  ^{2}ds\right)  ^{-1},\\
b_{n}(c)  &  =%
%TCIMACRO{\dint \limits_{s=0}^{1}}%
%BeginExpansion
{\displaystyle\int\limits_{s=0}^{1}}
%EndExpansion
\left(  c-s-\left\lfloor c-s\right\rfloor \right)  \cos\left(  2\pi ns\right)
ds\left(
%TCIMACRO{\dint \limits_{s=0}^{1}}%
%BeginExpansion
{\displaystyle\int\limits_{s=0}^{1}}
%EndExpansion
\cos\left(  2\pi ns\right)  ^{2}ds\right)  ^{-1}.
\end{align*}
and%
\begin{align*}
\left(  \log\left(  x\right)  -s-\left\lfloor \log\left(  x\right)
-s\right\rfloor \right)   &  =a_{0}+%
%TCIMACRO{\dsum \limits_{n=1}^{\infty}}%
%BeginExpansion
{\displaystyle\sum\limits_{n=1}^{\infty}}
%EndExpansion
\left(  a_{n}\sin\left(  2\pi ns\right)  +b_{n}\cos\left(  2\pi ns\right)
\right) \\
a_{n}  &  =\frac{\cos(2\pi n\log\left(  x\right)  )}{n\pi},b_{n}=-\frac
{\sin(2\pi n\log\left(  x\right)  )}{n\pi},1\leq n.
\end{align*}
we have then%
\begin{align}
\digamma &  =%
%TCIMACRO{\dint \limits_{x=0}^{\infty}}%
%BeginExpansion
{\displaystyle\int\limits_{x=0}^{\infty}}
%EndExpansion
f(x)dx,\label{Fouri2}\\
A_{n}  &  =\frac{1}{n\pi}%
%TCIMACRO{\dint \limits_{x=0}^{\infty}}%
%BeginExpansion
{\displaystyle\int\limits_{x=0}^{\infty}}
%EndExpansion
f(x)\cos(2\pi n\log\left(  x\right)  )dx,\\
B_{n}  &  =-\frac{1}{n\pi}%
%TCIMACRO{\dint \limits_{x=0}^{\infty}}%
%BeginExpansion
{\displaystyle\int\limits_{x=0}^{\infty}}
%EndExpansion
f(x)\sin(2\pi n\log\left(  x\right)  )dx,1\leq n.\nonumber
\end{align}


\noindent in another form
\begin{align}
A_{n}+\imath B_{n}  &  =\frac{1}{n\pi}%
%TCIMACRO{\dint \limits_{x=0}^{\infty}}%
%BeginExpansion
{\displaystyle\int\limits_{x=0}^{\infty}}
%EndExpansion
f(x)\exp(-2\pi n\log\left(  x\right)  \imath)dx\label{A_B_Fourier}\\
&  =\frac{1}{n\pi}%
%TCIMACRO{\dint \limits_{x=0}^{\infty}}%
%BeginExpansion
{\displaystyle\int\limits_{x=0}^{\infty}}
%EndExpansion
x^{-\imath Y}f(x)dx,\nonumber\\
Y  &  =\frac{2\pi n}{\ln(10)}.\nonumber
\end{align}
or in complex polar form we have%
\[
A_{n}+\imath B_{n}=\Lambda(n)\exp(\imath\varphi(n))=\frac{1}{n\pi}%
%TCIMACRO{\dint \limits_{x=0}^{\infty}}%
%BeginExpansion
{\displaystyle\int\limits_{x=0}^{\infty}}
%EndExpansion
x^{-\imath Y}f(x)dx.
\]


Then%
\begin{align}
\rho(k)  &  =\digamma(\log\left(  k+1\right)  -\log\left(  k\right)
)+\label{Fouri3}\\
&
%TCIMACRO{\dsum \limits_{n=1}^{\infty}}%
%BeginExpansion
{\displaystyle\sum\limits_{n=1}^{\infty}}
%EndExpansion
\left(
\begin{array}
[c]{c}%
A_{n}\left(  \sin\left(  ns_{1}\right)  -\sin\left(  ns\right)  \right) \\
+B_{n}(\cos\left(  ns_{1}\right)  -\cos\left(  ns\right)  )
\end{array}
\right)  ,\nonumber\\
s_{1}  &  =2\pi\log\left(  k+1\right)  ,s=2\pi\log\left(  k\right)  .\nonumber
\end{align}


\noindent then%
\begin{align}
\rho(k)  &  =\digamma(\log\left(  k+1\right)  -\log\left(  k\right)
)+\label{Fouri4}\\
&
%TCIMACRO{\dsum \limits_{n=1}^{\infty}}%
%BeginExpansion
{\displaystyle\sum\limits_{n=1}^{\infty}}
%EndExpansion
\Lambda(n)\left(  \sin\left(  ns_{1}+\varphi(n)\right)  -\sin\left(
ns+\varphi(n)\right)  \right)  .\nonumber
\end{align}


\section{Ratio of Two Positive Numbers with a UD}%

%TCIMACRO{\TeXButton{\label{R2PNUD}}{\label{R2PNUD}}}%
%BeginExpansion
\label{R2PNUD}%
%EndExpansion


Because
\[
f_{Z}(z)=%
%TCIMACRO{\dint }%
%BeginExpansion
{\displaystyle\int}
%EndExpansion
\delta\left(  z-g(x_{1}\ldots x_{n})\right)  dx_{1}\ldots dx_{n}.
\]
\bigskip

\noindent where $\delta$\ is the Dirac delta function, we can find the PDF for
different transformations of stochastical variables.\ Let us assume that
$f_{X}(x)$ is a PDF for $0\leq x$ and $f_{Y}(y)$ is a PDF for $0\leq y$. We
need to find the density of $z=\frac{y}{x}$. If the two random variables are
independent, then%

\[
f_{XY}(x,y)=f_{X}(x)f_{Y}(y)
\]
The distribution function for $Z$ is%
\[
F(z)=%
%TCIMACRO{\dint \limits^{z}}%
%BeginExpansion
{\displaystyle\int\limits^{z}}
%EndExpansion
f_{Z}(s)ds.
\]
Let us transform the variables%
\[
x=\alpha,y=\beta\alpha.
\]
Then the Jacobian of the transformation is%
\[
J=\frac{\partial\left(  x,y\right)  }{\partial\left(  \alpha,\beta\right)
}=\det\left(
\begin{array}
[c]{cc}%
\frac{\partial x}{\partial\alpha} & \frac{\partial y}{\partial\alpha}\\
\frac{\partial x}{\partial\beta} & \frac{\partial y}{\partial\beta}%
\end{array}
\right)  =\det\left(
\begin{array}
[c]{cc}%
1 & \beta\\
0 & \alpha
\end{array}
\right)  =\alpha.
\]
in case ($x\in(-\infty\ldots+\infty)$ and $y\in(-\infty\ldots+\infty) $)
\begin{align*}
F(z)  &  =%
%TCIMACRO{\dint \limits_{-\infty}^{z}}%
%BeginExpansion
{\displaystyle\int\limits_{-\infty}^{z}}
%EndExpansion
\left[
%TCIMACRO{\dint \limits_{-\infty}^{+\infty}}%
%BeginExpansion
{\displaystyle\int\limits_{-\infty}^{+\infty}}
%EndExpansion
\alpha f_{XY}(\alpha,z\alpha)d\alpha\right]  d\beta\\
&  =%
%TCIMACRO{\dint \limits_{-\infty}^{z}}%
%BeginExpansion
{\displaystyle\int\limits_{-\infty}^{z}}
%EndExpansion
\left[
%TCIMACRO{\dint \limits_{-\infty}^{+\infty}}%
%BeginExpansion
{\displaystyle\int\limits_{-\infty}^{+\infty}}
%EndExpansion
\alpha f_{X}(\alpha)f_{Y}(z\alpha)d\alpha\right]  d\beta
\end{align*}
and then%
\[
f_{Z}(z)=%
%TCIMACRO{\dint \limits_{-\infty}^{\infty}}%
%BeginExpansion
{\displaystyle\int\limits_{-\infty}^{\infty}}
%EndExpansion
\alpha f_{XY}(\alpha,z\alpha)d\alpha
\]


\subsection{Uniform Distributions}%

%TCIMACRO{\TeXButton{\label{UD}}{\label{UD}}}%
%BeginExpansion
\label{UD}%
%EndExpansion


\bigskip%

\[
f_{X}(x)=\left\{
\begin{array}
[c]{cc}%
\frac{1}{b_{x}-a_{x}}, & x\in\lbrack a_{x},b_{x}]\\
0, & x\notin\lbrack a_{x},b_{x}]
\end{array}
\right.
\]%
\[
f_{Y}(y)=\left\{
\begin{array}
[c]{cc}%
\frac{1}{b_{y}-a_{y}}, & y\in\lbrack a_{y},b_{y}]\\
0, & y\notin\lbrack a_{y},b_{y}]
\end{array}
\right.
\]
%

\[
\frac{a_{y}}{b_{x}}\leq z\leq\frac{b_{y}}{a_{x}},0<a_{x}.
\]%
\begin{align*}
f_{Z}(z)  &  =\frac{1}{\left(  b_{x}-a_{x}\right)  \left(  b_{y}-a_{y}\right)
}%
%TCIMACRO{\dint \limits_{\max(a_{x},z^{-1}a_{y})}^{\min(b_{x},z^{-1}b_{y})}}%
%BeginExpansion
{\displaystyle\int\limits_{\max(a_{x},z^{-1}a_{y})}^{\min(b_{x},z^{-1}b_{y})}}
%EndExpansion
xdx\\
&  =\frac{\min(b_{x},z^{-1}b_{y})^{2}-\max(a_{x},z^{-1}a_{y})^{2}}{2\left(
b_{x}-a_{x}\right)  \left(  b_{y}-a_{y}\right)  },
\end{align*}%
\[%
%TCIMACRO{\dint \limits_{z=\frac{a_{y}}{b_{x}}}^{z=\frac{b_{y}}{a_{x}}}}%
%BeginExpansion
{\displaystyle\int\limits_{z=\frac{a_{y}}{b_{x}}}^{z=\frac{b_{y}}{a_{x}}}}
%EndExpansion
f_{z}(z)dz=1,0<a_{x}.
\]%
\[
f_{z}(z)=\left\{
\begin{array}
[c]{cc}%
\frac{\left(  b_{x}-a_{x}\right)  }{2\left(  b_{y}-a_{y}\right)  } &
\begin{array}
[c]{c}%
0<z,\min(b_{x},z^{-1}b_{y})=b_{x},\max(a_{x},z^{-1}a_{y})=a_{x},\\
\frac{a_{y}}{a_{x}}<z<\frac{b_{y}}{b_{x}}%
\end{array}
\\
\frac{b_{y}^{2}z^{-2}-a_{x}^{2}}{2\left(  b_{x}-a_{x}\right)  \left(
b_{y}-a_{y}\right)  } & \min(b_{x},z^{-1}b_{y})=z^{-1}b_{y},\max(a_{x}%
,z^{-1}a_{y})=a_{x},\\
\frac{z^{-2}\left(  b_{y}-a_{y}\right)  }{2\left(  b_{x}-a_{x}\right)  } &
\min(b_{x},z^{-1}b_{y})=z^{-1}b_{y},\max(a_{x},z^{-1}a_{y})=z^{-1}a_{y},\\
0, & \text{otherwise.}%
\end{array}
\right.
\]
and%
\begin{align*}
f_{Z}(z)  &  =\frac{\min(b_{x},z^{-1}b_{y})^{2}}{2b_{x}b_{y}},a_{x}%
=0,a_{y}=0,\\
f_{z}(z)  &  =\left\{
\begin{array}
[c]{cc}%
\frac{1}{2\delta} & 0<z,z<=\delta,\delta=\frac{b_{y}}{b_{x}},\\
\frac{\delta}{2}z^{-2} & \delta<=z.
\end{array}
\right.  ,\\
F(z)  &  =%
%TCIMACRO{\dint \limits_{0}^{z}}%
%BeginExpansion
{\displaystyle\int\limits_{0}^{z}}
%EndExpansion
f_{z}(z)dz=\left\{
\begin{array}
[c]{cc}%
\frac{z}{2\delta} & 0<z,z<=\delta,\\
1-\frac{\delta}{2}z^{-1} & \delta<=z.
\end{array}
\right.
\end{align*}


For example%
\[
a_{x}=1,b_{x}=2,a_{y}=3,b_{y}=7.
\]%
\[
f_{Z}(z)=\left\{
\begin{array}
[c]{cc}%
\frac{1}{2}-\frac{9}{8z^{2}} & \frac{3}{2}<z\leq3\\
\frac{3}{8} & 3<z\leq\frac{7}{2}\\
\frac{49}{8z^{2}}-\frac{1}{8} & \frac{7}{2}<z<7\\
0, & \text{otherwise.}%
\end{array}
\right.
\]
%

\[
\Omega(s)=\left\{
\begin{array}
[c]{cc}%
0, & s\leq0\\
\frac{1}{16}, & s=\log(2)\\
\frac{3}{8}, & s=\log(3)\\
\frac{23}{32}, & s=\log(4)\\
\frac{9}{10}, & s=\log(5)\\
\frac{47}{48}, & s=\log(6)\\
1, & \text{otherwise.}%
\end{array}
\right.
\]%
\[
\rho(k)=\Omega(\log(k+1))-\Omega(\log(k)),\Omega(1)-\Omega(0)=1.
\]
%

%TCIMACRO{\TeXButton{B}{\begin{table}[!htbp] \centering}}%
%BeginExpansion
\begin{table}[!htbp] \centering
%EndExpansion%
\begin{tabular}
[c]{|c|c|c|c|c|c|c|c|c|c|}\hline\hline
${\small k}$ & ${\small 1}$ & ${\small 2}$ & ${\small 3}$ & ${\small 4}$ &
${\small 5}$ & ${\small 6}$ & ${\small 7}$ & ${\small 8}$ & ${\small 9}%
$\\\hline
${\small \rho(k)}$ & $\frac{{\small 1}}{16}$ & $\frac{{\small 5}}{16}$ &
$\frac{{\small 11}}{32}$ & $\frac{{\small 29}}{160}$ & $\frac{{\small 19}%
}{240}$ & $\frac{{\small 1}}{48}$ & $0$ & ${\small 0}$ & ${\small 0}%
$\\\hline\hline
\end{tabular}
\caption{Table Caption 12}\label{TableKey12}%
%TCIMACRO{\TeXButton{E}{\end{table}}}%
%BeginExpansion
\end{table}%
%EndExpansion


\section{The Product of Two Random Positive Numbers from a UD}%

%TCIMACRO{\TeXButton{\label{P2RNUD}}{\label{P2RNUD}}}%
%BeginExpansion
\label{P2RNUD}%
%EndExpansion


Let $f_{X}(x)$ be a PDF for $0\leq x$ and $f_{Y}(y)$ a PDF for $0\leq y$. We
need to find the density of $z=xy$. If the two random variables are
independent, then in this case%

\[
f_{XY}(x,y)=f_{X}(x)f_{Y}(y)
\]
The distribution function for $Z$ is%
\[
F(z)=%
%TCIMACRO{\dint \limits^{z}}%
%BeginExpansion
{\displaystyle\int\limits^{z}}
%EndExpansion
f_{Z}(s)ds.
\]
Let us transform the variables%
\[
x=\alpha,y=\frac{\beta}{\alpha}.
\]
Then the Jacobian of the transformation is%
\[
J=\det\left(
\begin{array}
[c]{cc}%
\frac{\partial x}{\partial\alpha} & \frac{\partial y}{\partial\alpha}\\
\frac{\partial x}{\partial\beta} & \frac{\partial y}{\partial\beta}%
\end{array}
\right)  =\det\left(
\begin{array}
[c]{cc}%
1 & -\frac{\beta}{\alpha^{2}}\\
0 & \frac{1}{\alpha}%
\end{array}
\right)  =\frac{1}{\alpha}.
\]%
\[
f_{z}(z)=%
%TCIMACRO{\dint \limits_{-\infty}^{+\infty}}%
%BeginExpansion
{\displaystyle\int\limits_{-\infty}^{+\infty}}
%EndExpansion
\frac{1}{\alpha}f_{X}(\alpha)f_{Y}\left(  \frac{z}{\alpha}\right)  d\alpha
\]
In the case of a uniform distribution, we have%
\begin{align*}
f_{z}(z)  &  =\frac{\ln\left(  \min\left(  b_{x},\frac{z}{a_{y}}\right)
\right)  -\ln\left(  \max\left(  a_{x},\frac{z}{b_{y}}\right)  \right)
}{\left(  b_{x}-a_{x}\right)  \left(  b_{y}-a_{y}\right)  }.\\
a_{x}a_{y}  &  \leq z\leq b_{x}b_{y}.
\end{align*}%
\[%
%TCIMACRO{\dint \limits_{z=a_{x}a_{y}}^{z=b_{x}b_{y}}}%
%BeginExpansion
{\displaystyle\int\limits_{z=a_{x}a_{y}}^{z=b_{x}b_{y}}}
%EndExpansion
f_{z}(z)dz=1.
\]


When%
\begin{align*}
a_{x}  &  =0,a_{y}=0,\\
f_{z}(z)  &  =-\frac{\ln\left(  \frac{z}{b_{x}b_{y}}\right)  }{b_{x}b_{y}%
},0\leq z\leq b_{x}b_{y}%
\end{align*}
For example%
\[
a_{x}=1,b_{x}=2,a_{y}=3,b_{y}=7.
\]%
\[
f_{z}(z)=\left\{
\begin{array}
[c]{cc}%
\frac{\ln\left(  z\right)  -\ln\left(  3\right)  }{4} & 3<z\leq6\\
\frac{\ln(2)}{4} & 6<z\leq7\\
\frac{\ln(14)-\ln\left(  z\right)  }{4} & 7<z\leq14\\
0, & \text{otherwise.}%
\end{array}
\right.
\]%
\[%
%TCIMACRO{\dint \limits_{z=3}^{z=14}}%
%BeginExpansion
{\displaystyle\int\limits_{z=3}^{z=14}}
%EndExpansion
f_{z}(z)dz=1.
\]
For this distribution, the leading digits are $[1,3,4,5,6,7,8,9]$, and the
probability of the digit $2$ is $0$.

\bigskip%

%TCIMACRO{\TeXButton{B}{\begin{table}[!htbp] \centering}}%
%BeginExpansion
\begin{table}[!htbp] \centering
%EndExpansion%
\begin{tabular}
[c]{|c|c|c|c|c|c|c|c|c|c|}\hline\hline
${\small k}$ & ${\small 1}$ & ${\small 2}$ & ${\small 3}$ & ${\small 4}$ &
${\small 5}$ & ${\small 6}$ & ${\small 7}$ & ${\small 8}$ & ${\small 9}%
$\\\hline
${\small \rho(k)}$ & ${\small .1587}$ & ${\small 0}$ & ${\small .0337}$ &
${\small .1009}$ & ${\small .1511}$ & ${\small .1733}$ & ${\small .1563}$ &
${\small .1250}$ & ${\small .0970}$\\\hline\hline
\end{tabular}
\caption{Table Caption 13}\label{TableKey13}%
%TCIMACRO{\TeXButton{E}{\end{table}}}%
%BeginExpansion
\end{table}%
%EndExpansion


\subsection{Leading significant digits}%

%TCIMACRO{\TeXButton{\label{P2RNUDLSD}}{\label{P2RNUDLSD}}}%
%BeginExpansion
\label{P2RNUDLSD}%
%EndExpansion
%

\begin{align*}
a_{x}  &  =0,a_{y}=0,b_{x}=1,b_{y}=1,\\
f_{z}(z)  &  =-\ln\left(  z\right)  ,0\leq z\leq1,\\
F(z)  &  =z-z\ln\left(  z\right)  .
\end{align*}%
\[
\rho(k)=\frac{1}{9}\left(  k\ln(k)-(k+1)\ln(k+1)+1+\frac{10}{9}\ln(10)\right)
,1\leq k\leq9.
\]
%

\begin{align*}
\rho(k,L) &  =\frac{10^{-L}}{9}\left(  k\ln(k)-(k+1)\ln(k+1)+1+\frac{\left(
9L+10\right)  }{9}\ln(10)\right)  ,\\
10^{L} &  \leq k\leq10^{L+1}-1,L=0,1,\ldots.
\end{align*}%
\[%
%TCIMACRO{\dsum \limits_{k=10^{L}}^{10^{L+1}-1}}%
%BeginExpansion
{\displaystyle\sum\limits_{k=10^{L}}^{10^{L+1}-1}}
%EndExpansion
\rho(k,L)=1,L\in%
%TCIMACRO{\U{2124} }%
%BeginExpansion
\mathbb{Z}
%EndExpansion
^{\ast}.
\]


\section{Product of $n$ Numbers from a UD}%

%TCIMACRO{\TeXButton{\label{PRNUD}}{\label{PRNUD}}}%
%BeginExpansion
\label{PRNUD}%
%EndExpansion


\bigskip

We have%
\begin{align*}
f_{Z}(z,n) &  =\frac{(-\ln(z))^{in-1}}{\left(  n-1\right)  !},\\
F_{Z}(z,n) &  =\frac{\Gamma(n,-\ln(z))}{\Gamma(n)}=z%
%TCIMACRO{\dsum \limits_{i=0}^{n}}%
%BeginExpansion
{\displaystyle\sum\limits_{i=0}^{n}}
%EndExpansion
(-1)^{i}\frac{(\ln(z))^{i}}{i!},n\in%
%TCIMACRO{\U{2115} }%
%BeginExpansion
\mathbb{N}
%EndExpansion
.
\end{align*}%
\begin{align*}
\rho(k,n) &  =%
%TCIMACRO{\dsum \limits_{m=-\infty}^{-1}}%
%BeginExpansion
{\displaystyle\sum\limits_{m=-\infty}^{-1}}
%EndExpansion
\left(  F_{Z}(10^{m}(k+1))-F_{Z}(10^{m}k),n\right)  \\
&  =\Omega\left(  \log(k+1)-\left\lfloor \log(k)\right\rfloor \right)
-\Omega\left(  \log(k)-\left\lfloor \log(k)\right\rfloor \right)
\end{align*}


\bigskip From (\ref{EM00}) we have
\begin{align}
a  &  =-\infty,b=-1,\label{ProdN}\\
g(x,k,n)  &  =\frac{\Gamma(n,-\ln(10^{x}(k+1)))}{\Gamma(n)}-\frac
{\Gamma(n,-\ln(10^{x}k))}{\Gamma(n)},\nonumber\\
I_{1}(k,n)  &  =%
%TCIMACRO{\dint \limits_{x=-\infty}^{-1}}%
%BeginExpansion
{\displaystyle\int\limits_{x=-\infty}^{-1}}
%EndExpansion
g(x,k,n)dx,I_{2}=%
%TCIMACRO{\dint \limits_{x=-\infty}^{-1}}%
%BeginExpansion
{\displaystyle\int\limits_{x=-\infty}^{-1}}
%EndExpansion
\left(  x-\lfloor x\rfloor-\frac{1}{2}\right)  \frac{dg(x,k,n)}{dx}%
dx,\nonumber\\
I_{3}  &  =\frac{g(-1,k,n)}{2},I_{4}=\frac{g(-\infty,k,n)}{2}=0.\nonumber
\end{align}


\bigskip From the graph we have%

%TCIMACRO{\TeXButton{B}{\begin{figure}[H] \centering}}%
%BeginExpansion
\begin{figure}[H] \centering
%EndExpansion%
%TCIMACRO{\FRAME{itbpFU}{5.4509in}{3.9487in}{0in}{\Qcb{g(x,1,10) vs x}}%
%{}{integrant10.png}{\special{ language "Scientific Word";  type "GRAPHIC";
%maintain-aspect-ratio TRUE;  display "USEDEF";  valid_file "F";
%width 5.4509in;  height 3.9487in;  depth 0in;  original-width 5.393in;
%original-height 3.9003in;  cropleft "0";  croptop "0.997345";  cropright "1";
%cropbottom "-0.002655";  filename 'Integrant10.png';file-properties "XNPEU";}}
%}%
%BeginExpansion
{\parbox[b]{5.4509in}{\begin{center}
\includegraphics[
trim=0.000000in -0.010355in 0.000000in 0.010355in,
natheight=3.900300in,
natwidth=5.393000in,
height=3.9487in,
width=5.4509in
]%
{C:/Users/Vladimir/Documents/BENFORD/AAA/FOR ProofReading INTERNET/From PRS 4  (ver 9)/Integrant10__13.png}%
\\
g(x,1,10) vs x
\end{center}}}
%EndExpansion
\caption{$g(x,1,10)$ vs $x$}\label{figureKey14}%
%TCIMACRO{\TeXButton{E}{\end{figure}}}%
%BeginExpansion
\end{figure}%
%EndExpansion


\bigskip

The maximum of $g(x,k,n>>1)$\ is at the point $x=\zeta(n)$, where
$\zeta(n>>1)=-\frac{n}{\ln(10)}+o(n)$. Now we can, without loss of accuracy,
change the upper limit to $+\infty.$ With the knowledge of (\ref{EM2}) and
(\ref{Iform}),%
\begin{equation}
I_{1}(k,n\rightarrow\infty)=\log(k+1)-\log(k),\digamma=1. \label{Prof1}%
\end{equation}


We can prove that the integrals as $n\rightarrow+\infty$ have the limits%
\[
I_{2}(k,+\infty)=0,I_{3}(k,+\infty)=0.
\]


This is the NBL Distribution.

\section{Derivation of Formula (\ref{EM00})}

\bigskip%
%TCIMACRO{\TeXButton{\label{DerForm}}{\label{DerForm}}}%
%BeginExpansion
\label{DerForm}%
%EndExpansion


If we have integers $a$ and $b$ ($a<b$), then for an integer $j$ we have,
after integration by parts,
\begin{align*}%
%TCIMACRO{\dint \limits_{j}^{j+1}}%
%BeginExpansion
{\displaystyle\int\limits_{j}^{j+1}}
%EndExpansion
g(x)dx  &  =%
%TCIMACRO{\dint \limits_{j}^{j+1}}%
%BeginExpansion
{\displaystyle\int\limits_{j}^{j+1}}
%EndExpansion
g(x)\frac{d\left(  x-j-\frac{1}{2}\right)  }{dx}dx\\
&  =\frac{g(j+1)}{2}+\frac{g(j)}{2}-%
%TCIMACRO{\dint \limits_{x=j}^{j+1}}%
%BeginExpansion
{\displaystyle\int\limits_{x=j}^{j+1}}
%EndExpansion
\left(  x-j-\frac{1}{2}\right)  \frac{dg(x)}{dx}dx,\\
\left(  x-j-\frac{1}{2}\right)   &  =x-\lfloor x\rfloor-\frac{1}{2}.
\end{align*}
Then after summation%
\begin{align*}%
%TCIMACRO{\dsum \limits_{j=a}^{b-1}}%
%BeginExpansion
{\displaystyle\sum\limits_{j=a}^{b-1}}
%EndExpansion%
%TCIMACRO{\dint \limits_{j}^{j+1}}%
%BeginExpansion
{\displaystyle\int\limits_{j}^{j+1}}
%EndExpansion
g(x)dx  &  =%
%TCIMACRO{\dint \limits_{a}^{b}}%
%BeginExpansion
{\displaystyle\int\limits_{a}^{b}}
%EndExpansion
g(x)dx,\\%
%TCIMACRO{\dint \limits_{a}^{b}}%
%BeginExpansion
{\displaystyle\int\limits_{a}^{b}}
%EndExpansion
g(x)dx  &  =%
%TCIMACRO{\dsum \limits_{j=a}^{b}}%
%BeginExpansion
{\displaystyle\sum\limits_{j=a}^{b}}
%EndExpansion
g(j)-\frac{g(b)}{2}-\frac{g(a)}{2}-%
%TCIMACRO{\dint \limits_{x=a}^{b}}%
%BeginExpansion
{\displaystyle\int\limits_{x=a}^{b}}
%EndExpansion
\left(  x-\lfloor x\rfloor-\frac{1}{2}\right)  \frac{dg(x)}{dx}dx.
\end{align*}
Then%
\[%
%TCIMACRO{\dsum \limits_{j=a}^{b}}%
%BeginExpansion
{\displaystyle\sum\limits_{j=a}^{b}}
%EndExpansion
g(j)=%
%TCIMACRO{\dint \limits_{a}^{b}}%
%BeginExpansion
{\displaystyle\int\limits_{a}^{b}}
%EndExpansion
\left(  g(x)+\left(  x-\lfloor x\rfloor-\frac{1}{2}\right)  \frac{dg(x)}%
{dx}\right)  dx+\frac{g(b)}{2}+\frac{g(a)}{2}.
\]


\section{Useful Relations}

\bigskip

\begin{itemize}
\item For any non-negative function $f(x)$ with \
\[
0<\digamma=%
%TCIMACRO{\dint \limits_{0}^{\infty}}%
%BeginExpansion
{\displaystyle\int\limits_{0}^{\infty}}
%EndExpansion
f(t)dt<const
\]
and positive $a$ and $b$ we have, after integration by parts applied to the
left integral,
\[%
%TCIMACRO{\dint \limits_{-\infty}^{+\infty}}%
%BeginExpansion
{\displaystyle\int\limits_{-\infty}^{+\infty}}
%EndExpansion%
%TCIMACRO{\dint \limits_{10^{x}a}^{10^{x}b}}%
%BeginExpansion
{\displaystyle\int\limits_{10^{x}a}^{10^{x}b}}
%EndExpansion
f(t)dtdx=I_{b}-I_{a}.
\]%
\[
I_{a}=-a\ln(10)%
%TCIMACRO{\dint \limits_{-\infty}^{+\infty}}%
%BeginExpansion
{\displaystyle\int\limits_{-\infty}^{+\infty}}
%EndExpansion
x10^{x}f(10^{x}a)dx
\]%
\begin{align*}
I_{a}  &  =%
%TCIMACRO{\dint \limits_{0}^{+\infty}}%
%BeginExpansion
{\displaystyle\int\limits_{0}^{+\infty}}
%EndExpansion
f(t)(\log(a)-\log(t))dt,x=\log\left(  \frac{t}{a}\right)  ,\\
I_{b}  &  =%
%TCIMACRO{\dint \limits_{0}^{+\infty}}%
%BeginExpansion
{\displaystyle\int\limits_{0}^{+\infty}}
%EndExpansion
f(t)(\log(b)-\log(t))dt.
\end{align*}%
\begin{align*}
I_{b}-I_{a}  &  =(\log(b)-\log(a))%
%TCIMACRO{\dint \limits_{0}^{+\infty}}%
%BeginExpansion
{\displaystyle\int\limits_{0}^{+\infty}}
%EndExpansion
f(t)dt\\
&  =\digamma(\log(b)-\log(a)).
\end{align*}


\item For any real $x$ and $y$ and integer $n$ we have%
\begin{align}
\lfloor x\rfloor+\lfloor y\rfloor &  \leq\lfloor x+y\rfloor\leq\lfloor
x\rfloor+\lfloor y\rfloor+1,\label{Prop2}\\
\lfloor x+n\rfloor &  =\lfloor x\rfloor+n,\nonumber\\
\lceil x\rceil-\lfloor x\rfloor &  =\left\{
\begin{array}
[c]{cc}%
0, & x\in%
%TCIMACRO{\U{2124} }%
%BeginExpansion
\mathbb{Z}
%EndExpansion
,\\
1, & x\notin%
%TCIMACRO{\U{2124} }%
%BeginExpansion
\mathbb{Z}
%EndExpansion
.
\end{array}
\right.  ,\\
\lfloor x\rfloor+\lfloor-x\rfloor &  =\left\{
\begin{array}
[c]{cc}%
0, & x\in%
%TCIMACRO{\U{2124} }%
%BeginExpansion
\mathbb{Z}
%EndExpansion
,\\
-1, & x\notin%
%TCIMACRO{\U{2124} }%
%BeginExpansion
\mathbb{Z}
%EndExpansion
.
\end{array}
\right.  .
\end{align}

\end{itemize}

For every real number $x$ and for every positive integer $n$ the following
identity holds (Hermite's identity)%
\[
\lfloor nx\rfloor=%
%TCIMACRO{\dsum \limits_{i=0}^{n-1}}%
%BeginExpansion
{\displaystyle\sum\limits_{i=0}^{n-1}}
%EndExpansion
\left\lfloor x+\frac{i}{n}\right\rfloor .
\]
For a positive integer $n$, and arbitrary real numbers $m$ and $x$ ($x$ not an
integer)$\lfloor$%
\[
\left\lfloor \frac{\left\lfloor \frac{x}{m}\right\rfloor }{n}\right\rfloor
=\left\lfloor \frac{x}{mn}\right\rfloor
\]%
\begin{equation}
\lfloor x\rfloor=x-\frac{1}{2}+\frac{1}{\pi}%
%TCIMACRO{\dsum \limits_{j=1}^{\infty}}%
%BeginExpansion
{\displaystyle\sum\limits_{j=1}^{\infty}}
%EndExpansion
\frac{\sin(2\pi jx)}{j}. \label{Floor1}%
\end{equation}


Other presentation%
\[
\lfloor z\rfloor=z-\frac{1}{2}+\left\{
\begin{array}
[c]{cc}%
\frac{1}{2}, & z\in%
%TCIMACRO{\U{2124} }%
%BeginExpansion
\mathbb{Z}
%EndExpansion
,\\
0. & z=\frac{n}{2},n\in%
%TCIMACRO{\U{2124} }%
%BeginExpansion
\mathbb{Z}
%EndExpansion
,\\
\frac{1}{\pi}\arctan\left(  \cot(\pi z)\right)  , & otherwise.
\end{array}
\right.
\]
or%
\[
\lfloor z\rfloor=z-\frac{1}{2}+\left\{
\begin{array}
[c]{cc}%
\frac{1}{2}, & z\in%
%TCIMACRO{\U{2124} }%
%BeginExpansion
\mathbb{Z}
%EndExpansion
,\\
-\frac{1}{\pi}\arctan\left(  \tan\left(  \pi\left(  z-\frac{1}{2}\right)
\right)  \right)  , & z\notin%
%TCIMACRO{\U{2124} }%
%BeginExpansion
\mathbb{Z}
%EndExpansion
.
\end{array}
\right.
\]%
\[
\lceil z\rceil=z+\frac{1}{2}+\left\{
\begin{array}
[c]{cc}%
-\frac{1}{2}, & z\in%
%TCIMACRO{\U{2124} }%
%BeginExpansion
\mathbb{Z}
%EndExpansion
,\\
-\frac{1}{\pi}\arctan\left(  \tan\left(  \pi\left(  z-\frac{1}{2}\right)
\right)  \right)  , & z\notin%
%TCIMACRO{\U{2124} }%
%BeginExpansion
\mathbb{Z}
%EndExpansion
.
\end{array}
\right.
\]%
\begin{align*}
\cot(x)  &  =\tan\left(  \frac{\pi}{2}-x\right) \\
&  =-\tan\left(  x-\frac{\pi}{2}\right)  .
\end{align*}%
\begin{align*}
\arctan\left(  \frac{1}{x}\right)   &  =\operatorname{arccot}(x),x>0.\\
\arctan\left(  \frac{1}{x}\right)   &  =\frac{\pi}{2}-\arctan\left(  x\right)
,x>0.
\end{align*}%
\begin{align*}
\arctan\left(  \frac{1}{x}\right)   &  =\operatorname{arccot}(x),x<0.\\
\arctan\left(  \frac{1}{x}\right)   &  =-\frac{\pi}{2}-\arctan\left(
x\right)  ,x>0.
\end{align*}%
\begin{align*}
\arctan\left(  x\right)   &  =2\arctan\left(  \frac{x}{1+\sqrt{1-x^{2}}%
}\right)  ,\\
\arctan\left(  x\right)   &  =\arcsin\left(  \frac{x}{\sqrt{1-x^{2}}}\right)
,
\end{align*}%
\[
\tan(a+b)=\frac{\tan(a)+\tan(b)}{1-\tan(a)\tan(b)}.
\]%
\[
\arctan\left(  -x\right)  =-\arctan\left(  x\right)  .
\]%
\[
\frac{d\arctan\left(  x\right)  }{dx}=\frac{1}{1+x^{2}}.
\]%
\[
\arctan\left(  z\right)  =\frac{\imath}{2}\left(  \ln(1-\imath z)-\ln(1+\imath
z)\right)  .
\]%
\[
\arctan\left(  a\right)  -\arctan\left(  b\right)  =\arctan\left(  \frac
{a-b}{1+ab}\right)  .
\]%
\[
\tan\left(  a+b\right)  =\arctan\left(  \frac{\tan\left(  a\right)
+\tan\left(  b\right)  }{1-\tan\left(  a\right)  \tan\left(  b\right)
}\right)  .
\]
From RHS of (\ref{Floor1}) we have%
\begin{align}
&  x-\frac{1}{2}+\frac{1}{\pi}%
%TCIMACRO{\dsum \limits_{j=1}^{\infty}}%
%BeginExpansion
{\displaystyle\sum\limits_{j=1}^{\infty}}
%EndExpansion
\frac{\sin(2\pi jx)}{j}\\
&  =x-\frac{1}{2}+%
%TCIMACRO{\dsum \limits_{j=1}^{\infty}}%
%BeginExpansion
{\displaystyle\sum\limits_{j=1}^{\infty}}
%EndExpansion
\frac{1}{2\pi j\imath}\left(  \exp(2\pi jx\imath)-\exp(-2\pi jx\imath)\right)
.
\end{align}


after summation we have%
\begin{align*}%
%TCIMACRO{\dsum \limits_{j=1}^{\infty}}%
%BeginExpansion
{\displaystyle\sum\limits_{j=1}^{\infty}}
%EndExpansion
\frac{\sin(2\pi jx)}{j}  &  =\frac{1}{2\pi\imath}\left(  -\ln\left(
1-\exp(-2\pi x\imath)\right)  +\ln\left(  1-\exp(2\pi x\imath)\right)  \right)
\\
&  =\frac{1}{\pi}\arctan(\cot(\pi x)),x\notin%
%TCIMACRO{\U{2124} }%
%BeginExpansion
\mathbb{Z}
%EndExpansion
.
\end{align*}


\subsection{Code}

\medskip Here is R code

\texttt{V%
%TCIMACRO{\TEXTsymbol{<}}%
%BeginExpansion
$<$%
%EndExpansion
-function(s,n)\{if(s==floor(s)) return(s);\newline j%
%TCIMACRO{\TEXTsymbol{<}}%
%BeginExpansion
$<$%
%EndExpansion
-c(1:n); a%
%TCIMACRO{\TEXTsymbol{<}}%
%BeginExpansion
$<$%
%EndExpansion
-s-1/2+sum(sin(2*pi*j*s)/(pi*j));\newline return(a)\}}

\medskip

\texttt{VV%
%TCIMACRO{\TEXTsymbol{<}}%
%BeginExpansion
$<$%
%EndExpansion
-function(k,x,n)\{s%
%TCIMACRO{\TEXTsymbol{<}}%
%BeginExpansion
$<$%
%EndExpansion
-log10(x/k);\newline\ s1%
%TCIMACRO{\TEXTsymbol{<}}%
%BeginExpansion
$<$%
%EndExpansion
-log10(x/(k+1));\newline V(s,n)-V(s1,n)\}}

\medskip

\texttt{VVV%
%TCIMACRO{\TEXTsymbol{<}}%
%BeginExpansion
$<$%
%EndExpansion
-function(k,Sample,n)\{N%
%TCIMACRO{\TEXTsymbol{<}}%
%BeginExpansion
$<$%
%EndExpansion
-length(Sample);\newline for(j in 1:N)\{a[j]%
%TCIMACRO{\TEXTsymbol{<}}%
%BeginExpansion
$<$%
%EndExpansion
-VV(k,Sample[j],n)\};\newline return(sum(a)/N)\}\medskip}

Where \ \texttt{Sample} is list of positive numbers, \texttt{n}\ is positive integer.

\begin{acknowledgement}
I am grateful to Prof. T. Hill for pointing out to me the Benford Online
Bibliography. I thank Dr. F. Benford, Dr. S. Miller and Alex E. Kossovsky for
useful discussions and comments.
\end{acknowledgement}

\bigskip%
%TCIMACRO{\TeXButton{References}{\begin{thebibliography}{1}
%\bibitem{newcombs} S.Newcombs {\em
%Note on the frequencey of use of the different
%digits in natural numbers } 1881: Amer. J. Math. 4 39-40.
%\bibitem{benford} F.Benford {\em The law of anomalous numbers.}
%1938: Proceedings of the American Philosophical Society. 78 551-572.
%\bibitem{pinkham} R.S. Pinkham  {\em
%On the Distribution of First Significant Digits.}
%1961: Ann. Math. Statist. Volume 32, Number 4, 1223-1230.
%\bibitem{knuth} D.Knuth, {\em The Art of Computer Programming}
%1965: 2 219-229.
%Addison-Wesley, Reading, MA.
%\bibitem{wabramowitzstegun} M.Abramowitz, I.A.Stegun (Eds.){\em
%Handbook of Mathematical Functions with Formulas, Graphs, and Mathematical Tables}%
%,9th printing.
%1972: New York: Dover, pp. 16 and 806.
%\bibitem{diaconis} P.Diaconis {\em
%The distribution of leading digist and uniform distribution mod 1.}
%1977: Ann. Probab. 5 72-81.
%\bibitem{raimi} R.Raimi {\em The first digit
%phenomenon again.} 1985: Proceedings
%of the American Philosophical Society 129 211-219.
%\bibitem{nigrini1} M.J.Nigrini {\em The Detection of Income Tax Evasion
%Through an Analysis of Digital Distributions}
%1993. Ph.D. Dissertation, Univ. of Cincinnatti.
%\bibitem{knuth2} R.L. Graham, D.E. Knuth,O.Patashnik
%{\em Concrete Mathematics: A Foundation for Computer Science. 2nd Edition}
%1994: Addison-Wesley Professional,
%pp.672.
%\bibitem{hill} T.Hill,{ \em The significant-digit phenomenon }
%1995: Amer. Math.
%Monthly 102 322-327.
%\bibitem{nigrini2} M.J.Nigrini {\em
%A Taxpayer Compliance Application of Benford's Law.}
%1996: Journal of the American Tax Association, Spring 1996 pp 72-91.
%\bibitem{arnold0} V.I.Arnold {\em
%Statistics of first digits of degries of 2 and redivision of world.}
%1998: Kvant, 1.(in Russian)
%\bibitem{arnold1} V.I.Arnold
%{\em The antiscientifical revolution and mathematics}
%Talk at the meeting of the Pontifical Academy at Vatican,
%26 October 1998, Changing concepts of nature at the turn of the millennium.
%\bibitem{walck} C.Walck, {\em Handbook on
%Statistical Distributions
%for experimentalists } 2007.
%Particle Physics Group, Fysikum
%University of Stockholm..
%http://www.stat.rice.edu/~dobelman/textfiles/DistributionsHandbook.pdf
%\bibitem{formann0} A.K.Formann, Richard James ed.
%{\em"The Newcomb-Benford Law in Its Relation to Some Common Distributions". }
%2010:PLoS ONE. 5 (5): e10541. Bibcode:2010PLoSO.
%510541F. doi:10.1371/journal.pone.0010541. PMC 2866333free to read. PMID 2047987
%\bibitem{bergerhill} HA.Berger,T.P.Hill,
%{\em Benford Online Bibliography }
%2009: accessed May 14, 2010, at h   http://www.benfordonline.net.
%\bibitem{nigrini} M.Nigrini
%{\em
%Benford's Law: Applications for Forensic Accounting, Auditing, and Fraud Detection .}
%2012:Princeton University Press ,
%ISBN:  978-1118152850
%330 pp.
%\bibitem{arnotheodore} Arno Berger , Theodore P. Hill
%{\em An Introduction to Benford's Law.} 2015:
%Princeton University Press
%SBN: 9780691163062
%256 pp.
%\bibitem{miller} Steven J. Miller ed.
%{\em Benford's Law: Theory and Applications.}
%2015:Princeton University Press ,
%ISBN: 978-0-691-14761-1:
%464 pp.
%\end{thebibliography}}}%
%BeginExpansion
\begin{thebibliography}{1}
\bibitem{newcombs} S.Newcombs {\em
Note on the frequencey of use of the different
digits in natural numbers } 1881: Amer. J. Math. 4 39-40.
\bibitem{benford} F.Benford {\em The law of anomalous numbers.}
1938: Proceedings of the American Philosophical Society. 78 551-572.
\bibitem{pinkham} R.S. Pinkham  {\em
On the Distribution of First Significant Digits.}
1961: Ann. Math. Statist. Volume 32, Number 4, 1223-1230.
\bibitem{knuth} D.Knuth, {\em The Art of Computer Programming}
1965: 2 219-229.
Addison-Wesley, Reading, MA.
\bibitem{wabramowitzstegun} M.Abramowitz, I.A.Stegun (Eds.){\em
Handbook of Mathematical Functions with Formulas, Graphs, and Mathematical Tables}%
,9th printing.
1972: New York: Dover, pp. 16 and 806.
\bibitem{diaconis} P.Diaconis {\em
The distribution of leading digist and uniform distribution mod 1.}
1977: Ann. Probab. 5 72-81.
\bibitem{raimi} R.Raimi {\em The first digit
phenomenon again.} 1985: Proceedings
of the American Philosophical Society 129 211-219.
\bibitem{nigrini1} M.J.Nigrini {\em The Detection of Income Tax Evasion
Through an Analysis of Digital Distributions}
1993. Ph.D. Dissertation, Univ. of Cincinnatti.
\bibitem{knuth2} R.L. Graham, D.E. Knuth,O.Patashnik
{\em Concrete Mathematics: A Foundation for Computer Science. 2nd Edition}
1994: Addison-Wesley Professional,
pp.672.
\bibitem{hill} T.Hill,{ \em The significant-digit phenomenon }
1995: Amer. Math.
Monthly 102 322-327.
\bibitem{nigrini2} M.J.Nigrini {\em
A Taxpayer Compliance Application of Benford's Law.}
1996: Journal of the American Tax Association, Spring 1996 pp 72-91.
\bibitem{arnold0} V.I.Arnold {\em
Statistics of first digits of degries of 2 and redivision of world.}
1998: Kvant, 1.(in Russian)
\bibitem{arnold1} V.I.Arnold
{\em The antiscientifical revolution and mathematics}
Talk at the meeting of the Pontifical Academy at Vatican,
26 October 1998, Changing concepts of nature at the turn of the millennium.
\bibitem{walck} C.Walck, {\em Handbook on
Statistical Distributions
for experimentalists } 2007.
Particle Physics Group, Fysikum
University of Stockholm..
http://www.stat.rice.edu/~dobelman/textfiles/DistributionsHandbook.pdf
\bibitem{formann0} A.K.Formann, Richard James ed.
{\em"The Newcomb-Benford Law in Its Relation to Some Common Distributions". }
2010:PLoS ONE. 5 (5): e10541. Bibcode:2010PLoSO.
510541F. doi:10.1371/journal.pone.0010541. PMC 2866333free to read. PMID 2047987
\bibitem{bergerhill} HA.Berger,T.P.Hill,
{\em Benford Online Bibliography }
2009: accessed May 14, 2010, at h   http://www.benfordonline.net.
\bibitem{nigrini} M.Nigrini
{\em
Benford's Law: Applications for Forensic Accounting, Auditing, and Fraud Detection .}
2012:Princeton University Press ,
ISBN:  978-1118152850
330 pp.
\bibitem{arnotheodore} Arno Berger , Theodore P. Hill
{\em An Introduction to Benford's Law.} 2015:
Princeton University Press
SBN: 9780691163062
256 pp.
\bibitem{miller} Steven J. Miller ed.
{\em Benford's Law: Theory and Applications.}
2015:Princeton University Press ,
ISBN: 978-0-691-14761-1:
464 pp.
\end{thebibliography}%
%EndExpansion



\end{document}